\section{The Very Basics}
This section is intended to help you revise same fundamental algebra concepts and make you confortable with the notation conventions we will use. 


\subsection{$\mathcal{Y}$ as an explicit linear function of $\mathcal{X}$: the 1-D case}

Let's start with a very simple scenario. We are given a table with the values of one event of interest - $\mathcal{Y}$ - as a function of another variable we can observe, $\mathcal{X}$. 

\begin{align}
\begin{tabular}{c|c}
$\mathcal{Y}$ & $\mathcal{X}$ \\
\hline
$y_1$ & $x_{11}$\\
$y_2$ & $x_{21}$\\
\end{tabular}
\label{table_1d_abstract}
\end{align}

Here we are assuming the $\mathcal{X}$ observations are made over a single attribute, $x_{i1}$. We are using the double subscript notation to make more explicit the fact that we are looking at the first attribute - and so far the only one - of the $\mathcal{X}$ variables. This will allow us to more easily generalize many of the equations to those cases where the $\mathcal{X}$ variables have multiple attributes.\\

Suppose that given the two data points in Table \ref{table_1d_abstract} we are asked to ``predict" the value of another yet unknown $\mathcal{Y}$, $y_u$, for which we have access to the value of the attribute $x_{u1}$. Let us also assume that $\mathcal{Y}$ is a linear function of $\mathcal{X}$. Figure \ref{fig.simple_line} illustrates this relation. Without loss of generality we opted for placing $x_{u1}$ in between $x_{11}$ and $x_{21}$.\\

\begin {figure}[H]
\begin{center}
  \input{chapter_1/line_example}
\end{center}
\caption{Plotting the points from Table \ref{table_1d_abstract}}
\label{fig.simple_line}
\end {figure}

By looking at Figure \ref{fig.simple_line}, we see we can express $y_u$ as:

\begin{equation}
y_u = y_0 + x_{u1} \cdot \tan\alpha_1
\label{eq.simple_linear}
\end{equation}
but we need the values of both $y_0$ and $\tan\alpha$. The value of $\tan\alpha$ is actually quite easy to compute and is given by:
\begin{equation}
\tan\alpha_1 = \frac{y_2 - y_1}{x_{21} - x_{11}}
\label{eq.simple_tangent}
\end{equation}
The value of $y_0$ can now be computed by making $x = 0$, and solving equation \ref{eq.simple_linear} for $y_0$ for any of the known (x,y) values. For example, solving for $y_u = y_1$ we get:
\begin{equation}
y_0 = y_1 - x_{11} * \tan\alpha_1
\end{equation}
The $y_0$ value is often called \emph{the intersect}, and will appear in many of the equations that follow.

\subsubsection{Expressing it in the matrix notation}
An alternative to maintaining one separate equation for the tangent and another for the intercept consists in observing that Equation \ref{eq.simple_linear} can be compatctly rewritten using a vector notation. Let's start by observing that:
\begin{equation}
y_u =  
1 \cdot y_0 + x_u \cdot \tan\alpha_1
=
\begin{bmatrix}
1 &
x_{u1}
\end{bmatrix}
\cdot
\begin{bmatrix}
y_0 \\ 
\tan\alpha_1
\end{bmatrix} 
\end{equation}
So, if we make 
\begin{equation}
X = 
\begin{bmatrix}
1 & x_{u1}
\end{bmatrix}
\end{equation}
and
\begin{equation}
\begin{bmatrix}
y_0 \\
\tan\alpha_1
\end{bmatrix}
=
\begin{bmatrix}
w_0 \\
w_1
\end{bmatrix}
= w
\end{equation}
we get this very compact representation:
\begin{equation}
Y = X w
\end{equation}

Let us then represent the points we were given in Table \ref{table_1d_abstract}  - $y_1$ and $y_2$ - using this vector notation. Since
\begin{equation}
y_1 =
1 \cdot y_0 + x_{11} \cdot \tan\alpha_1
\end{equation}
and
\begin{equation}
y_2 =
1 \cdot y_0 + x_{21} \cdot \tan\alpha_1
\end{equation}
we can represent each $y_i$ as an entry in a vertical vector Y, which allow us to write both equation as:
\begin{equation}
Y = \begin{bmatrix}
y_1 \\
y_2
\end{bmatrix}
=
\begin{bmatrix}
1 \\
1
\end{bmatrix}
\cdot y_0
+
\begin{bmatrix}
x_{11} \\
x_{21}
\end{bmatrix}
\cdot 
\tan\alpha_1
\label{equation_comb_of_dimension_vectors}
\end{equation}
This leads to this very compact representation of our initial data set:
\begin{equation}
\begin{bmatrix}
y_1 \\
y_2
\end{bmatrix}
=
\begin{bmatrix}
1 & x_{11} \\
1 & x_{21}
\end{bmatrix}
\cdot
\begin{bmatrix}
y_0 \\ 
\tan\alpha_1
\end{bmatrix} = X \cdot w
\end{equation}

It is now clear that if we have two data points (our ''training points"), we can compute the $w$ vector very simply by solving the linear system:
\begin{equation}
Y = X \cdot w
\end{equation}

and ''learn" the weight vector w that would allow us to make ""predictions" about other x points. The solution can be obtained by inversion of the X matrix.  Multiplying both sides of the equation by $X^{-1}$

\begin{equation}
X^{-1} Y = X^{-1}X \cdot w
\end{equation}
we get
\begin{equation}
w = X^{-1} Y
\label{slope_equation}
\end{equation}

\subsubsection{Show me the numbers!}
Let us now exemplify what we have seen so far. Assume then that we instantiate the X Y values presented in Table \ref{table_1d_abstract} to obtain the following table:

\begin{align}
\begin{tabular}{c|c}
$\mathcal{Y}$ & $\mathcal{X}$ \\
\hline
5 & 2\\
13 & 6\\
\end{tabular}
\label{table_1d_example}
\end{align}
\begin {figure}[H]
\begin{center}
  % GNUPLOT: LaTeX picture
\setlength{\unitlength}{0.240900pt}
\ifx\plotpoint\undefined\newsavebox{\plotpoint}\fi
\sbox{\plotpoint}{\rule[-0.200pt]{0.400pt}{0.400pt}}%
\begin{picture}(1500,900)(0,0)
\sbox{\plotpoint}{\rule[-0.200pt]{0.400pt}{0.400pt}}%
\put(110.0,82.0){\rule[-0.200pt]{4.818pt}{0.400pt}}
\put(90,82){\makebox(0,0)[r]{-2}}
\put(1419.0,82.0){\rule[-0.200pt]{4.818pt}{0.400pt}}
\put(110.0,168.0){\rule[-0.200pt]{4.818pt}{0.400pt}}
\put(90,168){\makebox(0,0)[r]{ 0}}
\put(1419.0,168.0){\rule[-0.200pt]{4.818pt}{0.400pt}}
\put(110.0,255.0){\rule[-0.200pt]{4.818pt}{0.400pt}}
\put(90,255){\makebox(0,0)[r]{ 2}}
\put(1419.0,255.0){\rule[-0.200pt]{4.818pt}{0.400pt}}
\put(110.0,341.0){\rule[-0.200pt]{4.818pt}{0.400pt}}
\put(90,341){\makebox(0,0)[r]{ 4}}
\put(1419.0,341.0){\rule[-0.200pt]{4.818pt}{0.400pt}}
\put(110.0,427.0){\rule[-0.200pt]{4.818pt}{0.400pt}}
\put(90,427){\makebox(0,0)[r]{ 6}}
\put(1419.0,427.0){\rule[-0.200pt]{4.818pt}{0.400pt}}
\put(110.0,514.0){\rule[-0.200pt]{4.818pt}{0.400pt}}
\put(90,514){\makebox(0,0)[r]{ 8}}
\put(1419.0,514.0){\rule[-0.200pt]{4.818pt}{0.400pt}}
\put(110.0,600.0){\rule[-0.200pt]{4.818pt}{0.400pt}}
\put(90,600){\makebox(0,0)[r]{ 10}}
\put(1419.0,600.0){\rule[-0.200pt]{4.818pt}{0.400pt}}
\put(110.0,686.0){\rule[-0.200pt]{4.818pt}{0.400pt}}
\put(90,686){\makebox(0,0)[r]{ 12}}
\put(1419.0,686.0){\rule[-0.200pt]{4.818pt}{0.400pt}}
\put(110.0,773.0){\rule[-0.200pt]{4.818pt}{0.400pt}}
\put(90,773){\makebox(0,0)[r]{ 14}}
\put(1419.0,773.0){\rule[-0.200pt]{4.818pt}{0.400pt}}
\put(110.0,859.0){\rule[-0.200pt]{4.818pt}{0.400pt}}
\put(90,859){\makebox(0,0)[r]{ 16}}
\put(1419.0,859.0){\rule[-0.200pt]{4.818pt}{0.400pt}}
\put(110.0,82.0){\rule[-0.200pt]{0.400pt}{4.818pt}}
\put(110,41){\makebox(0,0){-1}}
\put(110.0,839.0){\rule[-0.200pt]{0.400pt}{4.818pt}}
\put(276.0,82.0){\rule[-0.200pt]{0.400pt}{4.818pt}}
\put(276,41){\makebox(0,0){ 0}}
\put(276.0,839.0){\rule[-0.200pt]{0.400pt}{4.818pt}}
\put(442.0,82.0){\rule[-0.200pt]{0.400pt}{4.818pt}}
\put(442,41){\makebox(0,0){ 1}}
\put(442.0,839.0){\rule[-0.200pt]{0.400pt}{4.818pt}}
\put(608.0,82.0){\rule[-0.200pt]{0.400pt}{4.818pt}}
\put(608,41){\makebox(0,0){ 2}}
\put(608.0,839.0){\rule[-0.200pt]{0.400pt}{4.818pt}}
\put(775.0,82.0){\rule[-0.200pt]{0.400pt}{4.818pt}}
\put(775,41){\makebox(0,0){ 3}}
\put(775.0,839.0){\rule[-0.200pt]{0.400pt}{4.818pt}}
\put(941.0,82.0){\rule[-0.200pt]{0.400pt}{4.818pt}}
\put(941,41){\makebox(0,0){ 4}}
\put(941.0,839.0){\rule[-0.200pt]{0.400pt}{4.818pt}}
\put(1107.0,82.0){\rule[-0.200pt]{0.400pt}{4.818pt}}
\put(1107,41){\makebox(0,0){ 5}}
\put(1107.0,839.0){\rule[-0.200pt]{0.400pt}{4.818pt}}
\put(1273.0,82.0){\rule[-0.200pt]{0.400pt}{4.818pt}}
\put(1273,41){\makebox(0,0){ 6}}
\put(1273.0,839.0){\rule[-0.200pt]{0.400pt}{4.818pt}}
\put(1439.0,82.0){\rule[-0.200pt]{0.400pt}{4.818pt}}
\put(1439,41){\makebox(0,0){ 7}}
\put(1439.0,839.0){\rule[-0.200pt]{0.400pt}{4.818pt}}
\put(1439,125){\makebox(0,0)[l]{$x_{1}$}}
\put(359,449){\makebox(0,0)[l]{($x_{11}=2, y_1=5$)}}
\put(1024,794){\makebox(0,0)[l]{($x_{21}=6, y_2=13$)}}
\put(691,622){\makebox(0,0)[l]{($x_{u1}=x_u, y_u=?$)}}
\put(775,417){\makebox(0,0)[l]{${\alpha}_{1}$}}
\put(226,229){\makebox(0,0)[l]{$y_0$}}
\put(110,168){\vector(1,0){1362}}
\put(276,82){\vector(0,1){712}}
\put(567,384){\line(1,0){747}}
\put(1273,363){\line(0,1){388}}
\put(110,125){\usebox{\plotpoint}}
\multiput(110.00,125.59)(0.950,0.485){11}{\rule{0.843pt}{0.117pt}}
\multiput(110.00,124.17)(11.251,7.000){2}{\rule{0.421pt}{0.400pt}}
\multiput(123.00,132.59)(1.026,0.485){11}{\rule{0.900pt}{0.117pt}}
\multiput(123.00,131.17)(12.132,7.000){2}{\rule{0.450pt}{0.400pt}}
\multiput(137.00,139.59)(0.950,0.485){11}{\rule{0.843pt}{0.117pt}}
\multiput(137.00,138.17)(11.251,7.000){2}{\rule{0.421pt}{0.400pt}}
\multiput(150.00,146.59)(1.026,0.485){11}{\rule{0.900pt}{0.117pt}}
\multiput(150.00,145.17)(12.132,7.000){2}{\rule{0.450pt}{0.400pt}}
\multiput(164.00,153.59)(0.950,0.485){11}{\rule{0.843pt}{0.117pt}}
\multiput(164.00,152.17)(11.251,7.000){2}{\rule{0.421pt}{0.400pt}}
\multiput(177.00,160.59)(1.026,0.485){11}{\rule{0.900pt}{0.117pt}}
\multiput(177.00,159.17)(12.132,7.000){2}{\rule{0.450pt}{0.400pt}}
\multiput(191.00,167.59)(0.950,0.485){11}{\rule{0.843pt}{0.117pt}}
\multiput(191.00,166.17)(11.251,7.000){2}{\rule{0.421pt}{0.400pt}}
\multiput(204.00,174.59)(0.950,0.485){11}{\rule{0.843pt}{0.117pt}}
\multiput(204.00,173.17)(11.251,7.000){2}{\rule{0.421pt}{0.400pt}}
\multiput(217.00,181.59)(1.026,0.485){11}{\rule{0.900pt}{0.117pt}}
\multiput(217.00,180.17)(12.132,7.000){2}{\rule{0.450pt}{0.400pt}}
\multiput(231.00,188.59)(0.950,0.485){11}{\rule{0.843pt}{0.117pt}}
\multiput(231.00,187.17)(11.251,7.000){2}{\rule{0.421pt}{0.400pt}}
\multiput(244.00,195.59)(1.026,0.485){11}{\rule{0.900pt}{0.117pt}}
\multiput(244.00,194.17)(12.132,7.000){2}{\rule{0.450pt}{0.400pt}}
\multiput(258.00,202.59)(0.950,0.485){11}{\rule{0.843pt}{0.117pt}}
\multiput(258.00,201.17)(11.251,7.000){2}{\rule{0.421pt}{0.400pt}}
\multiput(271.00,209.59)(1.026,0.485){11}{\rule{0.900pt}{0.117pt}}
\multiput(271.00,208.17)(12.132,7.000){2}{\rule{0.450pt}{0.400pt}}
\multiput(285.00,216.59)(0.950,0.485){11}{\rule{0.843pt}{0.117pt}}
\multiput(285.00,215.17)(11.251,7.000){2}{\rule{0.421pt}{0.400pt}}
\multiput(298.00,223.59)(0.950,0.485){11}{\rule{0.843pt}{0.117pt}}
\multiput(298.00,222.17)(11.251,7.000){2}{\rule{0.421pt}{0.400pt}}
\multiput(311.00,230.59)(1.026,0.485){11}{\rule{0.900pt}{0.117pt}}
\multiput(311.00,229.17)(12.132,7.000){2}{\rule{0.450pt}{0.400pt}}
\multiput(325.00,237.59)(0.950,0.485){11}{\rule{0.843pt}{0.117pt}}
\multiput(325.00,236.17)(11.251,7.000){2}{\rule{0.421pt}{0.400pt}}
\multiput(338.00,244.59)(1.026,0.485){11}{\rule{0.900pt}{0.117pt}}
\multiput(338.00,243.17)(12.132,7.000){2}{\rule{0.450pt}{0.400pt}}
\multiput(352.00,251.59)(0.950,0.485){11}{\rule{0.843pt}{0.117pt}}
\multiput(352.00,250.17)(11.251,7.000){2}{\rule{0.421pt}{0.400pt}}
\multiput(365.00,258.59)(0.950,0.485){11}{\rule{0.843pt}{0.117pt}}
\multiput(365.00,257.17)(11.251,7.000){2}{\rule{0.421pt}{0.400pt}}
\multiput(378.00,265.59)(1.026,0.485){11}{\rule{0.900pt}{0.117pt}}
\multiput(378.00,264.17)(12.132,7.000){2}{\rule{0.450pt}{0.400pt}}
\multiput(392.00,272.59)(0.950,0.485){11}{\rule{0.843pt}{0.117pt}}
\multiput(392.00,271.17)(11.251,7.000){2}{\rule{0.421pt}{0.400pt}}
\multiput(405.00,279.59)(1.026,0.485){11}{\rule{0.900pt}{0.117pt}}
\multiput(405.00,278.17)(12.132,7.000){2}{\rule{0.450pt}{0.400pt}}
\multiput(419.00,286.59)(0.950,0.485){11}{\rule{0.843pt}{0.117pt}}
\multiput(419.00,285.17)(11.251,7.000){2}{\rule{0.421pt}{0.400pt}}
\multiput(432.00,293.59)(1.026,0.485){11}{\rule{0.900pt}{0.117pt}}
\multiput(432.00,292.17)(12.132,7.000){2}{\rule{0.450pt}{0.400pt}}
\multiput(446.00,300.59)(0.950,0.485){11}{\rule{0.843pt}{0.117pt}}
\multiput(446.00,299.17)(11.251,7.000){2}{\rule{0.421pt}{0.400pt}}
\multiput(459.00,307.59)(0.950,0.485){11}{\rule{0.843pt}{0.117pt}}
\multiput(459.00,306.17)(11.251,7.000){2}{\rule{0.421pt}{0.400pt}}
\multiput(472.00,314.59)(1.026,0.485){11}{\rule{0.900pt}{0.117pt}}
\multiput(472.00,313.17)(12.132,7.000){2}{\rule{0.450pt}{0.400pt}}
\multiput(486.00,321.59)(1.123,0.482){9}{\rule{0.967pt}{0.116pt}}
\multiput(486.00,320.17)(10.994,6.000){2}{\rule{0.483pt}{0.400pt}}
\multiput(499.00,327.59)(1.026,0.485){11}{\rule{0.900pt}{0.117pt}}
\multiput(499.00,326.17)(12.132,7.000){2}{\rule{0.450pt}{0.400pt}}
\multiput(513.00,334.59)(0.950,0.485){11}{\rule{0.843pt}{0.117pt}}
\multiput(513.00,333.17)(11.251,7.000){2}{\rule{0.421pt}{0.400pt}}
\multiput(526.00,341.59)(1.026,0.485){11}{\rule{0.900pt}{0.117pt}}
\multiput(526.00,340.17)(12.132,7.000){2}{\rule{0.450pt}{0.400pt}}
\multiput(540.00,348.59)(0.950,0.485){11}{\rule{0.843pt}{0.117pt}}
\multiput(540.00,347.17)(11.251,7.000){2}{\rule{0.421pt}{0.400pt}}
\multiput(553.00,355.59)(0.950,0.485){11}{\rule{0.843pt}{0.117pt}}
\multiput(553.00,354.17)(11.251,7.000){2}{\rule{0.421pt}{0.400pt}}
\multiput(566.00,362.59)(1.026,0.485){11}{\rule{0.900pt}{0.117pt}}
\multiput(566.00,361.17)(12.132,7.000){2}{\rule{0.450pt}{0.400pt}}
\multiput(580.00,369.59)(0.950,0.485){11}{\rule{0.843pt}{0.117pt}}
\multiput(580.00,368.17)(11.251,7.000){2}{\rule{0.421pt}{0.400pt}}
\multiput(593.00,376.59)(1.026,0.485){11}{\rule{0.900pt}{0.117pt}}
\multiput(593.00,375.17)(12.132,7.000){2}{\rule{0.450pt}{0.400pt}}
\multiput(607.00,383.59)(0.950,0.485){11}{\rule{0.843pt}{0.117pt}}
\multiput(607.00,382.17)(11.251,7.000){2}{\rule{0.421pt}{0.400pt}}
\multiput(620.00,390.59)(1.026,0.485){11}{\rule{0.900pt}{0.117pt}}
\multiput(620.00,389.17)(12.132,7.000){2}{\rule{0.450pt}{0.400pt}}
\multiput(634.00,397.59)(0.950,0.485){11}{\rule{0.843pt}{0.117pt}}
\multiput(634.00,396.17)(11.251,7.000){2}{\rule{0.421pt}{0.400pt}}
\multiput(647.00,404.59)(0.950,0.485){11}{\rule{0.843pt}{0.117pt}}
\multiput(647.00,403.17)(11.251,7.000){2}{\rule{0.421pt}{0.400pt}}
\multiput(660.00,411.59)(1.026,0.485){11}{\rule{0.900pt}{0.117pt}}
\multiput(660.00,410.17)(12.132,7.000){2}{\rule{0.450pt}{0.400pt}}
\multiput(674.00,418.59)(0.950,0.485){11}{\rule{0.843pt}{0.117pt}}
\multiput(674.00,417.17)(11.251,7.000){2}{\rule{0.421pt}{0.400pt}}
\multiput(687.00,425.59)(1.026,0.485){11}{\rule{0.900pt}{0.117pt}}
\multiput(687.00,424.17)(12.132,7.000){2}{\rule{0.450pt}{0.400pt}}
\multiput(701.00,432.59)(0.950,0.485){11}{\rule{0.843pt}{0.117pt}}
\multiput(701.00,431.17)(11.251,7.000){2}{\rule{0.421pt}{0.400pt}}
\multiput(714.00,439.59)(1.026,0.485){11}{\rule{0.900pt}{0.117pt}}
\multiput(714.00,438.17)(12.132,7.000){2}{\rule{0.450pt}{0.400pt}}
\multiput(728.00,446.59)(0.950,0.485){11}{\rule{0.843pt}{0.117pt}}
\multiput(728.00,445.17)(11.251,7.000){2}{\rule{0.421pt}{0.400pt}}
\multiput(741.00,453.59)(0.950,0.485){11}{\rule{0.843pt}{0.117pt}}
\multiput(741.00,452.17)(11.251,7.000){2}{\rule{0.421pt}{0.400pt}}
\multiput(754.00,460.59)(1.026,0.485){11}{\rule{0.900pt}{0.117pt}}
\multiput(754.00,459.17)(12.132,7.000){2}{\rule{0.450pt}{0.400pt}}
\multiput(768.00,467.59)(0.950,0.485){11}{\rule{0.843pt}{0.117pt}}
\multiput(768.00,466.17)(11.251,7.000){2}{\rule{0.421pt}{0.400pt}}
\multiput(781.00,474.59)(1.026,0.485){11}{\rule{0.900pt}{0.117pt}}
\multiput(781.00,473.17)(12.132,7.000){2}{\rule{0.450pt}{0.400pt}}
\multiput(795.00,481.59)(0.950,0.485){11}{\rule{0.843pt}{0.117pt}}
\multiput(795.00,480.17)(11.251,7.000){2}{\rule{0.421pt}{0.400pt}}
\multiput(808.00,488.59)(0.950,0.485){11}{\rule{0.843pt}{0.117pt}}
\multiput(808.00,487.17)(11.251,7.000){2}{\rule{0.421pt}{0.400pt}}
\multiput(821.00,495.59)(1.026,0.485){11}{\rule{0.900pt}{0.117pt}}
\multiput(821.00,494.17)(12.132,7.000){2}{\rule{0.450pt}{0.400pt}}
\multiput(835.00,502.59)(0.950,0.485){11}{\rule{0.843pt}{0.117pt}}
\multiput(835.00,501.17)(11.251,7.000){2}{\rule{0.421pt}{0.400pt}}
\multiput(848.00,509.59)(1.026,0.485){11}{\rule{0.900pt}{0.117pt}}
\multiput(848.00,508.17)(12.132,7.000){2}{\rule{0.450pt}{0.400pt}}
\multiput(862.00,516.59)(0.950,0.485){11}{\rule{0.843pt}{0.117pt}}
\multiput(862.00,515.17)(11.251,7.000){2}{\rule{0.421pt}{0.400pt}}
\multiput(875.00,523.59)(1.026,0.485){11}{\rule{0.900pt}{0.117pt}}
\multiput(875.00,522.17)(12.132,7.000){2}{\rule{0.450pt}{0.400pt}}
\multiput(889.00,530.59)(0.950,0.485){11}{\rule{0.843pt}{0.117pt}}
\multiput(889.00,529.17)(11.251,7.000){2}{\rule{0.421pt}{0.400pt}}
\multiput(902.00,537.59)(0.950,0.485){11}{\rule{0.843pt}{0.117pt}}
\multiput(902.00,536.17)(11.251,7.000){2}{\rule{0.421pt}{0.400pt}}
\multiput(915.00,544.59)(1.026,0.485){11}{\rule{0.900pt}{0.117pt}}
\multiput(915.00,543.17)(12.132,7.000){2}{\rule{0.450pt}{0.400pt}}
\multiput(929.00,551.59)(0.950,0.485){11}{\rule{0.843pt}{0.117pt}}
\multiput(929.00,550.17)(11.251,7.000){2}{\rule{0.421pt}{0.400pt}}
\multiput(942.00,558.59)(1.026,0.485){11}{\rule{0.900pt}{0.117pt}}
\multiput(942.00,557.17)(12.132,7.000){2}{\rule{0.450pt}{0.400pt}}
\multiput(956.00,565.59)(0.950,0.485){11}{\rule{0.843pt}{0.117pt}}
\multiput(956.00,564.17)(11.251,7.000){2}{\rule{0.421pt}{0.400pt}}
\multiput(969.00,572.59)(1.026,0.485){11}{\rule{0.900pt}{0.117pt}}
\multiput(969.00,571.17)(12.132,7.000){2}{\rule{0.450pt}{0.400pt}}
\multiput(983.00,579.59)(0.950,0.485){11}{\rule{0.843pt}{0.117pt}}
\multiput(983.00,578.17)(11.251,7.000){2}{\rule{0.421pt}{0.400pt}}
\multiput(996.00,586.59)(0.950,0.485){11}{\rule{0.843pt}{0.117pt}}
\multiput(996.00,585.17)(11.251,7.000){2}{\rule{0.421pt}{0.400pt}}
\multiput(1009.00,593.59)(1.026,0.485){11}{\rule{0.900pt}{0.117pt}}
\multiput(1009.00,592.17)(12.132,7.000){2}{\rule{0.450pt}{0.400pt}}
\multiput(1023.00,600.59)(0.950,0.485){11}{\rule{0.843pt}{0.117pt}}
\multiput(1023.00,599.17)(11.251,7.000){2}{\rule{0.421pt}{0.400pt}}
\multiput(1036.00,607.59)(1.026,0.485){11}{\rule{0.900pt}{0.117pt}}
\multiput(1036.00,606.17)(12.132,7.000){2}{\rule{0.450pt}{0.400pt}}
\multiput(1050.00,614.59)(1.123,0.482){9}{\rule{0.967pt}{0.116pt}}
\multiput(1050.00,613.17)(10.994,6.000){2}{\rule{0.483pt}{0.400pt}}
\multiput(1063.00,620.59)(1.026,0.485){11}{\rule{0.900pt}{0.117pt}}
\multiput(1063.00,619.17)(12.132,7.000){2}{\rule{0.450pt}{0.400pt}}
\multiput(1077.00,627.59)(0.950,0.485){11}{\rule{0.843pt}{0.117pt}}
\multiput(1077.00,626.17)(11.251,7.000){2}{\rule{0.421pt}{0.400pt}}
\multiput(1090.00,634.59)(0.950,0.485){11}{\rule{0.843pt}{0.117pt}}
\multiput(1090.00,633.17)(11.251,7.000){2}{\rule{0.421pt}{0.400pt}}
\multiput(1103.00,641.59)(1.026,0.485){11}{\rule{0.900pt}{0.117pt}}
\multiput(1103.00,640.17)(12.132,7.000){2}{\rule{0.450pt}{0.400pt}}
\multiput(1117.00,648.59)(0.950,0.485){11}{\rule{0.843pt}{0.117pt}}
\multiput(1117.00,647.17)(11.251,7.000){2}{\rule{0.421pt}{0.400pt}}
\multiput(1130.00,655.59)(1.026,0.485){11}{\rule{0.900pt}{0.117pt}}
\multiput(1130.00,654.17)(12.132,7.000){2}{\rule{0.450pt}{0.400pt}}
\multiput(1144.00,662.59)(0.950,0.485){11}{\rule{0.843pt}{0.117pt}}
\multiput(1144.00,661.17)(11.251,7.000){2}{\rule{0.421pt}{0.400pt}}
\multiput(1157.00,669.59)(1.026,0.485){11}{\rule{0.900pt}{0.117pt}}
\multiput(1157.00,668.17)(12.132,7.000){2}{\rule{0.450pt}{0.400pt}}
\multiput(1171.00,676.59)(0.950,0.485){11}{\rule{0.843pt}{0.117pt}}
\multiput(1171.00,675.17)(11.251,7.000){2}{\rule{0.421pt}{0.400pt}}
\multiput(1184.00,683.59)(0.950,0.485){11}{\rule{0.843pt}{0.117pt}}
\multiput(1184.00,682.17)(11.251,7.000){2}{\rule{0.421pt}{0.400pt}}
\multiput(1197.00,690.59)(1.026,0.485){11}{\rule{0.900pt}{0.117pt}}
\multiput(1197.00,689.17)(12.132,7.000){2}{\rule{0.450pt}{0.400pt}}
\multiput(1211.00,697.59)(0.950,0.485){11}{\rule{0.843pt}{0.117pt}}
\multiput(1211.00,696.17)(11.251,7.000){2}{\rule{0.421pt}{0.400pt}}
\multiput(1224.00,704.59)(1.026,0.485){11}{\rule{0.900pt}{0.117pt}}
\multiput(1224.00,703.17)(12.132,7.000){2}{\rule{0.450pt}{0.400pt}}
\multiput(1238.00,711.59)(0.950,0.485){11}{\rule{0.843pt}{0.117pt}}
\multiput(1238.00,710.17)(11.251,7.000){2}{\rule{0.421pt}{0.400pt}}
\multiput(1251.00,718.59)(0.950,0.485){11}{\rule{0.843pt}{0.117pt}}
\multiput(1251.00,717.17)(11.251,7.000){2}{\rule{0.421pt}{0.400pt}}
\multiput(1264.00,725.59)(1.026,0.485){11}{\rule{0.900pt}{0.117pt}}
\multiput(1264.00,724.17)(12.132,7.000){2}{\rule{0.450pt}{0.400pt}}
\multiput(1278.00,732.59)(0.950,0.485){11}{\rule{0.843pt}{0.117pt}}
\multiput(1278.00,731.17)(11.251,7.000){2}{\rule{0.421pt}{0.400pt}}
\multiput(1291.00,739.59)(1.026,0.485){11}{\rule{0.900pt}{0.117pt}}
\multiput(1291.00,738.17)(12.132,7.000){2}{\rule{0.450pt}{0.400pt}}
\multiput(1305.00,746.59)(0.950,0.485){11}{\rule{0.843pt}{0.117pt}}
\multiput(1305.00,745.17)(11.251,7.000){2}{\rule{0.421pt}{0.400pt}}
\multiput(1318.00,753.59)(1.026,0.485){11}{\rule{0.900pt}{0.117pt}}
\multiput(1318.00,752.17)(12.132,7.000){2}{\rule{0.450pt}{0.400pt}}
\multiput(1332.00,760.59)(0.950,0.485){11}{\rule{0.843pt}{0.117pt}}
\multiput(1332.00,759.17)(11.251,7.000){2}{\rule{0.421pt}{0.400pt}}
\multiput(1345.00,767.59)(0.950,0.485){11}{\rule{0.843pt}{0.117pt}}
\multiput(1345.00,766.17)(11.251,7.000){2}{\rule{0.421pt}{0.400pt}}
\multiput(1358.00,774.59)(1.026,0.485){11}{\rule{0.900pt}{0.117pt}}
\multiput(1358.00,773.17)(12.132,7.000){2}{\rule{0.450pt}{0.400pt}}
\multiput(1372.00,781.59)(0.950,0.485){11}{\rule{0.843pt}{0.117pt}}
\multiput(1372.00,780.17)(11.251,7.000){2}{\rule{0.421pt}{0.400pt}}
\multiput(1385.00,788.59)(1.026,0.485){11}{\rule{0.900pt}{0.117pt}}
\multiput(1385.00,787.17)(12.132,7.000){2}{\rule{0.450pt}{0.400pt}}
\multiput(1399.00,795.59)(0.950,0.485){11}{\rule{0.843pt}{0.117pt}}
\multiput(1399.00,794.17)(11.251,7.000){2}{\rule{0.421pt}{0.400pt}}
\multiput(1412.00,802.59)(1.026,0.485){11}{\rule{0.900pt}{0.117pt}}
\multiput(1412.00,801.17)(12.132,7.000){2}{\rule{0.450pt}{0.400pt}}
\multiput(1426.00,809.59)(0.950,0.485){11}{\rule{0.843pt}{0.117pt}}
\multiput(1426.00,808.17)(11.251,7.000){2}{\rule{0.421pt}{0.400pt}}
\put(608,384){\makebox(0,0){$\blacksquare$}}
\put(1273,730){\makebox(0,0){$\blacksquare$}}
\put(941,557){\makebox(0,0){$\blacksquare$}}
\end{picture}

\end{center}
\caption{Plotting the points from Table \ref{table_1d_example}}
\label{fig.simple_line_with_values}
\end {figure}

Let's stary by trying to explicitly compute the elements of the w vector i.e. $y_0$ and $\tan \alpha_1$. First we need to compute the tangent:

\begin{equation}
\tan\alpha = \frac{y_2 - y_1}{x_{21} - x_{11}} = \frac{13 - 5}{6 - 2} = 2
\end{equation}
And now, solving, equation \ref{eq.simple_linear} for $y_u = 5$, we get:
\begin{equation}
y_0 = y_1 - x_{11} * \tan\alpha_1 = 5 - 2 * 2 = 1
\end{equation}
Alternatively, the \emph{entire} data set can be described using the matrix notation using a single equation:
\begin{equation}
Y = X \cdot w
\end{equation}

This allows us to compute the value of vector w using the basic matrix calculus:
\begin{equation}
w = X^{-1} Y = \begin{bmatrix}
1 & 2 \\
1 & 6
\end{bmatrix}^{-1}
\cdot
\begin{bmatrix}
5 \\ 
13
\end{bmatrix}
= 
\begin{bmatrix}
1.5 & -0.5\\
-0.25 & 0.25\\
\end{bmatrix}
\cdot
\begin{bmatrix}
5 \\ 
13
\end{bmatrix}
= 
\begin{bmatrix}
1.5 \cdot 5 + -0.5 \cdot 13\\
-0.25 \cdot 5 + 0.25 \cdot 13\\
\end{bmatrix}
\label{slope_equation}
= 
\begin{bmatrix}
1\\
2\\
\end{bmatrix}
\end{equation}
We obviously get the same values for the vector w but we did not have to explictily deal with the equations related with the intercept and with the tangent. Overall, matrix algebra allow a much more compact representation.The usefulness of the matrix approach will become even more obvious when we move to a scenario with 3 or more dimension, where computing the components of the w vector explictly becomes very tedious.

\subsubsection{Solving $Y = Xw$ symbolically for 2 points, in the 1-D case}
Let us try to symbolically solve $Y = Xw$ for the 1-D case with two data points, so that we can obtain the $w$ vector. It is clear that vector w is given by:
Since:
\begin{equation}
X =
\begin{bmatrix}
1 & x_{11}\\
1 & x_{21}\\
\end{bmatrix}
\end{equation}
we have:
\begin{equation}
X^{-1} = \frac{1}{x_{21}-x_{11}}
\begin{bmatrix}
x_{21} & -x_{11}\\
-1 & 1\\
\end{bmatrix}
\end{equation}
Therefore:
\begin{equation}
w = X^{-1} Y = \frac{1}{x_{21}-x_{11}}
\begin{bmatrix}
x_{21} & -x_{11}\\
-1 & 1
\end{bmatrix}
\cdot
\begin{bmatrix}
y_1 \\
y_2
\end{bmatrix}
=
\begin{bmatrix}
\frac{x_{21} \cdot y_1 - x_{11} \cdot y_2}{x_{21} - x_{11}}  \\
\frac{y_2 - y_1}{x_2 - x_1}
\end{bmatrix}
\end{equation}
According to previous derivations, we know that:
\begin{equation}
\begin{bmatrix}
\frac{x_{21} \cdot y_1 - x_{11} \cdot y_2}{x_{21} - x_{11}}  \\
\frac{y_2 - y_1}{x_2 - x_1}
\end{bmatrix}
=
\begin{bmatrix}
y_{00}\\
\tan \alpha
\end{bmatrix}
\end{equation}
that is, the w store the \emph{intercept} ($y_{00}$) and the tangent on dimension 1. While it is quite clear that:
\begin{equation}
\tan \alpha = \frac{y_2 - y_1}{x_2 - x_1} 
\end{equation}
it is not so obvious that the intercept is in fact given by
\begin{equation}
y_{00} = \frac{x_{21} \cdot y_1 - x_{11} \cdot y_2}{x_{21} - x_{11}}
\label{intercept_equation_symbolic_2d_derivation}
\end{equation}
The demonstration is left as Exercise 1.1.

\subsection{Going Wild! Adding one extra dimension to the $\mathcal{X}$ observations}

Let's now assume that the value of Y is a function of not just one attribute of X, but of two attributes. Our obervations may now be described by a table such as:
\begin{align}
\begin{tabular}{c|cc}
$\mathcal{Y}$ & \multicolumn{2}{c}{$\mathcal{X}$} \\
\hline
$y_1$ & $x_{11}$ & $x_{12}$\\
$y_2$ & $x_{21}$ & $x_{22}$\\
$y_3$ & $x_{31}$ & $x_{32}$\\
\end{tabular}
\label{table_2d_abstract}
\end{align}
Instead of having a line that passes through the XY points, we now have a plane. We will need at least 2+1 points to define the plane. Figure \ref{fig.simple_plane} illustrates one of these situations when we have the following 3 data points:
\begin{align}
\begin{tabular}{c|cc}
$\mathcal{Y}$ & \multicolumn{2}{c}{$\mathcal{X}$} \\
\hline
1 & 1 & 4\\
6 & 2 &-2\\
12&6 & 2\\
\end{tabular}
\label{table_2d_numbers}
\end{align}
\begin {figure}[H]
\begin{center}
  % GNUPLOT: LaTeX picture
\setlength{\unitlength}{0.240900pt}
\ifx\plotpoint\undefined\newsavebox{\plotpoint}\fi
\sbox{\plotpoint}{\rule[-0.200pt]{0.400pt}{0.400pt}}%
\begin{picture}(1500,900)(0,0)
\sbox{\plotpoint}{\rule[-0.200pt]{0.400pt}{0.400pt}}%
\multiput(1261.29,412.59)(-3.716,0.477){7}{\rule{2.820pt}{0.115pt}}
\multiput(1267.15,411.17)(-28.147,5.000){2}{\rule{1.410pt}{0.400pt}}
\put(1307,404){\makebox(0,0)[l]{-2}}
\multiput(560.00,498.94)(4.868,-0.468){5}{\rule{3.500pt}{0.113pt}}
\multiput(560.00,499.17)(26.736,-4.000){2}{\rule{1.750pt}{0.400pt}}
\multiput(1203.47,381.60)(-4.868,0.468){5}{\rule{3.500pt}{0.113pt}}
\multiput(1210.74,380.17)(-26.736,4.000){2}{\rule{1.750pt}{0.400pt}}
\put(1252,372){\makebox(0,0)[l]{-1}}
\multiput(504.00,467.94)(5.014,-0.468){5}{\rule{3.600pt}{0.113pt}}
\multiput(504.00,468.17)(27.528,-4.000){2}{\rule{1.800pt}{0.400pt}}
\multiput(1147.47,349.60)(-4.868,0.468){5}{\rule{3.500pt}{0.113pt}}
\multiput(1154.74,348.17)(-26.736,4.000){2}{\rule{1.750pt}{0.400pt}}
\put(1196,340){\makebox(0,0)[l]{ 0}}
\multiput(449.00,436.93)(3.716,-0.477){7}{\rule{2.820pt}{0.115pt}}
\multiput(449.00,437.17)(28.147,-5.000){2}{\rule{1.410pt}{0.400pt}}
\multiput(1094.96,317.59)(-3.827,0.477){7}{\rule{2.900pt}{0.115pt}}
\multiput(1100.98,316.17)(-28.981,5.000){2}{\rule{1.450pt}{0.400pt}}
\put(1141,309){\makebox(0,0)[l]{ 1}}
\multiput(393.00,404.94)(5.014,-0.468){5}{\rule{3.600pt}{0.113pt}}
\multiput(393.00,405.17)(27.528,-4.000){2}{\rule{1.800pt}{0.400pt}}
\multiput(1036.47,286.60)(-4.868,0.468){5}{\rule{3.500pt}{0.113pt}}
\multiput(1043.74,285.17)(-26.736,4.000){2}{\rule{1.750pt}{0.400pt}}
\put(1085,277){\makebox(0,0)[l]{ 2}}
\multiput(338.00,372.94)(4.868,-0.468){5}{\rule{3.500pt}{0.113pt}}
\multiput(338.00,373.17)(26.736,-4.000){2}{\rule{1.750pt}{0.400pt}}
\multiput(981.06,254.60)(-5.014,0.468){5}{\rule{3.600pt}{0.113pt}}
\multiput(988.53,253.17)(-27.528,4.000){2}{\rule{1.800pt}{0.400pt}}
\put(1030,245){\makebox(0,0)[l]{ 3}}
\multiput(282.00,341.93)(3.716,-0.477){7}{\rule{2.820pt}{0.115pt}}
\multiput(282.00,342.17)(28.147,-5.000){2}{\rule{1.410pt}{0.400pt}}
\multiput(928.29,222.59)(-3.716,0.477){7}{\rule{2.820pt}{0.115pt}}
\multiput(934.15,221.17)(-28.147,5.000){2}{\rule{1.410pt}{0.400pt}}
\put(974,214){\makebox(0,0)[l]{ 4}}
\multiput(227.00,309.94)(4.868,-0.468){5}{\rule{3.500pt}{0.113pt}}
\multiput(227.00,310.17)(26.736,-4.000){2}{\rule{1.750pt}{0.400pt}}
\multiput(227.00,311.59)(0.821,0.477){7}{\rule{0.740pt}{0.115pt}}
\multiput(227.00,310.17)(6.464,5.000){2}{\rule{0.370pt}{0.400pt}}
\put(218,301){\makebox(0,0){ 0}}
\multiput(556.60,498.93)(-0.933,-0.477){7}{\rule{0.820pt}{0.115pt}}
\multiput(558.30,499.17)(-7.298,-5.000){2}{\rule{0.410pt}{0.400pt}}
\multiput(346.00,296.59)(0.821,0.477){7}{\rule{0.740pt}{0.115pt}}
\multiput(346.00,295.17)(6.464,5.000){2}{\rule{0.370pt}{0.400pt}}
\put(337,286){\makebox(0,0){ 1}}
\multiput(675.93,483.93)(-0.821,-0.477){7}{\rule{0.740pt}{0.115pt}}
\multiput(677.46,484.17)(-6.464,-5.000){2}{\rule{0.370pt}{0.400pt}}
\multiput(465.00,281.59)(0.821,0.477){7}{\rule{0.740pt}{0.115pt}}
\multiput(465.00,280.17)(6.464,5.000){2}{\rule{0.370pt}{0.400pt}}
\put(457,272){\makebox(0,0){ 2}}
\multiput(792.16,468.95)(-1.579,-0.447){3}{\rule{1.167pt}{0.108pt}}
\multiput(794.58,469.17)(-5.579,-3.000){2}{\rule{0.583pt}{0.400pt}}
\multiput(584.00,267.59)(0.821,0.477){7}{\rule{0.740pt}{0.115pt}}
\multiput(584.00,266.17)(6.464,5.000){2}{\rule{0.370pt}{0.400pt}}
\put(576,257){\makebox(0,0){ 3}}
\multiput(912.93,455.93)(-0.821,-0.477){7}{\rule{0.740pt}{0.115pt}}
\multiput(914.46,456.17)(-6.464,-5.000){2}{\rule{0.370pt}{0.400pt}}
\multiput(703.00,252.59)(0.821,0.477){7}{\rule{0.740pt}{0.115pt}}
\multiput(703.00,251.17)(6.464,5.000){2}{\rule{0.370pt}{0.400pt}}
\put(695,242){\makebox(0,0){ 4}}
\multiput(1031.93,440.93)(-0.821,-0.477){7}{\rule{0.740pt}{0.115pt}}
\multiput(1033.46,441.17)(-6.464,-5.000){2}{\rule{0.370pt}{0.400pt}}
\multiput(821.00,237.59)(0.821,0.477){7}{\rule{0.740pt}{0.115pt}}
\multiput(821.00,236.17)(6.464,5.000){2}{\rule{0.370pt}{0.400pt}}
\put(813,227){\makebox(0,0){ 5}}
\multiput(1150.93,425.93)(-0.821,-0.477){7}{\rule{0.740pt}{0.115pt}}
\multiput(1152.46,426.17)(-6.464,-5.000){2}{\rule{0.370pt}{0.400pt}}
\multiput(940.00,222.59)(0.933,0.477){7}{\rule{0.820pt}{0.115pt}}
\multiput(940.00,221.17)(7.298,5.000){2}{\rule{0.410pt}{0.400pt}}
\put(932,213){\makebox(0,0){ 6}}
\multiput(1269.93,410.93)(-0.821,-0.477){7}{\rule{0.740pt}{0.115pt}}
\multiput(1271.46,411.17)(-6.464,-5.000){2}{\rule{0.370pt}{0.400pt}}
\put(409,438){\makebox(0,0)[r]{ 0}}
\put(449.0,438.0){\rule[-0.200pt]{4.818pt}{0.400pt}}
\put(409,527){\makebox(0,0)[r]{ 5}}
\put(449.0,527.0){\rule[-0.200pt]{4.818pt}{0.400pt}}
\put(409,618){\makebox(0,0)[r]{ 10}}
\put(449.0,618.0){\rule[-0.200pt]{4.818pt}{0.400pt}}
\put(409,709){\makebox(0,0)[r]{ 15}}
\put(449.0,709.0){\rule[-0.200pt]{4.818pt}{0.400pt}}
\put(449.0,365.0){\rule[-0.200pt]{0.400pt}{87.206pt}}
\put(619,692){\makebox(0,0)[l]{($x_1 = 2, x_2 = -2, y = 6$)}}
\put(1117,468){\makebox(0,0)[l]{($x_1 = 6, x_2 = 2, y = 12$)}}
\put(231,199){\makebox(0,0)[l]{($x_1 = 1, x_2 = 4, y = 1$)}}
\put(262,537){\makebox(0,0)[l]{$y_{00}$}}
\put(1281,316){\makebox(0,0)[l]{$x_1$}}
\put(60,252){\makebox(0,0)[l]{$x_2$}}
\put(435,764){\makebox(0,0)[l]{y}}
\multiput(556.66,498.92)(-0.881,-0.500){501}{\rule{0.805pt}{0.120pt}}
\multiput(558.33,499.17)(-442.330,-252.000){2}{\rule{0.402pt}{0.400pt}}
\put(116,248){\vector(-2,-1){0}}
\put(449,401){\vector(0,1){344}}
\multiput(449.00,436.92)(4.011,-0.499){205}{\rule{3.300pt}{0.120pt}}
\multiput(449.00,437.17)(825.151,-104.000){2}{\rule{1.650pt}{0.400pt}}
\put(1281,334){\vector(4,-1){0}}
\sbox{\plotpoint}{\rule[-0.500pt]{1.000pt}{1.000pt}}%
\sbox{\plotpoint}{\rule[-0.200pt]{0.400pt}{0.400pt}}%
\multiput(330.00,434.58)(2.785,0.500){339}{\rule{2.325pt}{0.120pt}}
\multiput(330.00,433.17)(946.175,171.000){2}{\rule{1.162pt}{0.400pt}}
\sbox{\plotpoint}{\rule[-0.500pt]{1.000pt}{1.000pt}}%
\sbox{\plotpoint}{\rule[-0.200pt]{0.400pt}{0.400pt}}%
\multiput(668.32,616.92)(-0.682,-0.500){811}{\rule{0.645pt}{0.120pt}}
\multiput(669.66,617.17)(-553.660,-407.000){2}{\rule{0.323pt}{0.400pt}}
\sbox{\plotpoint}{\rule[-0.400pt]{0.800pt}{0.800pt}}%
\put(797,652){\vector(0,-1){73}}
\sbox{\plotpoint}{\rule[-0.200pt]{0.400pt}{0.400pt}}%
\multiput(1106.60,477.58)(-1.209,0.497){47}{\rule{1.060pt}{0.120pt}}
\multiput(1108.80,476.17)(-57.800,25.000){2}{\rule{0.530pt}{0.400pt}}
\put(1051,502){\vector(-3,1){0}}
\sbox{\plotpoint}{\rule[-0.400pt]{0.800pt}{0.800pt}}%
\sbox{\plotpoint}{\rule[-0.200pt]{0.400pt}{0.400pt}}%
\multiput(407.92,223.00)(-0.499,0.723){123}{\rule{0.120pt}{0.678pt}}
\multiput(408.17,223.00)(-63.000,89.593){2}{\rule{0.400pt}{0.339pt}}
\put(346,314){\vector(-2,3){0}}
\sbox{\plotpoint}{\rule[-0.400pt]{0.800pt}{0.800pt}}%
\sbox{\plotpoint}{\rule[-0.200pt]{0.400pt}{0.400pt}}%
\multiput(330.00,522.92)(0.876,-0.499){133}{\rule{0.800pt}{0.120pt}}
\multiput(330.00,523.17)(117.340,-68.000){2}{\rule{0.400pt}{0.400pt}}
\put(449,456){\vector(2,-1){0}}
\put(557,534.17){\rule{0.700pt}{0.400pt}}
\multiput(558.55,535.17)(-1.547,-2.000){2}{\rule{0.350pt}{0.400pt}}
\multiput(554.37,532.95)(-0.685,-0.447){3}{\rule{0.633pt}{0.108pt}}
\multiput(555.69,533.17)(-2.685,-3.000){2}{\rule{0.317pt}{0.400pt}}
\put(550,529.17){\rule{0.700pt}{0.400pt}}
\multiput(551.55,530.17)(-1.547,-2.000){2}{\rule{0.350pt}{0.400pt}}
\multiput(547.37,527.95)(-0.685,-0.447){3}{\rule{0.633pt}{0.108pt}}
\multiput(548.69,528.17)(-2.685,-3.000){2}{\rule{0.317pt}{0.400pt}}
\put(543,524.17){\rule{0.700pt}{0.400pt}}
\multiput(544.55,525.17)(-1.547,-2.000){2}{\rule{0.350pt}{0.400pt}}
\multiput(540.92,522.95)(-0.462,-0.447){3}{\rule{0.500pt}{0.108pt}}
\multiput(541.96,523.17)(-1.962,-3.000){2}{\rule{0.250pt}{0.400pt}}
\put(536,519.17){\rule{0.900pt}{0.400pt}}
\multiput(538.13,520.17)(-2.132,-2.000){2}{\rule{0.450pt}{0.400pt}}
\multiput(533.92,517.95)(-0.462,-0.447){3}{\rule{0.500pt}{0.108pt}}
\multiput(534.96,518.17)(-1.962,-3.000){2}{\rule{0.250pt}{0.400pt}}
\put(530,514.17){\rule{0.700pt}{0.400pt}}
\multiput(531.55,515.17)(-1.547,-2.000){2}{\rule{0.350pt}{0.400pt}}
\multiput(527.37,512.95)(-0.685,-0.447){3}{\rule{0.633pt}{0.108pt}}
\multiput(528.69,513.17)(-2.685,-3.000){2}{\rule{0.317pt}{0.400pt}}
\put(523,509.17){\rule{0.700pt}{0.400pt}}
\multiput(524.55,510.17)(-1.547,-2.000){2}{\rule{0.350pt}{0.400pt}}
\put(520,507.17){\rule{0.700pt}{0.400pt}}
\multiput(521.55,508.17)(-1.547,-2.000){2}{\rule{0.350pt}{0.400pt}}
\multiput(517.37,505.95)(-0.685,-0.447){3}{\rule{0.633pt}{0.108pt}}
\multiput(518.69,506.17)(-2.685,-3.000){2}{\rule{0.317pt}{0.400pt}}
\put(513,502.17){\rule{0.700pt}{0.400pt}}
\multiput(514.55,503.17)(-1.547,-2.000){2}{\rule{0.350pt}{0.400pt}}
\multiput(510.37,500.95)(-0.685,-0.447){3}{\rule{0.633pt}{0.108pt}}
\multiput(511.69,501.17)(-2.685,-3.000){2}{\rule{0.317pt}{0.400pt}}
\put(506,497.17){\rule{0.700pt}{0.400pt}}
\multiput(507.55,498.17)(-1.547,-2.000){2}{\rule{0.350pt}{0.400pt}}
\multiput(503.92,495.95)(-0.462,-0.447){3}{\rule{0.500pt}{0.108pt}}
\multiput(504.96,496.17)(-1.962,-3.000){2}{\rule{0.250pt}{0.400pt}}
\put(499,492.17){\rule{0.900pt}{0.400pt}}
\multiput(501.13,493.17)(-2.132,-2.000){2}{\rule{0.450pt}{0.400pt}}
\multiput(496.92,490.95)(-0.462,-0.447){3}{\rule{0.500pt}{0.108pt}}
\multiput(497.96,491.17)(-1.962,-3.000){2}{\rule{0.250pt}{0.400pt}}
\put(493,487.17){\rule{0.700pt}{0.400pt}}
\multiput(494.55,488.17)(-1.547,-2.000){2}{\rule{0.350pt}{0.400pt}}
\multiput(490.37,485.95)(-0.685,-0.447){3}{\rule{0.633pt}{0.108pt}}
\multiput(491.69,486.17)(-2.685,-3.000){2}{\rule{0.317pt}{0.400pt}}
\put(486,482.17){\rule{0.700pt}{0.400pt}}
\multiput(487.55,483.17)(-1.547,-2.000){2}{\rule{0.350pt}{0.400pt}}
\multiput(483.92,480.95)(-0.462,-0.447){3}{\rule{0.500pt}{0.108pt}}
\multiput(484.96,481.17)(-1.962,-3.000){2}{\rule{0.250pt}{0.400pt}}
\put(479,477.17){\rule{0.900pt}{0.400pt}}
\multiput(481.13,478.17)(-2.132,-2.000){2}{\rule{0.450pt}{0.400pt}}
\multiput(476.92,475.95)(-0.462,-0.447){3}{\rule{0.500pt}{0.108pt}}
\multiput(477.96,476.17)(-1.962,-3.000){2}{\rule{0.250pt}{0.400pt}}
\put(472,472.17){\rule{0.900pt}{0.400pt}}
\multiput(474.13,473.17)(-2.132,-2.000){2}{\rule{0.450pt}{0.400pt}}
\put(469,470.17){\rule{0.700pt}{0.400pt}}
\multiput(470.55,471.17)(-1.547,-2.000){2}{\rule{0.350pt}{0.400pt}}
\put(466,468.17){\rule{0.700pt}{0.400pt}}
\multiput(467.55,469.17)(-1.547,-2.000){2}{\rule{0.350pt}{0.400pt}}
\put(462,466.17){\rule{0.900pt}{0.400pt}}
\multiput(464.13,467.17)(-2.132,-2.000){2}{\rule{0.450pt}{0.400pt}}
\multiput(459.92,464.95)(-0.462,-0.447){3}{\rule{0.500pt}{0.108pt}}
\multiput(460.96,465.17)(-1.962,-3.000){2}{\rule{0.250pt}{0.400pt}}
\put(456,461.17){\rule{0.700pt}{0.400pt}}
\multiput(457.55,462.17)(-1.547,-2.000){2}{\rule{0.350pt}{0.400pt}}
\multiput(453.37,459.95)(-0.685,-0.447){3}{\rule{0.633pt}{0.108pt}}
\multiput(454.69,460.17)(-2.685,-3.000){2}{\rule{0.317pt}{0.400pt}}
\put(449,456.17){\rule{0.700pt}{0.400pt}}
\multiput(450.55,457.17)(-1.547,-2.000){2}{\rule{0.350pt}{0.400pt}}
\multiput(446.92,454.95)(-0.462,-0.447){3}{\rule{0.500pt}{0.108pt}}
\multiput(447.96,455.17)(-1.962,-3.000){2}{\rule{0.250pt}{0.400pt}}
\put(442,451.17){\rule{0.900pt}{0.400pt}}
\multiput(444.13,452.17)(-2.132,-2.000){2}{\rule{0.450pt}{0.400pt}}
\multiput(439.92,449.95)(-0.462,-0.447){3}{\rule{0.500pt}{0.108pt}}
\multiput(440.96,450.17)(-1.962,-3.000){2}{\rule{0.250pt}{0.400pt}}
\put(435,446.17){\rule{0.900pt}{0.400pt}}
\multiput(437.13,447.17)(-2.132,-2.000){2}{\rule{0.450pt}{0.400pt}}
\multiput(432.92,444.95)(-0.462,-0.447){3}{\rule{0.500pt}{0.108pt}}
\multiput(433.96,445.17)(-1.962,-3.000){2}{\rule{0.250pt}{0.400pt}}
\put(429,441.17){\rule{0.700pt}{0.400pt}}
\multiput(430.55,442.17)(-1.547,-2.000){2}{\rule{0.350pt}{0.400pt}}
\multiput(426.37,439.95)(-0.685,-0.447){3}{\rule{0.633pt}{0.108pt}}
\multiput(427.69,440.17)(-2.685,-3.000){2}{\rule{0.317pt}{0.400pt}}
\put(422,436.17){\rule{0.700pt}{0.400pt}}
\multiput(423.55,437.17)(-1.547,-2.000){2}{\rule{0.350pt}{0.400pt}}
\multiput(419.92,434.95)(-0.462,-0.447){3}{\rule{0.500pt}{0.108pt}}
\multiput(420.96,435.17)(-1.962,-3.000){2}{\rule{0.250pt}{0.400pt}}
\put(415,431.17){\rule{0.900pt}{0.400pt}}
\multiput(417.13,432.17)(-2.132,-2.000){2}{\rule{0.450pt}{0.400pt}}
\put(412,429.17){\rule{0.700pt}{0.400pt}}
\multiput(413.55,430.17)(-1.547,-2.000){2}{\rule{0.350pt}{0.400pt}}
\multiput(409.92,427.95)(-0.462,-0.447){3}{\rule{0.500pt}{0.108pt}}
\multiput(410.96,428.17)(-1.962,-3.000){2}{\rule{0.250pt}{0.400pt}}
\put(405,424.17){\rule{0.900pt}{0.400pt}}
\multiput(407.13,425.17)(-2.132,-2.000){2}{\rule{0.450pt}{0.400pt}}
\multiput(402.92,422.95)(-0.462,-0.447){3}{\rule{0.500pt}{0.108pt}}
\multiput(403.96,423.17)(-1.962,-3.000){2}{\rule{0.250pt}{0.400pt}}
\put(398,419.17){\rule{0.900pt}{0.400pt}}
\multiput(400.13,420.17)(-2.132,-2.000){2}{\rule{0.450pt}{0.400pt}}
\multiput(395.92,417.95)(-0.462,-0.447){3}{\rule{0.500pt}{0.108pt}}
\multiput(396.96,418.17)(-1.962,-3.000){2}{\rule{0.250pt}{0.400pt}}
\put(392,414.17){\rule{0.700pt}{0.400pt}}
\multiput(393.55,415.17)(-1.547,-2.000){2}{\rule{0.350pt}{0.400pt}}
\multiput(389.37,412.95)(-0.685,-0.447){3}{\rule{0.633pt}{0.108pt}}
\multiput(390.69,413.17)(-2.685,-3.000){2}{\rule{0.317pt}{0.400pt}}
\put(385,409.17){\rule{0.700pt}{0.400pt}}
\multiput(386.55,410.17)(-1.547,-2.000){2}{\rule{0.350pt}{0.400pt}}
\multiput(382.92,407.95)(-0.462,-0.447){3}{\rule{0.500pt}{0.108pt}}
\multiput(383.96,408.17)(-1.962,-3.000){2}{\rule{0.250pt}{0.400pt}}
\put(378,404.17){\rule{0.900pt}{0.400pt}}
\multiput(380.13,405.17)(-2.132,-2.000){2}{\rule{0.450pt}{0.400pt}}
\multiput(375.92,402.95)(-0.462,-0.447){3}{\rule{0.500pt}{0.108pt}}
\multiput(376.96,403.17)(-1.962,-3.000){2}{\rule{0.250pt}{0.400pt}}
\put(372,399.17){\rule{0.700pt}{0.400pt}}
\multiput(373.55,400.17)(-1.547,-2.000){2}{\rule{0.350pt}{0.400pt}}
\multiput(369.37,397.95)(-0.685,-0.447){3}{\rule{0.633pt}{0.108pt}}
\multiput(370.69,398.17)(-2.685,-3.000){2}{\rule{0.317pt}{0.400pt}}
\put(365,394.17){\rule{0.700pt}{0.400pt}}
\multiput(366.55,395.17)(-1.547,-2.000){2}{\rule{0.350pt}{0.400pt}}
\put(361,392.17){\rule{0.900pt}{0.400pt}}
\multiput(363.13,393.17)(-2.132,-2.000){2}{\rule{0.450pt}{0.400pt}}
\multiput(358.92,390.95)(-0.462,-0.447){3}{\rule{0.500pt}{0.108pt}}
\multiput(359.96,391.17)(-1.962,-3.000){2}{\rule{0.250pt}{0.400pt}}
\put(355,387.17){\rule{0.700pt}{0.400pt}}
\multiput(356.55,388.17)(-1.547,-2.000){2}{\rule{0.350pt}{0.400pt}}
\multiput(352.37,385.95)(-0.685,-0.447){3}{\rule{0.633pt}{0.108pt}}
\multiput(353.69,386.17)(-2.685,-3.000){2}{\rule{0.317pt}{0.400pt}}
\put(348,382.17){\rule{0.700pt}{0.400pt}}
\multiput(349.55,383.17)(-1.547,-2.000){2}{\rule{0.350pt}{0.400pt}}
\multiput(345.92,380.95)(-0.462,-0.447){3}{\rule{0.500pt}{0.108pt}}
\multiput(346.96,381.17)(-1.962,-3.000){2}{\rule{0.250pt}{0.400pt}}
\put(341,377.17){\rule{0.900pt}{0.400pt}}
\multiput(343.13,378.17)(-2.132,-2.000){2}{\rule{0.450pt}{0.400pt}}
\multiput(338.92,375.95)(-0.462,-0.447){3}{\rule{0.500pt}{0.108pt}}
\multiput(339.96,376.17)(-1.962,-3.000){2}{\rule{0.250pt}{0.400pt}}
\put(335,372.17){\rule{0.700pt}{0.400pt}}
\multiput(336.55,373.17)(-1.547,-2.000){2}{\rule{0.350pt}{0.400pt}}
\multiput(332.37,370.95)(-0.685,-0.447){3}{\rule{0.633pt}{0.108pt}}
\multiput(333.69,371.17)(-2.685,-3.000){2}{\rule{0.317pt}{0.400pt}}
\put(328,367.17){\rule{0.700pt}{0.400pt}}
\multiput(329.55,368.17)(-1.547,-2.000){2}{\rule{0.350pt}{0.400pt}}
\multiput(325.37,365.95)(-0.685,-0.447){3}{\rule{0.633pt}{0.108pt}}
\multiput(326.69,366.17)(-2.685,-3.000){2}{\rule{0.317pt}{0.400pt}}
\put(321,362.17){\rule{0.700pt}{0.400pt}}
\multiput(322.55,363.17)(-1.547,-2.000){2}{\rule{0.350pt}{0.400pt}}
\multiput(318.92,360.95)(-0.462,-0.447){3}{\rule{0.500pt}{0.108pt}}
\multiput(319.96,361.17)(-1.962,-3.000){2}{\rule{0.250pt}{0.400pt}}
\put(314,357.17){\rule{0.900pt}{0.400pt}}
\multiput(316.13,358.17)(-2.132,-2.000){2}{\rule{0.450pt}{0.400pt}}
\put(311,355.17){\rule{0.700pt}{0.400pt}}
\multiput(312.55,356.17)(-1.547,-2.000){2}{\rule{0.350pt}{0.400pt}}
\multiput(308.92,353.95)(-0.462,-0.447){3}{\rule{0.500pt}{0.108pt}}
\multiput(309.96,354.17)(-1.962,-3.000){2}{\rule{0.250pt}{0.400pt}}
\put(304,350.17){\rule{0.900pt}{0.400pt}}
\multiput(306.13,351.17)(-2.132,-2.000){2}{\rule{0.450pt}{0.400pt}}
\multiput(301.92,348.95)(-0.462,-0.447){3}{\rule{0.500pt}{0.108pt}}
\multiput(302.96,349.17)(-1.962,-3.000){2}{\rule{0.250pt}{0.400pt}}
\put(298,345.17){\rule{0.700pt}{0.400pt}}
\multiput(299.55,346.17)(-1.547,-2.000){2}{\rule{0.350pt}{0.400pt}}
\multiput(295.37,343.95)(-0.685,-0.447){3}{\rule{0.633pt}{0.108pt}}
\multiput(296.69,344.17)(-2.685,-3.000){2}{\rule{0.317pt}{0.400pt}}
\put(291,340.17){\rule{0.700pt}{0.400pt}}
\multiput(292.55,341.17)(-1.547,-2.000){2}{\rule{0.350pt}{0.400pt}}
\multiput(288.37,338.95)(-0.685,-0.447){3}{\rule{0.633pt}{0.108pt}}
\multiput(289.69,339.17)(-2.685,-3.000){2}{\rule{0.317pt}{0.400pt}}
\put(284,335.17){\rule{0.700pt}{0.400pt}}
\multiput(285.55,336.17)(-1.547,-2.000){2}{\rule{0.350pt}{0.400pt}}
\multiput(281.92,333.95)(-0.462,-0.447){3}{\rule{0.500pt}{0.108pt}}
\multiput(282.96,334.17)(-1.962,-3.000){2}{\rule{0.250pt}{0.400pt}}
\put(277,330.17){\rule{0.900pt}{0.400pt}}
\multiput(279.13,331.17)(-2.132,-2.000){2}{\rule{0.450pt}{0.400pt}}
\multiput(274.92,328.95)(-0.462,-0.447){3}{\rule{0.500pt}{0.108pt}}
\multiput(275.96,329.17)(-1.962,-3.000){2}{\rule{0.250pt}{0.400pt}}
\put(271,325.17){\rule{0.700pt}{0.400pt}}
\multiput(272.55,326.17)(-1.547,-2.000){2}{\rule{0.350pt}{0.400pt}}
\multiput(268.37,323.95)(-0.685,-0.447){3}{\rule{0.633pt}{0.108pt}}
\multiput(269.69,324.17)(-2.685,-3.000){2}{\rule{0.317pt}{0.400pt}}
\put(264,320.17){\rule{0.700pt}{0.400pt}}
\multiput(265.55,321.17)(-1.547,-2.000){2}{\rule{0.350pt}{0.400pt}}
\put(261,318.17){\rule{0.700pt}{0.400pt}}
\multiput(262.55,319.17)(-1.547,-2.000){2}{\rule{0.350pt}{0.400pt}}
\multiput(258.37,316.95)(-0.685,-0.447){3}{\rule{0.633pt}{0.108pt}}
\multiput(259.69,317.17)(-2.685,-3.000){2}{\rule{0.317pt}{0.400pt}}
\put(254,313.17){\rule{0.700pt}{0.400pt}}
\multiput(255.55,314.17)(-1.547,-2.000){2}{\rule{0.350pt}{0.400pt}}
\multiput(251.37,311.95)(-0.685,-0.447){3}{\rule{0.633pt}{0.108pt}}
\multiput(252.69,312.17)(-2.685,-3.000){2}{\rule{0.317pt}{0.400pt}}
\put(247,308.17){\rule{0.700pt}{0.400pt}}
\multiput(248.55,309.17)(-1.547,-2.000){2}{\rule{0.350pt}{0.400pt}}
\multiput(244.92,306.95)(-0.462,-0.447){3}{\rule{0.500pt}{0.108pt}}
\multiput(245.96,307.17)(-1.962,-3.000){2}{\rule{0.250pt}{0.400pt}}
\put(240,303.17){\rule{0.900pt}{0.400pt}}
\multiput(242.13,304.17)(-2.132,-2.000){2}{\rule{0.450pt}{0.400pt}}
\multiput(237.92,301.95)(-0.462,-0.447){3}{\rule{0.500pt}{0.108pt}}
\multiput(238.96,302.17)(-1.962,-3.000){2}{\rule{0.250pt}{0.400pt}}
\put(234,298.17){\rule{0.700pt}{0.400pt}}
\multiput(235.55,299.17)(-1.547,-2.000){2}{\rule{0.350pt}{0.400pt}}
\multiput(231.37,296.95)(-0.685,-0.447){3}{\rule{0.633pt}{0.108pt}}
\multiput(232.69,297.17)(-2.685,-3.000){2}{\rule{0.317pt}{0.400pt}}
\put(227,293.17){\rule{0.700pt}{0.400pt}}
\multiput(228.55,294.17)(-1.547,-2.000){2}{\rule{0.350pt}{0.400pt}}
\multiput(676.92,556.95)(-0.462,-0.447){3}{\rule{0.500pt}{0.108pt}}
\multiput(677.96,557.17)(-1.962,-3.000){2}{\rule{0.250pt}{0.400pt}}
\put(672,553.17){\rule{0.900pt}{0.400pt}}
\multiput(674.13,554.17)(-2.132,-2.000){2}{\rule{0.450pt}{0.400pt}}
\multiput(669.92,551.95)(-0.462,-0.447){3}{\rule{0.500pt}{0.108pt}}
\multiput(670.96,552.17)(-1.962,-3.000){2}{\rule{0.250pt}{0.400pt}}
\put(665,548.17){\rule{0.900pt}{0.400pt}}
\multiput(667.13,549.17)(-2.132,-2.000){2}{\rule{0.450pt}{0.400pt}}
\multiput(662.92,546.95)(-0.462,-0.447){3}{\rule{0.500pt}{0.108pt}}
\multiput(663.96,547.17)(-1.962,-3.000){2}{\rule{0.250pt}{0.400pt}}
\put(659,543.17){\rule{0.700pt}{0.400pt}}
\multiput(660.55,544.17)(-1.547,-2.000){2}{\rule{0.350pt}{0.400pt}}
\multiput(656.37,541.95)(-0.685,-0.447){3}{\rule{0.633pt}{0.108pt}}
\multiput(657.69,542.17)(-2.685,-3.000){2}{\rule{0.317pt}{0.400pt}}
\put(652,538.17){\rule{0.700pt}{0.400pt}}
\multiput(653.55,539.17)(-1.547,-2.000){2}{\rule{0.350pt}{0.400pt}}
\multiput(649.92,536.95)(-0.462,-0.447){3}{\rule{0.500pt}{0.108pt}}
\multiput(650.96,537.17)(-1.962,-3.000){2}{\rule{0.250pt}{0.400pt}}
\put(645,533.17){\rule{0.900pt}{0.400pt}}
\multiput(647.13,534.17)(-2.132,-2.000){2}{\rule{0.450pt}{0.400pt}}
\put(642,531.17){\rule{0.700pt}{0.400pt}}
\multiput(643.55,532.17)(-1.547,-2.000){2}{\rule{0.350pt}{0.400pt}}
\multiput(639.92,529.95)(-0.462,-0.447){3}{\rule{0.500pt}{0.108pt}}
\multiput(640.96,530.17)(-1.962,-3.000){2}{\rule{0.250pt}{0.400pt}}
\put(635,526.17){\rule{0.900pt}{0.400pt}}
\multiput(637.13,527.17)(-2.132,-2.000){2}{\rule{0.450pt}{0.400pt}}
\multiput(632.92,524.95)(-0.462,-0.447){3}{\rule{0.500pt}{0.108pt}}
\multiput(633.96,525.17)(-1.962,-3.000){2}{\rule{0.250pt}{0.400pt}}
\put(628,521.17){\rule{0.900pt}{0.400pt}}
\multiput(630.13,522.17)(-2.132,-2.000){2}{\rule{0.450pt}{0.400pt}}
\multiput(625.92,519.95)(-0.462,-0.447){3}{\rule{0.500pt}{0.108pt}}
\multiput(626.96,520.17)(-1.962,-3.000){2}{\rule{0.250pt}{0.400pt}}
\put(622,516.17){\rule{0.700pt}{0.400pt}}
\multiput(623.55,517.17)(-1.547,-2.000){2}{\rule{0.350pt}{0.400pt}}
\multiput(619.37,514.95)(-0.685,-0.447){3}{\rule{0.633pt}{0.108pt}}
\multiput(620.69,515.17)(-2.685,-3.000){2}{\rule{0.317pt}{0.400pt}}
\put(615,511.17){\rule{0.700pt}{0.400pt}}
\multiput(616.55,512.17)(-1.547,-2.000){2}{\rule{0.350pt}{0.400pt}}
\multiput(612.92,509.95)(-0.462,-0.447){3}{\rule{0.500pt}{0.108pt}}
\multiput(613.96,510.17)(-1.962,-3.000){2}{\rule{0.250pt}{0.400pt}}
\put(608,506.17){\rule{0.900pt}{0.400pt}}
\multiput(610.13,507.17)(-2.132,-2.000){2}{\rule{0.450pt}{0.400pt}}
\multiput(605.92,504.95)(-0.462,-0.447){3}{\rule{0.500pt}{0.108pt}}
\multiput(606.96,505.17)(-1.962,-3.000){2}{\rule{0.250pt}{0.400pt}}
\put(602,501.17){\rule{0.700pt}{0.400pt}}
\multiput(603.55,502.17)(-1.547,-2.000){2}{\rule{0.350pt}{0.400pt}}
\multiput(599.37,499.95)(-0.685,-0.447){3}{\rule{0.633pt}{0.108pt}}
\multiput(600.69,500.17)(-2.685,-3.000){2}{\rule{0.317pt}{0.400pt}}
\put(595,496.17){\rule{0.700pt}{0.400pt}}
\multiput(596.55,497.17)(-1.547,-2.000){2}{\rule{0.350pt}{0.400pt}}
\put(591,494.17){\rule{0.900pt}{0.400pt}}
\multiput(593.13,495.17)(-2.132,-2.000){2}{\rule{0.450pt}{0.400pt}}
\multiput(588.92,492.95)(-0.462,-0.447){3}{\rule{0.500pt}{0.108pt}}
\multiput(589.96,493.17)(-1.962,-3.000){2}{\rule{0.250pt}{0.400pt}}
\put(585,489.17){\rule{0.700pt}{0.400pt}}
\multiput(586.55,490.17)(-1.547,-2.000){2}{\rule{0.350pt}{0.400pt}}
\multiput(582.37,487.95)(-0.685,-0.447){3}{\rule{0.633pt}{0.108pt}}
\multiput(583.69,488.17)(-2.685,-3.000){2}{\rule{0.317pt}{0.400pt}}
\put(578,484.17){\rule{0.700pt}{0.400pt}}
\multiput(579.55,485.17)(-1.547,-2.000){2}{\rule{0.350pt}{0.400pt}}
\multiput(575.92,482.95)(-0.462,-0.447){3}{\rule{0.500pt}{0.108pt}}
\multiput(576.96,483.17)(-1.962,-3.000){2}{\rule{0.250pt}{0.400pt}}
\put(571,479.17){\rule{0.900pt}{0.400pt}}
\multiput(573.13,480.17)(-2.132,-2.000){2}{\rule{0.450pt}{0.400pt}}
\multiput(568.92,477.95)(-0.462,-0.447){3}{\rule{0.500pt}{0.108pt}}
\multiput(569.96,478.17)(-1.962,-3.000){2}{\rule{0.250pt}{0.400pt}}
\put(565,474.17){\rule{0.700pt}{0.400pt}}
\multiput(566.55,475.17)(-1.547,-2.000){2}{\rule{0.350pt}{0.400pt}}
\multiput(562.37,472.95)(-0.685,-0.447){3}{\rule{0.633pt}{0.108pt}}
\multiput(563.69,473.17)(-2.685,-3.000){2}{\rule{0.317pt}{0.400pt}}
\put(558,469.67){\rule{0.723pt}{0.400pt}}
\multiput(559.50,470.17)(-1.500,-1.000){2}{\rule{0.361pt}{0.400pt}}
\multiput(555.37,468.95)(-0.685,-0.447){3}{\rule{0.633pt}{0.108pt}}
\multiput(556.69,469.17)(-2.685,-3.000){2}{\rule{0.317pt}{0.400pt}}
\put(551,465.17){\rule{0.700pt}{0.400pt}}
\multiput(552.55,466.17)(-1.547,-2.000){2}{\rule{0.350pt}{0.400pt}}
\multiput(548.92,463.95)(-0.462,-0.447){3}{\rule{0.500pt}{0.108pt}}
\multiput(549.96,464.17)(-1.962,-3.000){2}{\rule{0.250pt}{0.400pt}}
\put(544,460.17){\rule{0.900pt}{0.400pt}}
\multiput(546.13,461.17)(-2.132,-2.000){2}{\rule{0.450pt}{0.400pt}}
\multiput(541.92,458.95)(-0.462,-0.447){3}{\rule{0.500pt}{0.108pt}}
\multiput(542.96,459.17)(-1.962,-3.000){2}{\rule{0.250pt}{0.400pt}}
\put(538,455.17){\rule{0.700pt}{0.400pt}}
\multiput(539.55,456.17)(-1.547,-2.000){2}{\rule{0.350pt}{0.400pt}}
\put(534,453.17){\rule{0.900pt}{0.400pt}}
\multiput(536.13,454.17)(-2.132,-2.000){2}{\rule{0.450pt}{0.400pt}}
\multiput(531.92,451.95)(-0.462,-0.447){3}{\rule{0.500pt}{0.108pt}}
\multiput(532.96,452.17)(-1.962,-3.000){2}{\rule{0.250pt}{0.400pt}}
\put(528,448.17){\rule{0.700pt}{0.400pt}}
\multiput(529.55,449.17)(-1.547,-2.000){2}{\rule{0.350pt}{0.400pt}}
\multiput(525.37,446.95)(-0.685,-0.447){3}{\rule{0.633pt}{0.108pt}}
\multiput(526.69,447.17)(-2.685,-3.000){2}{\rule{0.317pt}{0.400pt}}
\put(521,443.17){\rule{0.700pt}{0.400pt}}
\multiput(522.55,444.17)(-1.547,-2.000){2}{\rule{0.350pt}{0.400pt}}
\multiput(518.37,441.95)(-0.685,-0.447){3}{\rule{0.633pt}{0.108pt}}
\multiput(519.69,442.17)(-2.685,-3.000){2}{\rule{0.317pt}{0.400pt}}
\put(514,438.17){\rule{0.700pt}{0.400pt}}
\multiput(515.55,439.17)(-1.547,-2.000){2}{\rule{0.350pt}{0.400pt}}
\multiput(511.92,436.95)(-0.462,-0.447){3}{\rule{0.500pt}{0.108pt}}
\multiput(512.96,437.17)(-1.962,-3.000){2}{\rule{0.250pt}{0.400pt}}
\put(507,433.17){\rule{0.900pt}{0.400pt}}
\multiput(509.13,434.17)(-2.132,-2.000){2}{\rule{0.450pt}{0.400pt}}
\multiput(504.92,431.95)(-0.462,-0.447){3}{\rule{0.500pt}{0.108pt}}
\multiput(505.96,432.17)(-1.962,-3.000){2}{\rule{0.250pt}{0.400pt}}
\put(501,428.17){\rule{0.700pt}{0.400pt}}
\multiput(502.55,429.17)(-1.547,-2.000){2}{\rule{0.350pt}{0.400pt}}
\multiput(498.37,426.95)(-0.685,-0.447){3}{\rule{0.633pt}{0.108pt}}
\multiput(499.69,427.17)(-2.685,-3.000){2}{\rule{0.317pt}{0.400pt}}
\put(494,423.17){\rule{0.700pt}{0.400pt}}
\multiput(495.55,424.17)(-1.547,-2.000){2}{\rule{0.350pt}{0.400pt}}
\multiput(491.92,421.95)(-0.462,-0.447){3}{\rule{0.500pt}{0.108pt}}
\multiput(492.96,422.17)(-1.962,-3.000){2}{\rule{0.250pt}{0.400pt}}
\put(487,418.17){\rule{0.900pt}{0.400pt}}
\multiput(489.13,419.17)(-2.132,-2.000){2}{\rule{0.450pt}{0.400pt}}
\put(484,416.17){\rule{0.700pt}{0.400pt}}
\multiput(485.55,417.17)(-1.547,-2.000){2}{\rule{0.350pt}{0.400pt}}
\multiput(481.37,414.95)(-0.685,-0.447){3}{\rule{0.633pt}{0.108pt}}
\multiput(482.69,415.17)(-2.685,-3.000){2}{\rule{0.317pt}{0.400pt}}
\put(477,411.17){\rule{0.700pt}{0.400pt}}
\multiput(478.55,412.17)(-1.547,-2.000){2}{\rule{0.350pt}{0.400pt}}
\multiput(474.92,409.95)(-0.462,-0.447){3}{\rule{0.500pt}{0.108pt}}
\multiput(475.96,410.17)(-1.962,-3.000){2}{\rule{0.250pt}{0.400pt}}
\put(470,406.17){\rule{0.900pt}{0.400pt}}
\multiput(472.13,407.17)(-2.132,-2.000){2}{\rule{0.450pt}{0.400pt}}
\multiput(467.92,404.95)(-0.462,-0.447){3}{\rule{0.500pt}{0.108pt}}
\multiput(468.96,405.17)(-1.962,-3.000){2}{\rule{0.250pt}{0.400pt}}
\put(464,401.17){\rule{0.700pt}{0.400pt}}
\multiput(465.55,402.17)(-1.547,-2.000){2}{\rule{0.350pt}{0.400pt}}
\multiput(461.37,399.95)(-0.685,-0.447){3}{\rule{0.633pt}{0.108pt}}
\multiput(462.69,400.17)(-2.685,-3.000){2}{\rule{0.317pt}{0.400pt}}
\put(457,396.17){\rule{0.700pt}{0.400pt}}
\multiput(458.55,397.17)(-1.547,-2.000){2}{\rule{0.350pt}{0.400pt}}
\multiput(454.92,394.95)(-0.462,-0.447){3}{\rule{0.500pt}{0.108pt}}
\multiput(455.96,395.17)(-1.962,-3.000){2}{\rule{0.250pt}{0.400pt}}
\put(450,391.17){\rule{0.900pt}{0.400pt}}
\multiput(452.13,392.17)(-2.132,-2.000){2}{\rule{0.450pt}{0.400pt}}
\multiput(447.92,389.95)(-0.462,-0.447){3}{\rule{0.500pt}{0.108pt}}
\multiput(448.96,390.17)(-1.962,-3.000){2}{\rule{0.250pt}{0.400pt}}
\put(443,386.17){\rule{0.900pt}{0.400pt}}
\multiput(445.13,387.17)(-2.132,-2.000){2}{\rule{0.450pt}{0.400pt}}
\multiput(440.92,384.95)(-0.462,-0.447){3}{\rule{0.500pt}{0.108pt}}
\multiput(441.96,385.17)(-1.962,-3.000){2}{\rule{0.250pt}{0.400pt}}
\put(437,381.17){\rule{0.700pt}{0.400pt}}
\multiput(438.55,382.17)(-1.547,-2.000){2}{\rule{0.350pt}{0.400pt}}
\put(433,379.17){\rule{0.900pt}{0.400pt}}
\multiput(435.13,380.17)(-2.132,-2.000){2}{\rule{0.450pt}{0.400pt}}
\multiput(430.92,377.95)(-0.462,-0.447){3}{\rule{0.500pt}{0.108pt}}
\multiput(431.96,378.17)(-1.962,-3.000){2}{\rule{0.250pt}{0.400pt}}
\put(427,374.17){\rule{0.700pt}{0.400pt}}
\multiput(428.55,375.17)(-1.547,-2.000){2}{\rule{0.350pt}{0.400pt}}
\multiput(424.37,372.95)(-0.685,-0.447){3}{\rule{0.633pt}{0.108pt}}
\multiput(425.69,373.17)(-2.685,-3.000){2}{\rule{0.317pt}{0.400pt}}
\put(420,369.17){\rule{0.700pt}{0.400pt}}
\multiput(421.55,370.17)(-1.547,-2.000){2}{\rule{0.350pt}{0.400pt}}
\multiput(417.92,367.95)(-0.462,-0.447){3}{\rule{0.500pt}{0.108pt}}
\multiput(418.96,368.17)(-1.962,-3.000){2}{\rule{0.250pt}{0.400pt}}
\put(413,364.17){\rule{0.900pt}{0.400pt}}
\multiput(415.13,365.17)(-2.132,-2.000){2}{\rule{0.450pt}{0.400pt}}
\multiput(410.92,362.95)(-0.462,-0.447){3}{\rule{0.500pt}{0.108pt}}
\multiput(411.96,363.17)(-1.962,-3.000){2}{\rule{0.250pt}{0.400pt}}
\put(406,359.17){\rule{0.900pt}{0.400pt}}
\multiput(408.13,360.17)(-2.132,-2.000){2}{\rule{0.450pt}{0.400pt}}
\multiput(403.92,357.95)(-0.462,-0.447){3}{\rule{0.500pt}{0.108pt}}
\multiput(404.96,358.17)(-1.962,-3.000){2}{\rule{0.250pt}{0.400pt}}
\put(400,354.17){\rule{0.700pt}{0.400pt}}
\multiput(401.55,355.17)(-1.547,-2.000){2}{\rule{0.350pt}{0.400pt}}
\multiput(397.37,352.95)(-0.685,-0.447){3}{\rule{0.633pt}{0.108pt}}
\multiput(398.69,353.17)(-2.685,-3.000){2}{\rule{0.317pt}{0.400pt}}
\put(393,349.17){\rule{0.700pt}{0.400pt}}
\multiput(394.55,350.17)(-1.547,-2.000){2}{\rule{0.350pt}{0.400pt}}
\multiput(390.92,347.95)(-0.462,-0.447){3}{\rule{0.500pt}{0.108pt}}
\multiput(391.96,348.17)(-1.962,-3.000){2}{\rule{0.250pt}{0.400pt}}
\put(386,344.17){\rule{0.900pt}{0.400pt}}
\multiput(388.13,345.17)(-2.132,-2.000){2}{\rule{0.450pt}{0.400pt}}
\put(383,342.17){\rule{0.700pt}{0.400pt}}
\multiput(384.55,343.17)(-1.547,-2.000){2}{\rule{0.350pt}{0.400pt}}
\multiput(380.92,340.95)(-0.462,-0.447){3}{\rule{0.500pt}{0.108pt}}
\multiput(381.96,341.17)(-1.962,-3.000){2}{\rule{0.250pt}{0.400pt}}
\put(376,337.17){\rule{0.900pt}{0.400pt}}
\multiput(378.13,338.17)(-2.132,-2.000){2}{\rule{0.450pt}{0.400pt}}
\multiput(373.92,335.95)(-0.462,-0.447){3}{\rule{0.500pt}{0.108pt}}
\multiput(374.96,336.17)(-1.962,-3.000){2}{\rule{0.250pt}{0.400pt}}
\put(369,332.17){\rule{0.900pt}{0.400pt}}
\multiput(371.13,333.17)(-2.132,-2.000){2}{\rule{0.450pt}{0.400pt}}
\multiput(366.92,330.95)(-0.462,-0.447){3}{\rule{0.500pt}{0.108pt}}
\multiput(367.96,331.17)(-1.962,-3.000){2}{\rule{0.250pt}{0.400pt}}
\put(363,327.17){\rule{0.700pt}{0.400pt}}
\multiput(364.55,328.17)(-1.547,-2.000){2}{\rule{0.350pt}{0.400pt}}
\multiput(360.37,325.95)(-0.685,-0.447){3}{\rule{0.633pt}{0.108pt}}
\multiput(361.69,326.17)(-2.685,-3.000){2}{\rule{0.317pt}{0.400pt}}
\put(356,322.17){\rule{0.700pt}{0.400pt}}
\multiput(357.55,323.17)(-1.547,-2.000){2}{\rule{0.350pt}{0.400pt}}
\multiput(353.92,320.95)(-0.462,-0.447){3}{\rule{0.500pt}{0.108pt}}
\multiput(354.96,321.17)(-1.962,-3.000){2}{\rule{0.250pt}{0.400pt}}
\put(349,317.17){\rule{0.900pt}{0.400pt}}
\multiput(351.13,318.17)(-2.132,-2.000){2}{\rule{0.450pt}{0.400pt}}
\multiput(346.92,315.95)(-0.462,-0.447){3}{\rule{0.500pt}{0.108pt}}
\multiput(347.96,316.17)(-1.962,-3.000){2}{\rule{0.250pt}{0.400pt}}
\put(794,577.17){\rule{0.700pt}{0.400pt}}
\multiput(795.55,578.17)(-1.547,-2.000){2}{\rule{0.350pt}{0.400pt}}
\multiput(791.37,575.95)(-0.685,-0.447){3}{\rule{0.633pt}{0.108pt}}
\multiput(792.69,576.17)(-2.685,-3.000){2}{\rule{0.317pt}{0.400pt}}
\put(787,572.17){\rule{0.700pt}{0.400pt}}
\multiput(788.55,573.17)(-1.547,-2.000){2}{\rule{0.350pt}{0.400pt}}
\multiput(784.92,570.95)(-0.462,-0.447){3}{\rule{0.500pt}{0.108pt}}
\multiput(785.96,571.17)(-1.962,-3.000){2}{\rule{0.250pt}{0.400pt}}
\put(780,567.17){\rule{0.900pt}{0.400pt}}
\multiput(782.13,568.17)(-2.132,-2.000){2}{\rule{0.450pt}{0.400pt}}
\multiput(777.92,565.95)(-0.462,-0.447){3}{\rule{0.500pt}{0.108pt}}
\multiput(778.96,566.17)(-1.962,-3.000){2}{\rule{0.250pt}{0.400pt}}
\put(773,562.17){\rule{0.900pt}{0.400pt}}
\multiput(775.13,563.17)(-2.132,-2.000){2}{\rule{0.450pt}{0.400pt}}
\multiput(770.92,560.95)(-0.462,-0.447){3}{\rule{0.500pt}{0.108pt}}
\multiput(771.96,561.17)(-1.962,-3.000){2}{\rule{0.250pt}{0.400pt}}
\put(767,557.17){\rule{0.700pt}{0.400pt}}
\multiput(768.55,558.17)(-1.547,-2.000){2}{\rule{0.350pt}{0.400pt}}
\put(763,555.17){\rule{0.900pt}{0.400pt}}
\multiput(765.13,556.17)(-2.132,-2.000){2}{\rule{0.450pt}{0.400pt}}
\multiput(760.92,553.95)(-0.462,-0.447){3}{\rule{0.500pt}{0.108pt}}
\multiput(761.96,554.17)(-1.962,-3.000){2}{\rule{0.250pt}{0.400pt}}
\put(757,550.17){\rule{0.700pt}{0.400pt}}
\multiput(758.55,551.17)(-1.547,-2.000){2}{\rule{0.350pt}{0.400pt}}
\multiput(754.37,548.95)(-0.685,-0.447){3}{\rule{0.633pt}{0.108pt}}
\multiput(755.69,549.17)(-2.685,-3.000){2}{\rule{0.317pt}{0.400pt}}
\put(750,545.17){\rule{0.700pt}{0.400pt}}
\multiput(751.55,546.17)(-1.547,-2.000){2}{\rule{0.350pt}{0.400pt}}
\put(748.17,542){\rule{0.400pt}{0.700pt}}
\multiput(749.17,543.55)(-2.000,-1.547){2}{\rule{0.400pt}{0.350pt}}
\put(744,540.17){\rule{0.900pt}{0.400pt}}
\multiput(746.13,541.17)(-2.132,-2.000){2}{\rule{0.450pt}{0.400pt}}
\multiput(741.92,538.95)(-0.462,-0.447){3}{\rule{0.500pt}{0.108pt}}
\multiput(742.96,539.17)(-1.962,-3.000){2}{\rule{0.250pt}{0.400pt}}
\put(737,535.17){\rule{0.900pt}{0.400pt}}
\multiput(739.13,536.17)(-2.132,-2.000){2}{\rule{0.450pt}{0.400pt}}
\multiput(734.92,533.95)(-0.462,-0.447){3}{\rule{0.500pt}{0.108pt}}
\multiput(735.96,534.17)(-1.962,-3.000){2}{\rule{0.250pt}{0.400pt}}
\put(731,530.17){\rule{0.700pt}{0.400pt}}
\multiput(732.55,531.17)(-1.547,-2.000){2}{\rule{0.350pt}{0.400pt}}
\multiput(728.37,528.95)(-0.685,-0.447){3}{\rule{0.633pt}{0.108pt}}
\multiput(729.69,529.17)(-2.685,-3.000){2}{\rule{0.317pt}{0.400pt}}
\put(724,525.17){\rule{0.700pt}{0.400pt}}
\multiput(725.55,526.17)(-1.547,-2.000){2}{\rule{0.350pt}{0.400pt}}
\multiput(721.92,523.95)(-0.462,-0.447){3}{\rule{0.500pt}{0.108pt}}
\multiput(722.96,524.17)(-1.962,-3.000){2}{\rule{0.250pt}{0.400pt}}
\put(717,520.17){\rule{0.900pt}{0.400pt}}
\multiput(719.13,521.17)(-2.132,-2.000){2}{\rule{0.450pt}{0.400pt}}
\multiput(714.92,518.95)(-0.462,-0.447){3}{\rule{0.500pt}{0.108pt}}
\multiput(715.96,519.17)(-1.962,-3.000){2}{\rule{0.250pt}{0.400pt}}
\put(711,515.17){\rule{0.700pt}{0.400pt}}
\multiput(712.55,516.17)(-1.547,-2.000){2}{\rule{0.350pt}{0.400pt}}
\put(707,513.17){\rule{0.900pt}{0.400pt}}
\multiput(709.13,514.17)(-2.132,-2.000){2}{\rule{0.450pt}{0.400pt}}
\multiput(704.92,511.95)(-0.462,-0.447){3}{\rule{0.500pt}{0.108pt}}
\multiput(705.96,512.17)(-1.962,-3.000){2}{\rule{0.250pt}{0.400pt}}
\put(700,508.17){\rule{0.900pt}{0.400pt}}
\multiput(702.13,509.17)(-2.132,-2.000){2}{\rule{0.450pt}{0.400pt}}
\multiput(697.92,506.95)(-0.462,-0.447){3}{\rule{0.500pt}{0.108pt}}
\multiput(698.96,507.17)(-1.962,-3.000){2}{\rule{0.250pt}{0.400pt}}
\put(694,503.17){\rule{0.700pt}{0.400pt}}
\multiput(695.55,504.17)(-1.547,-2.000){2}{\rule{0.350pt}{0.400pt}}
\multiput(691.37,501.95)(-0.685,-0.447){3}{\rule{0.633pt}{0.108pt}}
\multiput(692.69,502.17)(-2.685,-3.000){2}{\rule{0.317pt}{0.400pt}}
\put(687,498.17){\rule{0.700pt}{0.400pt}}
\multiput(688.55,499.17)(-1.547,-2.000){2}{\rule{0.350pt}{0.400pt}}
\multiput(684.92,496.95)(-0.462,-0.447){3}{\rule{0.500pt}{0.108pt}}
\multiput(685.96,497.17)(-1.962,-3.000){2}{\rule{0.250pt}{0.400pt}}
\put(680,493.17){\rule{0.900pt}{0.400pt}}
\multiput(682.13,494.17)(-2.132,-2.000){2}{\rule{0.450pt}{0.400pt}}
\multiput(677.92,491.95)(-0.462,-0.447){3}{\rule{0.500pt}{0.108pt}}
\multiput(678.96,492.17)(-1.962,-3.000){2}{\rule{0.250pt}{0.400pt}}
\put(674,488.17){\rule{0.700pt}{0.400pt}}
\multiput(675.55,489.17)(-1.547,-2.000){2}{\rule{0.350pt}{0.400pt}}
\multiput(671.37,486.95)(-0.685,-0.447){3}{\rule{0.633pt}{0.108pt}}
\multiput(672.69,487.17)(-2.685,-3.000){2}{\rule{0.317pt}{0.400pt}}
\put(667,483.17){\rule{0.700pt}{0.400pt}}
\multiput(668.55,484.17)(-1.547,-2.000){2}{\rule{0.350pt}{0.400pt}}
\multiput(664.37,481.95)(-0.685,-0.447){3}{\rule{0.633pt}{0.108pt}}
\multiput(665.69,482.17)(-2.685,-3.000){2}{\rule{0.317pt}{0.400pt}}
\put(660,478.17){\rule{0.700pt}{0.400pt}}
\multiput(661.55,479.17)(-1.547,-2.000){2}{\rule{0.350pt}{0.400pt}}
\put(657,476.17){\rule{0.700pt}{0.400pt}}
\multiput(658.55,477.17)(-1.547,-2.000){2}{\rule{0.350pt}{0.400pt}}
\multiput(654.37,474.95)(-0.685,-0.447){3}{\rule{0.633pt}{0.108pt}}
\multiput(655.69,475.17)(-2.685,-3.000){2}{\rule{0.317pt}{0.400pt}}
\put(650,471.17){\rule{0.700pt}{0.400pt}}
\multiput(651.55,472.17)(-1.547,-2.000){2}{\rule{0.350pt}{0.400pt}}
\put(647,469.17){\rule{0.700pt}{0.400pt}}
\multiput(648.55,470.17)(-1.547,-2.000){2}{\rule{0.350pt}{0.400pt}}
\put(643,467.17){\rule{0.900pt}{0.400pt}}
\multiput(645.13,468.17)(-2.132,-2.000){2}{\rule{0.450pt}{0.400pt}}
\multiput(640.92,465.95)(-0.462,-0.447){3}{\rule{0.500pt}{0.108pt}}
\multiput(641.96,466.17)(-1.962,-3.000){2}{\rule{0.250pt}{0.400pt}}
\put(637,462.17){\rule{0.700pt}{0.400pt}}
\multiput(638.55,463.17)(-1.547,-2.000){2}{\rule{0.350pt}{0.400pt}}
\multiput(634.37,460.95)(-0.685,-0.447){3}{\rule{0.633pt}{0.108pt}}
\multiput(635.69,461.17)(-2.685,-3.000){2}{\rule{0.317pt}{0.400pt}}
\put(630,457.17){\rule{0.700pt}{0.400pt}}
\multiput(631.55,458.17)(-1.547,-2.000){2}{\rule{0.350pt}{0.400pt}}
\multiput(627.37,455.95)(-0.685,-0.447){3}{\rule{0.633pt}{0.108pt}}
\multiput(628.69,456.17)(-2.685,-3.000){2}{\rule{0.317pt}{0.400pt}}
\put(623,452.17){\rule{0.700pt}{0.400pt}}
\multiput(624.55,453.17)(-1.547,-2.000){2}{\rule{0.350pt}{0.400pt}}
\multiput(620.92,450.95)(-0.462,-0.447){3}{\rule{0.500pt}{0.108pt}}
\multiput(621.96,451.17)(-1.962,-3.000){2}{\rule{0.250pt}{0.400pt}}
\put(616,447.17){\rule{0.900pt}{0.400pt}}
\multiput(618.13,448.17)(-2.132,-2.000){2}{\rule{0.450pt}{0.400pt}}
\multiput(613.92,445.95)(-0.462,-0.447){3}{\rule{0.500pt}{0.108pt}}
\multiput(614.96,446.17)(-1.962,-3.000){2}{\rule{0.250pt}{0.400pt}}
\put(610,442.17){\rule{0.700pt}{0.400pt}}
\multiput(611.55,443.17)(-1.547,-2.000){2}{\rule{0.350pt}{0.400pt}}
\put(606,440.17){\rule{0.900pt}{0.400pt}}
\multiput(608.13,441.17)(-2.132,-2.000){2}{\rule{0.450pt}{0.400pt}}
\multiput(603.92,438.95)(-0.462,-0.447){3}{\rule{0.500pt}{0.108pt}}
\multiput(604.96,439.17)(-1.962,-3.000){2}{\rule{0.250pt}{0.400pt}}
\put(600,435.17){\rule{0.700pt}{0.400pt}}
\multiput(601.55,436.17)(-1.547,-2.000){2}{\rule{0.350pt}{0.400pt}}
\multiput(597.37,433.95)(-0.685,-0.447){3}{\rule{0.633pt}{0.108pt}}
\multiput(598.69,434.17)(-2.685,-3.000){2}{\rule{0.317pt}{0.400pt}}
\put(593,430.17){\rule{0.700pt}{0.400pt}}
\multiput(594.55,431.17)(-1.547,-2.000){2}{\rule{0.350pt}{0.400pt}}
\multiput(590.37,428.95)(-0.685,-0.447){3}{\rule{0.633pt}{0.108pt}}
\multiput(591.69,429.17)(-2.685,-3.000){2}{\rule{0.317pt}{0.400pt}}
\put(586,425.17){\rule{0.700pt}{0.400pt}}
\multiput(587.55,426.17)(-1.547,-2.000){2}{\rule{0.350pt}{0.400pt}}
\multiput(583.92,423.95)(-0.462,-0.447){3}{\rule{0.500pt}{0.108pt}}
\multiput(584.96,424.17)(-1.962,-3.000){2}{\rule{0.250pt}{0.400pt}}
\put(579,420.17){\rule{0.900pt}{0.400pt}}
\multiput(581.13,421.17)(-2.132,-2.000){2}{\rule{0.450pt}{0.400pt}}
\multiput(576.92,418.95)(-0.462,-0.447){3}{\rule{0.500pt}{0.108pt}}
\multiput(577.96,419.17)(-1.962,-3.000){2}{\rule{0.250pt}{0.400pt}}
\put(573,415.17){\rule{0.700pt}{0.400pt}}
\multiput(574.55,416.17)(-1.547,-2.000){2}{\rule{0.350pt}{0.400pt}}
\multiput(570.37,413.95)(-0.685,-0.447){3}{\rule{0.633pt}{0.108pt}}
\multiput(571.69,414.17)(-2.685,-3.000){2}{\rule{0.317pt}{0.400pt}}
\put(566,410.17){\rule{0.700pt}{0.400pt}}
\multiput(567.55,411.17)(-1.547,-2.000){2}{\rule{0.350pt}{0.400pt}}
\multiput(563.92,408.95)(-0.462,-0.447){3}{\rule{0.500pt}{0.108pt}}
\multiput(564.96,409.17)(-1.962,-3.000){2}{\rule{0.250pt}{0.400pt}}
\put(559,405.17){\rule{0.900pt}{0.400pt}}
\multiput(561.13,406.17)(-2.132,-2.000){2}{\rule{0.450pt}{0.400pt}}
\put(556,403.17){\rule{0.700pt}{0.400pt}}
\multiput(557.55,404.17)(-1.547,-2.000){2}{\rule{0.350pt}{0.400pt}}
\multiput(553.37,401.95)(-0.685,-0.447){3}{\rule{0.633pt}{0.108pt}}
\multiput(554.69,402.17)(-2.685,-3.000){2}{\rule{0.317pt}{0.400pt}}
\put(549,398.17){\rule{0.700pt}{0.400pt}}
\multiput(550.55,399.17)(-1.547,-2.000){2}{\rule{0.350pt}{0.400pt}}
\multiput(546.92,396.95)(-0.462,-0.447){3}{\rule{0.500pt}{0.108pt}}
\multiput(547.96,397.17)(-1.962,-3.000){2}{\rule{0.250pt}{0.400pt}}
\put(542,393.17){\rule{0.900pt}{0.400pt}}
\multiput(544.13,394.17)(-2.132,-2.000){2}{\rule{0.450pt}{0.400pt}}
\multiput(539.92,391.95)(-0.462,-0.447){3}{\rule{0.500pt}{0.108pt}}
\multiput(540.96,392.17)(-1.962,-3.000){2}{\rule{0.250pt}{0.400pt}}
\put(536,388.17){\rule{0.700pt}{0.400pt}}
\multiput(537.55,389.17)(-1.547,-2.000){2}{\rule{0.350pt}{0.400pt}}
\multiput(533.37,386.95)(-0.685,-0.447){3}{\rule{0.633pt}{0.108pt}}
\multiput(534.69,387.17)(-2.685,-3.000){2}{\rule{0.317pt}{0.400pt}}
\put(529,383.17){\rule{0.700pt}{0.400pt}}
\multiput(530.55,384.17)(-1.547,-2.000){2}{\rule{0.350pt}{0.400pt}}
\multiput(526.92,381.95)(-0.462,-0.447){3}{\rule{0.500pt}{0.108pt}}
\multiput(527.96,382.17)(-1.962,-3.000){2}{\rule{0.250pt}{0.400pt}}
\put(522,378.17){\rule{0.900pt}{0.400pt}}
\multiput(524.13,379.17)(-2.132,-2.000){2}{\rule{0.450pt}{0.400pt}}
\multiput(519.92,376.95)(-0.462,-0.447){3}{\rule{0.500pt}{0.108pt}}
\multiput(520.96,377.17)(-1.962,-3.000){2}{\rule{0.250pt}{0.400pt}}
\put(515,373.17){\rule{0.900pt}{0.400pt}}
\multiput(517.13,374.17)(-2.132,-2.000){2}{\rule{0.450pt}{0.400pt}}
\multiput(512.92,371.95)(-0.462,-0.447){3}{\rule{0.500pt}{0.108pt}}
\multiput(513.96,372.17)(-1.962,-3.000){2}{\rule{0.250pt}{0.400pt}}
\put(509,368.17){\rule{0.700pt}{0.400pt}}
\multiput(510.55,369.17)(-1.547,-2.000){2}{\rule{0.350pt}{0.400pt}}
\put(505,366.17){\rule{0.900pt}{0.400pt}}
\multiput(507.13,367.17)(-2.132,-2.000){2}{\rule{0.450pt}{0.400pt}}
\multiput(502.92,364.95)(-0.462,-0.447){3}{\rule{0.500pt}{0.108pt}}
\multiput(503.96,365.17)(-1.962,-3.000){2}{\rule{0.250pt}{0.400pt}}
\put(499,361.17){\rule{0.700pt}{0.400pt}}
\multiput(500.55,362.17)(-1.547,-2.000){2}{\rule{0.350pt}{0.400pt}}
\multiput(496.37,359.95)(-0.685,-0.447){3}{\rule{0.633pt}{0.108pt}}
\multiput(497.69,360.17)(-2.685,-3.000){2}{\rule{0.317pt}{0.400pt}}
\put(492,356.17){\rule{0.700pt}{0.400pt}}
\multiput(493.55,357.17)(-1.547,-2.000){2}{\rule{0.350pt}{0.400pt}}
\multiput(489.92,354.95)(-0.462,-0.447){3}{\rule{0.500pt}{0.108pt}}
\multiput(490.96,355.17)(-1.962,-3.000){2}{\rule{0.250pt}{0.400pt}}
\put(485,351.17){\rule{0.900pt}{0.400pt}}
\multiput(487.13,352.17)(-2.132,-2.000){2}{\rule{0.450pt}{0.400pt}}
\multiput(482.92,349.95)(-0.462,-0.447){3}{\rule{0.500pt}{0.108pt}}
\multiput(483.96,350.17)(-1.962,-3.000){2}{\rule{0.250pt}{0.400pt}}
\put(478,346.17){\rule{0.900pt}{0.400pt}}
\multiput(480.13,347.17)(-2.132,-2.000){2}{\rule{0.450pt}{0.400pt}}
\multiput(475.92,344.95)(-0.462,-0.447){3}{\rule{0.500pt}{0.108pt}}
\multiput(476.96,345.17)(-1.962,-3.000){2}{\rule{0.250pt}{0.400pt}}
\put(472,341.17){\rule{0.700pt}{0.400pt}}
\multiput(473.55,342.17)(-1.547,-2.000){2}{\rule{0.350pt}{0.400pt}}
\multiput(469.37,339.95)(-0.685,-0.447){3}{\rule{0.633pt}{0.108pt}}
\multiput(470.69,340.17)(-2.685,-3.000){2}{\rule{0.317pt}{0.400pt}}
\put(465,336.17){\rule{0.700pt}{0.400pt}}
\multiput(466.55,337.17)(-1.547,-2.000){2}{\rule{0.350pt}{0.400pt}}
\multiput(913.92,599.95)(-0.462,-0.447){3}{\rule{0.500pt}{0.108pt}}
\multiput(914.96,600.17)(-1.962,-3.000){2}{\rule{0.250pt}{0.400pt}}
\put(909,596.17){\rule{0.900pt}{0.400pt}}
\multiput(911.13,597.17)(-2.132,-2.000){2}{\rule{0.450pt}{0.400pt}}
\multiput(906.92,594.95)(-0.462,-0.447){3}{\rule{0.500pt}{0.108pt}}
\multiput(907.96,595.17)(-1.962,-3.000){2}{\rule{0.250pt}{0.400pt}}
\put(903,591.17){\rule{0.700pt}{0.400pt}}
\multiput(904.55,592.17)(-1.547,-2.000){2}{\rule{0.350pt}{0.400pt}}
\multiput(900.37,589.95)(-0.685,-0.447){3}{\rule{0.633pt}{0.108pt}}
\multiput(901.69,590.17)(-2.685,-3.000){2}{\rule{0.317pt}{0.400pt}}
\put(896,586.17){\rule{0.700pt}{0.400pt}}
\multiput(897.55,587.17)(-1.547,-2.000){2}{\rule{0.350pt}{0.400pt}}
\multiput(893.37,584.95)(-0.685,-0.447){3}{\rule{0.633pt}{0.108pt}}
\multiput(894.69,585.17)(-2.685,-3.000){2}{\rule{0.317pt}{0.400pt}}
\put(889,581.17){\rule{0.700pt}{0.400pt}}
\multiput(890.55,582.17)(-1.547,-2.000){2}{\rule{0.350pt}{0.400pt}}
\put(886,579.17){\rule{0.700pt}{0.400pt}}
\multiput(887.55,580.17)(-1.547,-2.000){2}{\rule{0.350pt}{0.400pt}}
\multiput(883.37,577.95)(-0.685,-0.447){3}{\rule{0.633pt}{0.108pt}}
\multiput(884.69,578.17)(-2.685,-3.000){2}{\rule{0.317pt}{0.400pt}}
\put(879,574.17){\rule{0.700pt}{0.400pt}}
\multiput(880.55,575.17)(-1.547,-2.000){2}{\rule{0.350pt}{0.400pt}}
\multiput(876.92,572.95)(-0.462,-0.447){3}{\rule{0.500pt}{0.108pt}}
\multiput(877.96,573.17)(-1.962,-3.000){2}{\rule{0.250pt}{0.400pt}}
\put(872,569.17){\rule{0.900pt}{0.400pt}}
\multiput(874.13,570.17)(-2.132,-2.000){2}{\rule{0.450pt}{0.400pt}}
\multiput(869.92,567.95)(-0.462,-0.447){3}{\rule{0.500pt}{0.108pt}}
\multiput(870.96,568.17)(-1.962,-3.000){2}{\rule{0.250pt}{0.400pt}}
\put(866,564.17){\rule{0.700pt}{0.400pt}}
\multiput(867.55,565.17)(-1.547,-2.000){2}{\rule{0.350pt}{0.400pt}}
\multiput(863.37,562.95)(-0.685,-0.447){3}{\rule{0.633pt}{0.108pt}}
\multiput(864.69,563.17)(-2.685,-3.000){2}{\rule{0.317pt}{0.400pt}}
\put(859,559.17){\rule{0.700pt}{0.400pt}}
\multiput(860.55,560.17)(-1.547,-2.000){2}{\rule{0.350pt}{0.400pt}}
\multiput(856.37,557.95)(-0.685,-0.447){3}{\rule{0.633pt}{0.108pt}}
\multiput(857.69,558.17)(-2.685,-3.000){2}{\rule{0.317pt}{0.400pt}}
\put(852,554.17){\rule{0.700pt}{0.400pt}}
\multiput(853.55,555.17)(-1.547,-2.000){2}{\rule{0.350pt}{0.400pt}}
\multiput(849.92,552.95)(-0.462,-0.447){3}{\rule{0.500pt}{0.108pt}}
\multiput(850.96,553.17)(-1.962,-3.000){2}{\rule{0.250pt}{0.400pt}}
\put(845,549.17){\rule{0.900pt}{0.400pt}}
\multiput(847.13,550.17)(-2.132,-2.000){2}{\rule{0.450pt}{0.400pt}}
\multiput(842.92,547.95)(-0.462,-0.447){3}{\rule{0.500pt}{0.108pt}}
\multiput(843.96,548.17)(-1.962,-3.000){2}{\rule{0.250pt}{0.400pt}}
\put(839,544.17){\rule{0.700pt}{0.400pt}}
\multiput(840.55,545.17)(-1.547,-2.000){2}{\rule{0.350pt}{0.400pt}}
\multiput(836.37,542.95)(-0.685,-0.447){3}{\rule{0.633pt}{0.108pt}}
\multiput(837.69,543.17)(-2.685,-3.000){2}{\rule{0.317pt}{0.400pt}}
\put(832,539.17){\rule{0.700pt}{0.400pt}}
\multiput(833.55,540.17)(-1.547,-2.000){2}{\rule{0.350pt}{0.400pt}}
\put(829,537.17){\rule{0.700pt}{0.400pt}}
\multiput(830.55,538.17)(-1.547,-2.000){2}{\rule{0.350pt}{0.400pt}}
\multiput(826.37,535.95)(-0.685,-0.447){3}{\rule{0.633pt}{0.108pt}}
\multiput(827.69,536.17)(-2.685,-3.000){2}{\rule{0.317pt}{0.400pt}}
\put(822,532.17){\rule{0.700pt}{0.400pt}}
\multiput(823.55,533.17)(-1.547,-2.000){2}{\rule{0.350pt}{0.400pt}}
\multiput(819.37,530.95)(-0.685,-0.447){3}{\rule{0.633pt}{0.108pt}}
\multiput(820.69,531.17)(-2.685,-3.000){2}{\rule{0.317pt}{0.400pt}}
\put(815,527.17){\rule{0.700pt}{0.400pt}}
\multiput(816.55,528.17)(-1.547,-2.000){2}{\rule{0.350pt}{0.400pt}}
\multiput(812.92,525.95)(-0.462,-0.447){3}{\rule{0.500pt}{0.108pt}}
\multiput(813.96,526.17)(-1.962,-3.000){2}{\rule{0.250pt}{0.400pt}}
\put(808,522.17){\rule{0.900pt}{0.400pt}}
\multiput(810.13,523.17)(-2.132,-2.000){2}{\rule{0.450pt}{0.400pt}}
\multiput(805.92,520.95)(-0.462,-0.447){3}{\rule{0.500pt}{0.108pt}}
\multiput(806.96,521.17)(-1.962,-3.000){2}{\rule{0.250pt}{0.400pt}}
\put(802,517.17){\rule{0.700pt}{0.400pt}}
\multiput(803.55,518.17)(-1.547,-2.000){2}{\rule{0.350pt}{0.400pt}}
\multiput(799.37,515.95)(-0.685,-0.447){3}{\rule{0.633pt}{0.108pt}}
\multiput(800.69,516.17)(-2.685,-3.000){2}{\rule{0.317pt}{0.400pt}}
\put(795,512.17){\rule{0.700pt}{0.400pt}}
\multiput(796.55,513.17)(-1.547,-2.000){2}{\rule{0.350pt}{0.400pt}}
\multiput(792.92,510.95)(-0.462,-0.447){3}{\rule{0.500pt}{0.108pt}}
\multiput(793.96,511.17)(-1.962,-3.000){2}{\rule{0.250pt}{0.400pt}}
\put(788,507.17){\rule{0.900pt}{0.400pt}}
\multiput(790.13,508.17)(-2.132,-2.000){2}{\rule{0.450pt}{0.400pt}}
\multiput(785.92,505.95)(-0.462,-0.447){3}{\rule{0.500pt}{0.108pt}}
\multiput(786.96,506.17)(-1.962,-3.000){2}{\rule{0.250pt}{0.400pt}}
\put(781,502.17){\rule{0.900pt}{0.400pt}}
\multiput(783.13,503.17)(-2.132,-2.000){2}{\rule{0.450pt}{0.400pt}}
\put(778,500.17){\rule{0.700pt}{0.400pt}}
\multiput(779.55,501.17)(-1.547,-2.000){2}{\rule{0.350pt}{0.400pt}}
\multiput(775.92,498.95)(-0.462,-0.447){3}{\rule{0.500pt}{0.108pt}}
\multiput(776.96,499.17)(-1.962,-3.000){2}{\rule{0.250pt}{0.400pt}}
\put(771,495.17){\rule{0.900pt}{0.400pt}}
\multiput(773.13,496.17)(-2.132,-2.000){2}{\rule{0.450pt}{0.400pt}}
\multiput(768.92,493.95)(-0.462,-0.447){3}{\rule{0.500pt}{0.108pt}}
\multiput(769.96,494.17)(-1.962,-3.000){2}{\rule{0.250pt}{0.400pt}}
\put(765,490.17){\rule{0.700pt}{0.400pt}}
\multiput(766.55,491.17)(-1.547,-2.000){2}{\rule{0.350pt}{0.400pt}}
\multiput(762.37,488.95)(-0.685,-0.447){3}{\rule{0.633pt}{0.108pt}}
\multiput(763.69,489.17)(-2.685,-3.000){2}{\rule{0.317pt}{0.400pt}}
\put(758,485.17){\rule{0.700pt}{0.400pt}}
\multiput(759.55,486.17)(-1.547,-2.000){2}{\rule{0.350pt}{0.400pt}}
\multiput(755.92,483.95)(-0.462,-0.447){3}{\rule{0.500pt}{0.108pt}}
\multiput(756.96,484.17)(-1.962,-3.000){2}{\rule{0.250pt}{0.400pt}}
\put(751,480.17){\rule{0.900pt}{0.400pt}}
\multiput(753.13,481.17)(-2.132,-2.000){2}{\rule{0.450pt}{0.400pt}}
\put(749.17,477){\rule{0.400pt}{0.700pt}}
\multiput(750.17,478.55)(-2.000,-1.547){2}{\rule{0.400pt}{0.350pt}}
\put(745,475.17){\rule{0.900pt}{0.400pt}}
\multiput(747.13,476.17)(-2.132,-2.000){2}{\rule{0.450pt}{0.400pt}}
\multiput(742.92,473.95)(-0.462,-0.447){3}{\rule{0.500pt}{0.108pt}}
\multiput(743.96,474.17)(-1.962,-3.000){2}{\rule{0.250pt}{0.400pt}}
\put(739,470.17){\rule{0.700pt}{0.400pt}}
\multiput(740.55,471.17)(-1.547,-2.000){2}{\rule{0.350pt}{0.400pt}}
\put(735,468.17){\rule{0.900pt}{0.400pt}}
\multiput(737.13,469.17)(-2.132,-2.000){2}{\rule{0.450pt}{0.400pt}}
\put(732,466.17){\rule{0.700pt}{0.400pt}}
\multiput(733.55,467.17)(-1.547,-2.000){2}{\rule{0.350pt}{0.400pt}}
\put(729,464.17){\rule{0.700pt}{0.400pt}}
\multiput(730.55,465.17)(-1.547,-2.000){2}{\rule{0.350pt}{0.400pt}}
\multiput(726.37,462.95)(-0.685,-0.447){3}{\rule{0.633pt}{0.108pt}}
\multiput(727.69,463.17)(-2.685,-3.000){2}{\rule{0.317pt}{0.400pt}}
\put(722,459.17){\rule{0.700pt}{0.400pt}}
\multiput(723.55,460.17)(-1.547,-2.000){2}{\rule{0.350pt}{0.400pt}}
\multiput(719.92,457.95)(-0.462,-0.447){3}{\rule{0.500pt}{0.108pt}}
\multiput(720.96,458.17)(-1.962,-3.000){2}{\rule{0.250pt}{0.400pt}}
\put(715,454.17){\rule{0.900pt}{0.400pt}}
\multiput(717.13,455.17)(-2.132,-2.000){2}{\rule{0.450pt}{0.400pt}}
\multiput(712.92,452.95)(-0.462,-0.447){3}{\rule{0.500pt}{0.108pt}}
\multiput(713.96,453.17)(-1.962,-3.000){2}{\rule{0.250pt}{0.400pt}}
\put(708,449.17){\rule{0.900pt}{0.400pt}}
\multiput(710.13,450.17)(-2.132,-2.000){2}{\rule{0.450pt}{0.400pt}}
\multiput(705.92,447.95)(-0.462,-0.447){3}{\rule{0.500pt}{0.108pt}}
\multiput(706.96,448.17)(-1.962,-3.000){2}{\rule{0.250pt}{0.400pt}}
\put(702,444.17){\rule{0.700pt}{0.400pt}}
\multiput(703.55,445.17)(-1.547,-2.000){2}{\rule{0.350pt}{0.400pt}}
\multiput(699.37,442.95)(-0.685,-0.447){3}{\rule{0.633pt}{0.108pt}}
\multiput(700.69,443.17)(-2.685,-3.000){2}{\rule{0.317pt}{0.400pt}}
\put(695,439.17){\rule{0.700pt}{0.400pt}}
\multiput(696.55,440.17)(-1.547,-2.000){2}{\rule{0.350pt}{0.400pt}}
\multiput(692.92,437.95)(-0.462,-0.447){3}{\rule{0.500pt}{0.108pt}}
\multiput(693.96,438.17)(-1.962,-3.000){2}{\rule{0.250pt}{0.400pt}}
\put(688,434.17){\rule{0.900pt}{0.400pt}}
\multiput(690.13,435.17)(-2.132,-2.000){2}{\rule{0.450pt}{0.400pt}}
\multiput(685.92,432.95)(-0.462,-0.447){3}{\rule{0.500pt}{0.108pt}}
\multiput(686.96,433.17)(-1.962,-3.000){2}{\rule{0.250pt}{0.400pt}}
\put(682,429.17){\rule{0.700pt}{0.400pt}}
\multiput(683.55,430.17)(-1.547,-2.000){2}{\rule{0.350pt}{0.400pt}}
\put(678,427.17){\rule{0.900pt}{0.400pt}}
\multiput(680.13,428.17)(-2.132,-2.000){2}{\rule{0.450pt}{0.400pt}}
\multiput(675.92,425.95)(-0.462,-0.447){3}{\rule{0.500pt}{0.108pt}}
\multiput(676.96,426.17)(-1.962,-3.000){2}{\rule{0.250pt}{0.400pt}}
\put(671,422.17){\rule{0.900pt}{0.400pt}}
\multiput(673.13,423.17)(-2.132,-2.000){2}{\rule{0.450pt}{0.400pt}}
\multiput(668.92,420.95)(-0.462,-0.447){3}{\rule{0.500pt}{0.108pt}}
\multiput(669.96,421.17)(-1.962,-3.000){2}{\rule{0.250pt}{0.400pt}}
\put(665,417.17){\rule{0.700pt}{0.400pt}}
\multiput(666.55,418.17)(-1.547,-2.000){2}{\rule{0.350pt}{0.400pt}}
\multiput(662.37,415.95)(-0.685,-0.447){3}{\rule{0.633pt}{0.108pt}}
\multiput(663.69,416.17)(-2.685,-3.000){2}{\rule{0.317pt}{0.400pt}}
\put(658,412.17){\rule{0.700pt}{0.400pt}}
\multiput(659.55,413.17)(-1.547,-2.000){2}{\rule{0.350pt}{0.400pt}}
\multiput(655.92,410.95)(-0.462,-0.447){3}{\rule{0.500pt}{0.108pt}}
\multiput(656.96,411.17)(-1.962,-3.000){2}{\rule{0.250pt}{0.400pt}}
\put(651,407.17){\rule{0.900pt}{0.400pt}}
\multiput(653.13,408.17)(-2.132,-2.000){2}{\rule{0.450pt}{0.400pt}}
\multiput(648.92,405.95)(-0.462,-0.447){3}{\rule{0.500pt}{0.108pt}}
\multiput(649.96,406.17)(-1.962,-3.000){2}{\rule{0.250pt}{0.400pt}}
\put(645,402.17){\rule{0.700pt}{0.400pt}}
\multiput(646.55,403.17)(-1.547,-2.000){2}{\rule{0.350pt}{0.400pt}}
\multiput(642.37,400.95)(-0.685,-0.447){3}{\rule{0.633pt}{0.108pt}}
\multiput(643.69,401.17)(-2.685,-3.000){2}{\rule{0.317pt}{0.400pt}}
\put(638,397.17){\rule{0.700pt}{0.400pt}}
\multiput(639.55,398.17)(-1.547,-2.000){2}{\rule{0.350pt}{0.400pt}}
\multiput(635.37,395.95)(-0.685,-0.447){3}{\rule{0.633pt}{0.108pt}}
\multiput(636.69,396.17)(-2.685,-3.000){2}{\rule{0.317pt}{0.400pt}}
\put(631,392.17){\rule{0.700pt}{0.400pt}}
\multiput(632.55,393.17)(-1.547,-2.000){2}{\rule{0.350pt}{0.400pt}}
\put(628,390.17){\rule{0.700pt}{0.400pt}}
\multiput(629.55,391.17)(-1.547,-2.000){2}{\rule{0.350pt}{0.400pt}}
\multiput(625.37,388.95)(-0.685,-0.447){3}{\rule{0.633pt}{0.108pt}}
\multiput(626.69,389.17)(-2.685,-3.000){2}{\rule{0.317pt}{0.400pt}}
\put(621,385.17){\rule{0.700pt}{0.400pt}}
\multiput(622.55,386.17)(-1.547,-2.000){2}{\rule{0.350pt}{0.400pt}}
\multiput(618.92,383.95)(-0.462,-0.447){3}{\rule{0.500pt}{0.108pt}}
\multiput(619.96,384.17)(-1.962,-3.000){2}{\rule{0.250pt}{0.400pt}}
\put(614,380.17){\rule{0.900pt}{0.400pt}}
\multiput(616.13,381.17)(-2.132,-2.000){2}{\rule{0.450pt}{0.400pt}}
\multiput(611.92,378.95)(-0.462,-0.447){3}{\rule{0.500pt}{0.108pt}}
\multiput(612.96,379.17)(-1.962,-3.000){2}{\rule{0.250pt}{0.400pt}}
\put(608,375.17){\rule{0.700pt}{0.400pt}}
\multiput(609.55,376.17)(-1.547,-2.000){2}{\rule{0.350pt}{0.400pt}}
\multiput(605.37,373.95)(-0.685,-0.447){3}{\rule{0.633pt}{0.108pt}}
\multiput(606.69,374.17)(-2.685,-3.000){2}{\rule{0.317pt}{0.400pt}}
\put(601,370.17){\rule{0.700pt}{0.400pt}}
\multiput(602.55,371.17)(-1.547,-2.000){2}{\rule{0.350pt}{0.400pt}}
\multiput(598.37,368.95)(-0.685,-0.447){3}{\rule{0.633pt}{0.108pt}}
\multiput(599.69,369.17)(-2.685,-3.000){2}{\rule{0.317pt}{0.400pt}}
\put(594,365.17){\rule{0.700pt}{0.400pt}}
\multiput(595.55,366.17)(-1.547,-2.000){2}{\rule{0.350pt}{0.400pt}}
\multiput(591.92,363.95)(-0.462,-0.447){3}{\rule{0.500pt}{0.108pt}}
\multiput(592.96,364.17)(-1.962,-3.000){2}{\rule{0.250pt}{0.400pt}}
\put(587,360.17){\rule{0.900pt}{0.400pt}}
\multiput(589.13,361.17)(-2.132,-2.000){2}{\rule{0.450pt}{0.400pt}}
\multiput(584.92,358.95)(-0.462,-0.447){3}{\rule{0.500pt}{0.108pt}}
\multiput(585.96,359.17)(-1.962,-3.000){2}{\rule{0.250pt}{0.400pt}}
\put(1032,620.17){\rule{0.700pt}{0.400pt}}
\multiput(1033.55,621.17)(-1.547,-2.000){2}{\rule{0.350pt}{0.400pt}}
\multiput(1029.37,618.95)(-0.685,-0.447){3}{\rule{0.633pt}{0.108pt}}
\multiput(1030.69,619.17)(-2.685,-3.000){2}{\rule{0.317pt}{0.400pt}}
\put(1025,615.17){\rule{0.700pt}{0.400pt}}
\multiput(1026.55,616.17)(-1.547,-2.000){2}{\rule{0.350pt}{0.400pt}}
\multiput(1022.92,613.95)(-0.462,-0.447){3}{\rule{0.500pt}{0.108pt}}
\multiput(1023.96,614.17)(-1.962,-3.000){2}{\rule{0.250pt}{0.400pt}}
\put(1018,610.17){\rule{0.900pt}{0.400pt}}
\multiput(1020.13,611.17)(-2.132,-2.000){2}{\rule{0.450pt}{0.400pt}}
\multiput(1015.92,608.95)(-0.462,-0.447){3}{\rule{0.500pt}{0.108pt}}
\multiput(1016.96,609.17)(-1.962,-3.000){2}{\rule{0.250pt}{0.400pt}}
\put(1011,605.17){\rule{0.900pt}{0.400pt}}
\multiput(1013.13,606.17)(-2.132,-2.000){2}{\rule{0.450pt}{0.400pt}}
\put(1008,603.17){\rule{0.700pt}{0.400pt}}
\multiput(1009.55,604.17)(-1.547,-2.000){2}{\rule{0.350pt}{0.400pt}}
\multiput(1005.92,601.95)(-0.462,-0.447){3}{\rule{0.500pt}{0.108pt}}
\multiput(1006.96,602.17)(-1.962,-3.000){2}{\rule{0.250pt}{0.400pt}}
\put(1001,598.17){\rule{0.900pt}{0.400pt}}
\multiput(1003.13,599.17)(-2.132,-2.000){2}{\rule{0.450pt}{0.400pt}}
\multiput(998.92,596.95)(-0.462,-0.447){3}{\rule{0.500pt}{0.108pt}}
\multiput(999.96,597.17)(-1.962,-3.000){2}{\rule{0.250pt}{0.400pt}}
\put(995,593.17){\rule{0.700pt}{0.400pt}}
\multiput(996.55,594.17)(-1.547,-2.000){2}{\rule{0.350pt}{0.400pt}}
\multiput(992.37,591.95)(-0.685,-0.447){3}{\rule{0.633pt}{0.108pt}}
\multiput(993.69,592.17)(-2.685,-3.000){2}{\rule{0.317pt}{0.400pt}}
\put(988,588.17){\rule{0.700pt}{0.400pt}}
\multiput(989.55,589.17)(-1.547,-2.000){2}{\rule{0.350pt}{0.400pt}}
\multiput(985.92,586.95)(-0.462,-0.447){3}{\rule{0.500pt}{0.108pt}}
\multiput(986.96,587.17)(-1.962,-3.000){2}{\rule{0.250pt}{0.400pt}}
\put(981,583.17){\rule{0.900pt}{0.400pt}}
\multiput(983.13,584.17)(-2.132,-2.000){2}{\rule{0.450pt}{0.400pt}}
\multiput(978.92,581.95)(-0.462,-0.447){3}{\rule{0.500pt}{0.108pt}}
\multiput(979.96,582.17)(-1.962,-3.000){2}{\rule{0.250pt}{0.400pt}}
\put(974,578.17){\rule{0.900pt}{0.400pt}}
\multiput(976.13,579.17)(-2.132,-2.000){2}{\rule{0.450pt}{0.400pt}}
\multiput(971.92,576.95)(-0.462,-0.447){3}{\rule{0.500pt}{0.108pt}}
\multiput(972.96,577.17)(-1.962,-3.000){2}{\rule{0.250pt}{0.400pt}}
\put(968,573.17){\rule{0.700pt}{0.400pt}}
\multiput(969.55,574.17)(-1.547,-2.000){2}{\rule{0.350pt}{0.400pt}}
\multiput(965.37,571.95)(-0.685,-0.447){3}{\rule{0.633pt}{0.108pt}}
\multiput(966.69,572.17)(-2.685,-3.000){2}{\rule{0.317pt}{0.400pt}}
\put(961,568.17){\rule{0.700pt}{0.400pt}}
\multiput(962.55,569.17)(-1.547,-2.000){2}{\rule{0.350pt}{0.400pt}}
\multiput(958.92,566.95)(-0.462,-0.447){3}{\rule{0.500pt}{0.108pt}}
\multiput(959.96,567.17)(-1.962,-3.000){2}{\rule{0.250pt}{0.400pt}}
\put(954,563.17){\rule{0.900pt}{0.400pt}}
\multiput(956.13,564.17)(-2.132,-2.000){2}{\rule{0.450pt}{0.400pt}}
\put(951,561.17){\rule{0.700pt}{0.400pt}}
\multiput(952.55,562.17)(-1.547,-2.000){2}{\rule{0.350pt}{0.400pt}}
\multiput(948.92,559.95)(-0.462,-0.447){3}{\rule{0.500pt}{0.108pt}}
\multiput(949.96,560.17)(-1.962,-3.000){2}{\rule{0.250pt}{0.400pt}}
\put(944,556.17){\rule{0.900pt}{0.400pt}}
\multiput(946.13,557.17)(-2.132,-2.000){2}{\rule{0.450pt}{0.400pt}}
\multiput(941.92,554.95)(-0.462,-0.447){3}{\rule{0.500pt}{0.108pt}}
\multiput(942.96,555.17)(-1.962,-3.000){2}{\rule{0.250pt}{0.400pt}}
\put(937,551.17){\rule{0.900pt}{0.400pt}}
\multiput(939.13,552.17)(-2.132,-2.000){2}{\rule{0.450pt}{0.400pt}}
\multiput(934.92,549.95)(-0.462,-0.447){3}{\rule{0.500pt}{0.108pt}}
\multiput(935.96,550.17)(-1.962,-3.000){2}{\rule{0.250pt}{0.400pt}}
\put(931,546.17){\rule{0.700pt}{0.400pt}}
\multiput(932.55,547.17)(-1.547,-2.000){2}{\rule{0.350pt}{0.400pt}}
\multiput(928.37,544.95)(-0.685,-0.447){3}{\rule{0.633pt}{0.108pt}}
\multiput(929.69,545.17)(-2.685,-3.000){2}{\rule{0.317pt}{0.400pt}}
\put(924,541.17){\rule{0.700pt}{0.400pt}}
\multiput(925.55,542.17)(-1.547,-2.000){2}{\rule{0.350pt}{0.400pt}}
\multiput(921.92,539.95)(-0.462,-0.447){3}{\rule{0.500pt}{0.108pt}}
\multiput(922.96,540.17)(-1.962,-3.000){2}{\rule{0.250pt}{0.400pt}}
\put(917,536.17){\rule{0.900pt}{0.400pt}}
\multiput(919.13,537.17)(-2.132,-2.000){2}{\rule{0.450pt}{0.400pt}}
\multiput(914.92,534.95)(-0.462,-0.447){3}{\rule{0.500pt}{0.108pt}}
\multiput(915.96,535.17)(-1.962,-3.000){2}{\rule{0.250pt}{0.400pt}}
\put(911,531.17){\rule{0.700pt}{0.400pt}}
\multiput(912.55,532.17)(-1.547,-2.000){2}{\rule{0.350pt}{0.400pt}}
\multiput(908.37,529.95)(-0.685,-0.447){3}{\rule{0.633pt}{0.108pt}}
\multiput(909.69,530.17)(-2.685,-3.000){2}{\rule{0.317pt}{0.400pt}}
\put(904,526.17){\rule{0.700pt}{0.400pt}}
\multiput(905.55,527.17)(-1.547,-2.000){2}{\rule{0.350pt}{0.400pt}}
\put(900,524.17){\rule{0.900pt}{0.400pt}}
\multiput(902.13,525.17)(-2.132,-2.000){2}{\rule{0.450pt}{0.400pt}}
\multiput(897.92,522.95)(-0.462,-0.447){3}{\rule{0.500pt}{0.108pt}}
\multiput(898.96,523.17)(-1.962,-3.000){2}{\rule{0.250pt}{0.400pt}}
\put(894,519.17){\rule{0.700pt}{0.400pt}}
\multiput(895.55,520.17)(-1.547,-2.000){2}{\rule{0.350pt}{0.400pt}}
\multiput(891.37,517.95)(-0.685,-0.447){3}{\rule{0.633pt}{0.108pt}}
\multiput(892.69,518.17)(-2.685,-3.000){2}{\rule{0.317pt}{0.400pt}}
\put(887,514.17){\rule{0.700pt}{0.400pt}}
\multiput(888.55,515.17)(-1.547,-2.000){2}{\rule{0.350pt}{0.400pt}}
\multiput(884.92,512.95)(-0.462,-0.447){3}{\rule{0.500pt}{0.108pt}}
\multiput(885.96,513.17)(-1.962,-3.000){2}{\rule{0.250pt}{0.400pt}}
\put(880,509.17){\rule{0.900pt}{0.400pt}}
\multiput(882.13,510.17)(-2.132,-2.000){2}{\rule{0.450pt}{0.400pt}}
\multiput(877.92,507.95)(-0.462,-0.447){3}{\rule{0.500pt}{0.108pt}}
\multiput(878.96,508.17)(-1.962,-3.000){2}{\rule{0.250pt}{0.400pt}}
\put(874,504.17){\rule{0.700pt}{0.400pt}}
\multiput(875.55,505.17)(-1.547,-2.000){2}{\rule{0.350pt}{0.400pt}}
\multiput(871.37,502.95)(-0.685,-0.447){3}{\rule{0.633pt}{0.108pt}}
\multiput(872.69,503.17)(-2.685,-3.000){2}{\rule{0.317pt}{0.400pt}}
\put(867,499.17){\rule{0.700pt}{0.400pt}}
\multiput(868.55,500.17)(-1.547,-2.000){2}{\rule{0.350pt}{0.400pt}}
\multiput(864.37,497.95)(-0.685,-0.447){3}{\rule{0.633pt}{0.108pt}}
\multiput(865.69,498.17)(-2.685,-3.000){2}{\rule{0.317pt}{0.400pt}}
\put(860,494.17){\rule{0.700pt}{0.400pt}}
\multiput(861.55,495.17)(-1.547,-2.000){2}{\rule{0.350pt}{0.400pt}}
\multiput(857.92,492.95)(-0.462,-0.447){3}{\rule{0.500pt}{0.108pt}}
\multiput(858.96,493.17)(-1.962,-3.000){2}{\rule{0.250pt}{0.400pt}}
\put(853,489.17){\rule{0.900pt}{0.400pt}}
\multiput(855.13,490.17)(-2.132,-2.000){2}{\rule{0.450pt}{0.400pt}}
\put(850,487.17){\rule{0.700pt}{0.400pt}}
\multiput(851.55,488.17)(-1.547,-2.000){2}{\rule{0.350pt}{0.400pt}}
\multiput(847.92,485.95)(-0.462,-0.447){3}{\rule{0.500pt}{0.108pt}}
\multiput(848.96,486.17)(-1.962,-3.000){2}{\rule{0.250pt}{0.400pt}}
\put(843,482.17){\rule{0.900pt}{0.400pt}}
\multiput(845.13,483.17)(-2.132,-2.000){2}{\rule{0.450pt}{0.400pt}}
\multiput(840.92,480.95)(-0.462,-0.447){3}{\rule{0.500pt}{0.108pt}}
\multiput(841.96,481.17)(-1.962,-3.000){2}{\rule{0.250pt}{0.400pt}}
\put(837,477.17){\rule{0.700pt}{0.400pt}}
\multiput(838.55,478.17)(-1.547,-2.000){2}{\rule{0.350pt}{0.400pt}}
\multiput(834.37,475.95)(-0.685,-0.447){3}{\rule{0.633pt}{0.108pt}}
\multiput(835.69,476.17)(-2.685,-3.000){2}{\rule{0.317pt}{0.400pt}}
\put(830,472.17){\rule{0.700pt}{0.400pt}}
\multiput(831.55,473.17)(-1.547,-2.000){2}{\rule{0.350pt}{0.400pt}}
\put(826,470.17){\rule{0.900pt}{0.400pt}}
\multiput(828.13,471.17)(-2.132,-2.000){2}{\rule{0.450pt}{0.400pt}}
\put(823,468.17){\rule{0.700pt}{0.400pt}}
\multiput(824.55,469.17)(-1.547,-2.000){2}{\rule{0.350pt}{0.400pt}}
\multiput(820.92,466.95)(-0.462,-0.447){3}{\rule{0.500pt}{0.108pt}}
\multiput(821.96,467.17)(-1.962,-3.000){2}{\rule{0.250pt}{0.400pt}}
\put(816,463.17){\rule{0.900pt}{0.400pt}}
\multiput(818.13,464.17)(-2.132,-2.000){2}{\rule{0.450pt}{0.400pt}}
\multiput(813.92,461.95)(-0.462,-0.447){3}{\rule{0.500pt}{0.108pt}}
\multiput(814.96,462.17)(-1.962,-3.000){2}{\rule{0.250pt}{0.400pt}}
\put(810,458.17){\rule{0.700pt}{0.400pt}}
\multiput(811.55,459.17)(-1.547,-2.000){2}{\rule{0.350pt}{0.400pt}}
\multiput(807.37,456.95)(-0.685,-0.447){3}{\rule{0.633pt}{0.108pt}}
\multiput(808.69,457.17)(-2.685,-3.000){2}{\rule{0.317pt}{0.400pt}}
\put(803,453.17){\rule{0.700pt}{0.400pt}}
\multiput(804.55,454.17)(-1.547,-2.000){2}{\rule{0.350pt}{0.400pt}}
\put(800,451.17){\rule{0.700pt}{0.400pt}}
\multiput(801.55,452.17)(-1.547,-2.000){2}{\rule{0.350pt}{0.400pt}}
\multiput(797.37,449.95)(-0.685,-0.447){3}{\rule{0.633pt}{0.108pt}}
\multiput(798.69,450.17)(-2.685,-3.000){2}{\rule{0.317pt}{0.400pt}}
\put(793,446.17){\rule{0.700pt}{0.400pt}}
\multiput(794.55,447.17)(-1.547,-2.000){2}{\rule{0.350pt}{0.400pt}}
\multiput(790.37,444.95)(-0.685,-0.447){3}{\rule{0.633pt}{0.108pt}}
\multiput(791.69,445.17)(-2.685,-3.000){2}{\rule{0.317pt}{0.400pt}}
\put(786,441.17){\rule{0.700pt}{0.400pt}}
\multiput(787.55,442.17)(-1.547,-2.000){2}{\rule{0.350pt}{0.400pt}}
\multiput(783.92,439.95)(-0.462,-0.447){3}{\rule{0.500pt}{0.108pt}}
\multiput(784.96,440.17)(-1.962,-3.000){2}{\rule{0.250pt}{0.400pt}}
\put(779,436.17){\rule{0.900pt}{0.400pt}}
\multiput(781.13,437.17)(-2.132,-2.000){2}{\rule{0.450pt}{0.400pt}}
\multiput(776.92,434.95)(-0.462,-0.447){3}{\rule{0.500pt}{0.108pt}}
\multiput(777.96,435.17)(-1.962,-3.000){2}{\rule{0.250pt}{0.400pt}}
\put(773,431.17){\rule{0.700pt}{0.400pt}}
\multiput(774.55,432.17)(-1.547,-2.000){2}{\rule{0.350pt}{0.400pt}}
\multiput(770.37,429.95)(-0.685,-0.447){3}{\rule{0.633pt}{0.108pt}}
\multiput(771.69,430.17)(-2.685,-3.000){2}{\rule{0.317pt}{0.400pt}}
\put(766,426.17){\rule{0.700pt}{0.400pt}}
\multiput(767.55,427.17)(-1.547,-2.000){2}{\rule{0.350pt}{0.400pt}}
\multiput(763.92,424.95)(-0.462,-0.447){3}{\rule{0.500pt}{0.108pt}}
\multiput(764.96,425.17)(-1.962,-3.000){2}{\rule{0.250pt}{0.400pt}}
\put(759,421.17){\rule{0.900pt}{0.400pt}}
\multiput(761.13,422.17)(-2.132,-2.000){2}{\rule{0.450pt}{0.400pt}}
\multiput(756.92,419.95)(-0.462,-0.447){3}{\rule{0.500pt}{0.108pt}}
\multiput(757.96,420.17)(-1.962,-3.000){2}{\rule{0.250pt}{0.400pt}}
\put(752,416.17){\rule{0.900pt}{0.400pt}}
\multiput(754.13,417.17)(-2.132,-2.000){2}{\rule{0.450pt}{0.400pt}}
\put(750.17,413){\rule{0.400pt}{0.700pt}}
\multiput(751.17,414.55)(-2.000,-1.547){2}{\rule{0.400pt}{0.350pt}}
\put(747,411.17){\rule{0.700pt}{0.400pt}}
\multiput(748.55,412.17)(-1.547,-2.000){2}{\rule{0.350pt}{0.400pt}}
\put(743,409.17){\rule{0.900pt}{0.400pt}}
\multiput(745.13,410.17)(-2.132,-2.000){2}{\rule{0.450pt}{0.400pt}}
\multiput(740.92,407.95)(-0.462,-0.447){3}{\rule{0.500pt}{0.108pt}}
\multiput(741.96,408.17)(-1.962,-3.000){2}{\rule{0.250pt}{0.400pt}}
\put(737,404.17){\rule{0.700pt}{0.400pt}}
\multiput(738.55,405.17)(-1.547,-2.000){2}{\rule{0.350pt}{0.400pt}}
\multiput(734.37,402.95)(-0.685,-0.447){3}{\rule{0.633pt}{0.108pt}}
\multiput(735.69,403.17)(-2.685,-3.000){2}{\rule{0.317pt}{0.400pt}}
\put(730,399.17){\rule{0.700pt}{0.400pt}}
\multiput(731.55,400.17)(-1.547,-2.000){2}{\rule{0.350pt}{0.400pt}}
\multiput(727.92,397.95)(-0.462,-0.447){3}{\rule{0.500pt}{0.108pt}}
\multiput(728.96,398.17)(-1.962,-3.000){2}{\rule{0.250pt}{0.400pt}}
\put(723,394.17){\rule{0.900pt}{0.400pt}}
\multiput(725.13,395.17)(-2.132,-2.000){2}{\rule{0.450pt}{0.400pt}}
\multiput(720.92,392.95)(-0.462,-0.447){3}{\rule{0.500pt}{0.108pt}}
\multiput(721.96,393.17)(-1.962,-3.000){2}{\rule{0.250pt}{0.400pt}}
\put(716,389.17){\rule{0.900pt}{0.400pt}}
\multiput(718.13,390.17)(-2.132,-2.000){2}{\rule{0.450pt}{0.400pt}}
\multiput(713.92,387.95)(-0.462,-0.447){3}{\rule{0.500pt}{0.108pt}}
\multiput(714.96,388.17)(-1.962,-3.000){2}{\rule{0.250pt}{0.400pt}}
\put(710,384.17){\rule{0.700pt}{0.400pt}}
\multiput(711.55,385.17)(-1.547,-2.000){2}{\rule{0.350pt}{0.400pt}}
\multiput(707.37,382.95)(-0.685,-0.447){3}{\rule{0.633pt}{0.108pt}}
\multiput(708.69,383.17)(-2.685,-3.000){2}{\rule{0.317pt}{0.400pt}}
\put(703,379.17){\rule{0.700pt}{0.400pt}}
\multiput(704.55,380.17)(-1.547,-2.000){2}{\rule{0.350pt}{0.400pt}}
\multiput(1151.92,642.95)(-0.462,-0.447){3}{\rule{0.500pt}{0.108pt}}
\multiput(1152.96,643.17)(-1.962,-3.000){2}{\rule{0.250pt}{0.400pt}}
\put(1147,639.17){\rule{0.900pt}{0.400pt}}
\multiput(1149.13,640.17)(-2.132,-2.000){2}{\rule{0.450pt}{0.400pt}}
\multiput(1144.92,637.95)(-0.462,-0.447){3}{\rule{0.500pt}{0.108pt}}
\multiput(1145.96,638.17)(-1.962,-3.000){2}{\rule{0.250pt}{0.400pt}}
\put(1141,634.17){\rule{0.700pt}{0.400pt}}
\multiput(1142.55,635.17)(-1.547,-2.000){2}{\rule{0.350pt}{0.400pt}}
\multiput(1138.37,632.95)(-0.685,-0.447){3}{\rule{0.633pt}{0.108pt}}
\multiput(1139.69,633.17)(-2.685,-3.000){2}{\rule{0.317pt}{0.400pt}}
\put(1134,629.17){\rule{0.700pt}{0.400pt}}
\multiput(1135.55,630.17)(-1.547,-2.000){2}{\rule{0.350pt}{0.400pt}}
\put(1131,627.17){\rule{0.700pt}{0.400pt}}
\multiput(1132.55,628.17)(-1.547,-2.000){2}{\rule{0.350pt}{0.400pt}}
\multiput(1128.37,625.95)(-0.685,-0.447){3}{\rule{0.633pt}{0.108pt}}
\multiput(1129.69,626.17)(-2.685,-3.000){2}{\rule{0.317pt}{0.400pt}}
\put(1124,622.17){\rule{0.700pt}{0.400pt}}
\multiput(1125.55,623.17)(-1.547,-2.000){2}{\rule{0.350pt}{0.400pt}}
\multiput(1121.37,620.95)(-0.685,-0.447){3}{\rule{0.633pt}{0.108pt}}
\multiput(1122.69,621.17)(-2.685,-3.000){2}{\rule{0.317pt}{0.400pt}}
\put(1117,617.17){\rule{0.700pt}{0.400pt}}
\multiput(1118.55,618.17)(-1.547,-2.000){2}{\rule{0.350pt}{0.400pt}}
\multiput(1114.92,615.95)(-0.462,-0.447){3}{\rule{0.500pt}{0.108pt}}
\multiput(1115.96,616.17)(-1.962,-3.000){2}{\rule{0.250pt}{0.400pt}}
\put(1110,612.17){\rule{0.900pt}{0.400pt}}
\multiput(1112.13,613.17)(-2.132,-2.000){2}{\rule{0.450pt}{0.400pt}}
\multiput(1107.92,610.95)(-0.462,-0.447){3}{\rule{0.500pt}{0.108pt}}
\multiput(1108.96,611.17)(-1.962,-3.000){2}{\rule{0.250pt}{0.400pt}}
\put(1104,607.17){\rule{0.700pt}{0.400pt}}
\multiput(1105.55,608.17)(-1.547,-2.000){2}{\rule{0.350pt}{0.400pt}}
\multiput(1101.37,605.95)(-0.685,-0.447){3}{\rule{0.633pt}{0.108pt}}
\multiput(1102.69,606.17)(-2.685,-3.000){2}{\rule{0.317pt}{0.400pt}}
\put(1097,602.17){\rule{0.700pt}{0.400pt}}
\multiput(1098.55,603.17)(-1.547,-2.000){2}{\rule{0.350pt}{0.400pt}}
\multiput(1094.92,600.95)(-0.462,-0.447){3}{\rule{0.500pt}{0.108pt}}
\multiput(1095.96,601.17)(-1.962,-3.000){2}{\rule{0.250pt}{0.400pt}}
\put(1090,597.17){\rule{0.900pt}{0.400pt}}
\multiput(1092.13,598.17)(-2.132,-2.000){2}{\rule{0.450pt}{0.400pt}}
\multiput(1087.92,595.95)(-0.462,-0.447){3}{\rule{0.500pt}{0.108pt}}
\multiput(1088.96,596.17)(-1.962,-3.000){2}{\rule{0.250pt}{0.400pt}}
\put(1083,592.17){\rule{0.900pt}{0.400pt}}
\multiput(1085.13,593.17)(-2.132,-2.000){2}{\rule{0.450pt}{0.400pt}}
\multiput(1080.92,590.95)(-0.462,-0.447){3}{\rule{0.500pt}{0.108pt}}
\multiput(1081.96,591.17)(-1.962,-3.000){2}{\rule{0.250pt}{0.400pt}}
\put(1077,587.17){\rule{0.700pt}{0.400pt}}
\multiput(1078.55,588.17)(-1.547,-2.000){2}{\rule{0.350pt}{0.400pt}}
\put(1073,585.17){\rule{0.900pt}{0.400pt}}
\multiput(1075.13,586.17)(-2.132,-2.000){2}{\rule{0.450pt}{0.400pt}}
\multiput(1070.92,583.95)(-0.462,-0.447){3}{\rule{0.500pt}{0.108pt}}
\multiput(1071.96,584.17)(-1.962,-3.000){2}{\rule{0.250pt}{0.400pt}}
\put(1067,580.17){\rule{0.700pt}{0.400pt}}
\multiput(1068.55,581.17)(-1.547,-2.000){2}{\rule{0.350pt}{0.400pt}}
\multiput(1064.37,578.95)(-0.685,-0.447){3}{\rule{0.633pt}{0.108pt}}
\multiput(1065.69,579.17)(-2.685,-3.000){2}{\rule{0.317pt}{0.400pt}}
\put(1060,575.17){\rule{0.700pt}{0.400pt}}
\multiput(1061.55,576.17)(-1.547,-2.000){2}{\rule{0.350pt}{0.400pt}}
\multiput(1057.92,573.95)(-0.462,-0.447){3}{\rule{0.500pt}{0.108pt}}
\multiput(1058.96,574.17)(-1.962,-3.000){2}{\rule{0.250pt}{0.400pt}}
\put(1053,570.17){\rule{0.900pt}{0.400pt}}
\multiput(1055.13,571.17)(-2.132,-2.000){2}{\rule{0.450pt}{0.400pt}}
\multiput(1050.92,568.95)(-0.462,-0.447){3}{\rule{0.500pt}{0.108pt}}
\multiput(1051.96,569.17)(-1.962,-3.000){2}{\rule{0.250pt}{0.400pt}}
\put(1046,565.17){\rule{0.900pt}{0.400pt}}
\multiput(1048.13,566.17)(-2.132,-2.000){2}{\rule{0.450pt}{0.400pt}}
\multiput(1043.92,563.95)(-0.462,-0.447){3}{\rule{0.500pt}{0.108pt}}
\multiput(1044.96,564.17)(-1.962,-3.000){2}{\rule{0.250pt}{0.400pt}}
\put(1040,560.17){\rule{0.700pt}{0.400pt}}
\multiput(1041.55,561.17)(-1.547,-2.000){2}{\rule{0.350pt}{0.400pt}}
\multiput(1037.37,558.95)(-0.685,-0.447){3}{\rule{0.633pt}{0.108pt}}
\multiput(1038.69,559.17)(-2.685,-3.000){2}{\rule{0.317pt}{0.400pt}}
\put(1033,555.17){\rule{0.700pt}{0.400pt}}
\multiput(1034.55,556.17)(-1.547,-2.000){2}{\rule{0.350pt}{0.400pt}}
\multiput(1030.92,553.95)(-0.462,-0.447){3}{\rule{0.500pt}{0.108pt}}
\multiput(1031.96,554.17)(-1.962,-3.000){2}{\rule{0.250pt}{0.400pt}}
\put(1026,550.17){\rule{0.900pt}{0.400pt}}
\multiput(1028.13,551.17)(-2.132,-2.000){2}{\rule{0.450pt}{0.400pt}}
\put(1023,548.17){\rule{0.700pt}{0.400pt}}
\multiput(1024.55,549.17)(-1.547,-2.000){2}{\rule{0.350pt}{0.400pt}}
\multiput(1020.92,546.95)(-0.462,-0.447){3}{\rule{0.500pt}{0.108pt}}
\multiput(1021.96,547.17)(-1.962,-3.000){2}{\rule{0.250pt}{0.400pt}}
\put(1016,543.17){\rule{0.900pt}{0.400pt}}
\multiput(1018.13,544.17)(-2.132,-2.000){2}{\rule{0.450pt}{0.400pt}}
\multiput(1013.92,541.95)(-0.462,-0.447){3}{\rule{0.500pt}{0.108pt}}
\multiput(1014.96,542.17)(-1.962,-3.000){2}{\rule{0.250pt}{0.400pt}}
\put(1009,538.17){\rule{0.900pt}{0.400pt}}
\multiput(1011.13,539.17)(-2.132,-2.000){2}{\rule{0.450pt}{0.400pt}}
\multiput(1006.92,536.95)(-0.462,-0.447){3}{\rule{0.500pt}{0.108pt}}
\multiput(1007.96,537.17)(-1.962,-3.000){2}{\rule{0.250pt}{0.400pt}}
\put(1003,533.17){\rule{0.700pt}{0.400pt}}
\multiput(1004.55,534.17)(-1.547,-2.000){2}{\rule{0.350pt}{0.400pt}}
\multiput(1000.37,531.95)(-0.685,-0.447){3}{\rule{0.633pt}{0.108pt}}
\multiput(1001.69,532.17)(-2.685,-3.000){2}{\rule{0.317pt}{0.400pt}}
\put(996,528.17){\rule{0.700pt}{0.400pt}}
\multiput(997.55,529.17)(-1.547,-2.000){2}{\rule{0.350pt}{0.400pt}}
\multiput(993.92,526.95)(-0.462,-0.447){3}{\rule{0.500pt}{0.108pt}}
\multiput(994.96,527.17)(-1.962,-3.000){2}{\rule{0.250pt}{0.400pt}}
\put(989,523.17){\rule{0.900pt}{0.400pt}}
\multiput(991.13,524.17)(-2.132,-2.000){2}{\rule{0.450pt}{0.400pt}}
\multiput(986.92,521.95)(-0.462,-0.447){3}{\rule{0.500pt}{0.108pt}}
\multiput(987.96,522.17)(-1.962,-3.000){2}{\rule{0.250pt}{0.400pt}}
\put(983,518.17){\rule{0.700pt}{0.400pt}}
\multiput(984.55,519.17)(-1.547,-2.000){2}{\rule{0.350pt}{0.400pt}}
\multiput(980.37,516.95)(-0.685,-0.447){3}{\rule{0.633pt}{0.108pt}}
\multiput(981.69,517.17)(-2.685,-3.000){2}{\rule{0.317pt}{0.400pt}}
\put(976,513.17){\rule{0.700pt}{0.400pt}}
\multiput(977.55,514.17)(-1.547,-2.000){2}{\rule{0.350pt}{0.400pt}}
\put(972,511.17){\rule{0.900pt}{0.400pt}}
\multiput(974.13,512.17)(-2.132,-2.000){2}{\rule{0.450pt}{0.400pt}}
\multiput(969.92,509.95)(-0.462,-0.447){3}{\rule{0.500pt}{0.108pt}}
\multiput(970.96,510.17)(-1.962,-3.000){2}{\rule{0.250pt}{0.400pt}}
\put(966,506.17){\rule{0.700pt}{0.400pt}}
\multiput(967.55,507.17)(-1.547,-2.000){2}{\rule{0.350pt}{0.400pt}}
\multiput(963.37,504.95)(-0.685,-0.447){3}{\rule{0.633pt}{0.108pt}}
\multiput(964.69,505.17)(-2.685,-3.000){2}{\rule{0.317pt}{0.400pt}}
\put(959,501.17){\rule{0.700pt}{0.400pt}}
\multiput(960.55,502.17)(-1.547,-2.000){2}{\rule{0.350pt}{0.400pt}}
\multiput(956.92,499.95)(-0.462,-0.447){3}{\rule{0.500pt}{0.108pt}}
\multiput(957.96,500.17)(-1.962,-3.000){2}{\rule{0.250pt}{0.400pt}}
\put(952,496.17){\rule{0.900pt}{0.400pt}}
\multiput(954.13,497.17)(-2.132,-2.000){2}{\rule{0.450pt}{0.400pt}}
\multiput(949.92,494.95)(-0.462,-0.447){3}{\rule{0.500pt}{0.108pt}}
\multiput(950.96,495.17)(-1.962,-3.000){2}{\rule{0.250pt}{0.400pt}}
\put(946,491.17){\rule{0.700pt}{0.400pt}}
\multiput(947.55,492.17)(-1.547,-2.000){2}{\rule{0.350pt}{0.400pt}}
\multiput(943.37,489.95)(-0.685,-0.447){3}{\rule{0.633pt}{0.108pt}}
\multiput(944.69,490.17)(-2.685,-3.000){2}{\rule{0.317pt}{0.400pt}}
\put(939,486.17){\rule{0.700pt}{0.400pt}}
\multiput(940.55,487.17)(-1.547,-2.000){2}{\rule{0.350pt}{0.400pt}}
\multiput(936.37,484.95)(-0.685,-0.447){3}{\rule{0.633pt}{0.108pt}}
\multiput(937.69,485.17)(-2.685,-3.000){2}{\rule{0.317pt}{0.400pt}}
\put(932,481.17){\rule{0.700pt}{0.400pt}}
\multiput(933.55,482.17)(-1.547,-2.000){2}{\rule{0.350pt}{0.400pt}}
\multiput(929.92,479.95)(-0.462,-0.447){3}{\rule{0.500pt}{0.108pt}}
\multiput(930.96,480.17)(-1.962,-3.000){2}{\rule{0.250pt}{0.400pt}}
\put(925,476.17){\rule{0.900pt}{0.400pt}}
\multiput(927.13,477.17)(-2.132,-2.000){2}{\rule{0.450pt}{0.400pt}}
\put(922,474.17){\rule{0.700pt}{0.400pt}}
\multiput(923.55,475.17)(-1.547,-2.000){2}{\rule{0.350pt}{0.400pt}}
\multiput(919.92,472.95)(-0.462,-0.447){3}{\rule{0.500pt}{0.108pt}}
\multiput(920.96,473.17)(-1.962,-3.000){2}{\rule{0.250pt}{0.400pt}}
\put(915,469.67){\rule{0.964pt}{0.400pt}}
\multiput(917.00,470.17)(-2.000,-1.000){2}{\rule{0.482pt}{0.400pt}}
\multiput(912.92,468.95)(-0.462,-0.447){3}{\rule{0.500pt}{0.108pt}}
\multiput(913.96,469.17)(-1.962,-3.000){2}{\rule{0.250pt}{0.400pt}}
\put(909,465.17){\rule{0.700pt}{0.400pt}}
\multiput(910.55,466.17)(-1.547,-2.000){2}{\rule{0.350pt}{0.400pt}}
\multiput(906.37,463.95)(-0.685,-0.447){3}{\rule{0.633pt}{0.108pt}}
\multiput(907.69,464.17)(-2.685,-3.000){2}{\rule{0.317pt}{0.400pt}}
\put(902,460.17){\rule{0.700pt}{0.400pt}}
\multiput(903.55,461.17)(-1.547,-2.000){2}{\rule{0.350pt}{0.400pt}}
\multiput(899.37,458.95)(-0.685,-0.447){3}{\rule{0.633pt}{0.108pt}}
\multiput(900.69,459.17)(-2.685,-3.000){2}{\rule{0.317pt}{0.400pt}}
\put(895,455.17){\rule{0.700pt}{0.400pt}}
\multiput(896.55,456.17)(-1.547,-2.000){2}{\rule{0.350pt}{0.400pt}}
\multiput(892.92,453.95)(-0.462,-0.447){3}{\rule{0.500pt}{0.108pt}}
\multiput(893.96,454.17)(-1.962,-3.000){2}{\rule{0.250pt}{0.400pt}}
\put(888,450.17){\rule{0.900pt}{0.400pt}}
\multiput(890.13,451.17)(-2.132,-2.000){2}{\rule{0.450pt}{0.400pt}}
\multiput(885.92,448.95)(-0.462,-0.447){3}{\rule{0.500pt}{0.108pt}}
\multiput(886.96,449.17)(-1.962,-3.000){2}{\rule{0.250pt}{0.400pt}}
\put(882,445.17){\rule{0.700pt}{0.400pt}}
\multiput(883.55,446.17)(-1.547,-2.000){2}{\rule{0.350pt}{0.400pt}}
\multiput(879.37,443.95)(-0.685,-0.447){3}{\rule{0.633pt}{0.108pt}}
\multiput(880.69,444.17)(-2.685,-3.000){2}{\rule{0.317pt}{0.400pt}}
\put(875,440.17){\rule{0.700pt}{0.400pt}}
\multiput(876.55,441.17)(-1.547,-2.000){2}{\rule{0.350pt}{0.400pt}}
\multiput(872.92,438.95)(-0.462,-0.447){3}{\rule{0.500pt}{0.108pt}}
\multiput(873.96,439.17)(-1.962,-3.000){2}{\rule{0.250pt}{0.400pt}}
\put(868,435.17){\rule{0.900pt}{0.400pt}}
\multiput(870.13,436.17)(-2.132,-2.000){2}{\rule{0.450pt}{0.400pt}}
\put(865,433.17){\rule{0.700pt}{0.400pt}}
\multiput(866.55,434.17)(-1.547,-2.000){2}{\rule{0.350pt}{0.400pt}}
\multiput(862.37,431.95)(-0.685,-0.447){3}{\rule{0.633pt}{0.108pt}}
\multiput(863.69,432.17)(-2.685,-3.000){2}{\rule{0.317pt}{0.400pt}}
\put(858,428.17){\rule{0.700pt}{0.400pt}}
\multiput(859.55,429.17)(-1.547,-2.000){2}{\rule{0.350pt}{0.400pt}}
\multiput(855.92,426.95)(-0.462,-0.447){3}{\rule{0.500pt}{0.108pt}}
\multiput(856.96,427.17)(-1.962,-3.000){2}{\rule{0.250pt}{0.400pt}}
\put(851,423.17){\rule{0.900pt}{0.400pt}}
\multiput(853.13,424.17)(-2.132,-2.000){2}{\rule{0.450pt}{0.400pt}}
\multiput(848.92,421.95)(-0.462,-0.447){3}{\rule{0.500pt}{0.108pt}}
\multiput(849.96,422.17)(-1.962,-3.000){2}{\rule{0.250pt}{0.400pt}}
\put(845,418.17){\rule{0.700pt}{0.400pt}}
\multiput(846.55,419.17)(-1.547,-2.000){2}{\rule{0.350pt}{0.400pt}}
\multiput(842.37,416.95)(-0.685,-0.447){3}{\rule{0.633pt}{0.108pt}}
\multiput(843.69,417.17)(-2.685,-3.000){2}{\rule{0.317pt}{0.400pt}}
\put(838,413.17){\rule{0.700pt}{0.400pt}}
\multiput(839.55,414.17)(-1.547,-2.000){2}{\rule{0.350pt}{0.400pt}}
\multiput(835.92,411.95)(-0.462,-0.447){3}{\rule{0.500pt}{0.108pt}}
\multiput(836.96,412.17)(-1.962,-3.000){2}{\rule{0.250pt}{0.400pt}}
\put(831,408.17){\rule{0.900pt}{0.400pt}}
\multiput(833.13,409.17)(-2.132,-2.000){2}{\rule{0.450pt}{0.400pt}}
\multiput(828.92,406.95)(-0.462,-0.447){3}{\rule{0.500pt}{0.108pt}}
\multiput(829.96,407.17)(-1.962,-3.000){2}{\rule{0.250pt}{0.400pt}}
\put(824,403.17){\rule{0.900pt}{0.400pt}}
\multiput(826.13,404.17)(-2.132,-2.000){2}{\rule{0.450pt}{0.400pt}}
\multiput(821.92,401.95)(-0.462,-0.447){3}{\rule{0.500pt}{0.108pt}}
\multiput(822.96,402.17)(-1.962,-3.000){2}{\rule{0.250pt}{0.400pt}}
\put(1270,663.17){\rule{0.700pt}{0.400pt}}
\multiput(1271.55,664.17)(-1.547,-2.000){2}{\rule{0.350pt}{0.400pt}}
\multiput(1267.37,661.95)(-0.685,-0.447){3}{\rule{0.633pt}{0.108pt}}
\multiput(1268.69,662.17)(-2.685,-3.000){2}{\rule{0.317pt}{0.400pt}}
\put(1263,658.17){\rule{0.700pt}{0.400pt}}
\multiput(1264.55,659.17)(-1.547,-2.000){2}{\rule{0.350pt}{0.400pt}}
\multiput(1260.92,656.95)(-0.462,-0.447){3}{\rule{0.500pt}{0.108pt}}
\multiput(1261.96,657.17)(-1.962,-3.000){2}{\rule{0.250pt}{0.400pt}}
\put(1256,653.17){\rule{0.900pt}{0.400pt}}
\multiput(1258.13,654.17)(-2.132,-2.000){2}{\rule{0.450pt}{0.400pt}}
\put(1253,651.17){\rule{0.700pt}{0.400pt}}
\multiput(1254.55,652.17)(-1.547,-2.000){2}{\rule{0.350pt}{0.400pt}}
\multiput(1250.92,649.95)(-0.462,-0.447){3}{\rule{0.500pt}{0.108pt}}
\multiput(1251.96,650.17)(-1.962,-3.000){2}{\rule{0.250pt}{0.400pt}}
\put(1246,646.17){\rule{0.900pt}{0.400pt}}
\multiput(1248.13,647.17)(-2.132,-2.000){2}{\rule{0.450pt}{0.400pt}}
\multiput(1243.92,644.95)(-0.462,-0.447){3}{\rule{0.500pt}{0.108pt}}
\multiput(1244.96,645.17)(-1.962,-3.000){2}{\rule{0.250pt}{0.400pt}}
\put(1239,641.17){\rule{0.900pt}{0.400pt}}
\multiput(1241.13,642.17)(-2.132,-2.000){2}{\rule{0.450pt}{0.400pt}}
\multiput(1236.92,639.95)(-0.462,-0.447){3}{\rule{0.500pt}{0.108pt}}
\multiput(1237.96,640.17)(-1.962,-3.000){2}{\rule{0.250pt}{0.400pt}}
\put(1233,636.17){\rule{0.700pt}{0.400pt}}
\multiput(1234.55,637.17)(-1.547,-2.000){2}{\rule{0.350pt}{0.400pt}}
\multiput(1230.37,634.95)(-0.685,-0.447){3}{\rule{0.633pt}{0.108pt}}
\multiput(1231.69,635.17)(-2.685,-3.000){2}{\rule{0.317pt}{0.400pt}}
\put(1226,631.17){\rule{0.700pt}{0.400pt}}
\multiput(1227.55,632.17)(-1.547,-2.000){2}{\rule{0.350pt}{0.400pt}}
\multiput(1223.92,629.95)(-0.462,-0.447){3}{\rule{0.500pt}{0.108pt}}
\multiput(1224.96,630.17)(-1.962,-3.000){2}{\rule{0.250pt}{0.400pt}}
\put(1219,626.17){\rule{0.900pt}{0.400pt}}
\multiput(1221.13,627.17)(-2.132,-2.000){2}{\rule{0.450pt}{0.400pt}}
\multiput(1216.92,624.95)(-0.462,-0.447){3}{\rule{0.500pt}{0.108pt}}
\multiput(1217.96,625.17)(-1.962,-3.000){2}{\rule{0.250pt}{0.400pt}}
\put(1213,621.17){\rule{0.700pt}{0.400pt}}
\multiput(1214.55,622.17)(-1.547,-2.000){2}{\rule{0.350pt}{0.400pt}}
\multiput(1210.37,619.95)(-0.685,-0.447){3}{\rule{0.633pt}{0.108pt}}
\multiput(1211.69,620.17)(-2.685,-3.000){2}{\rule{0.317pt}{0.400pt}}
\put(1206,616.17){\rule{0.700pt}{0.400pt}}
\multiput(1207.55,617.17)(-1.547,-2.000){2}{\rule{0.350pt}{0.400pt}}
\multiput(1203.37,614.95)(-0.685,-0.447){3}{\rule{0.633pt}{0.108pt}}
\multiput(1204.69,615.17)(-2.685,-3.000){2}{\rule{0.317pt}{0.400pt}}
\put(1199,611.17){\rule{0.700pt}{0.400pt}}
\multiput(1200.55,612.17)(-1.547,-2.000){2}{\rule{0.350pt}{0.400pt}}
\put(1196,609.17){\rule{0.700pt}{0.400pt}}
\multiput(1197.55,610.17)(-1.547,-2.000){2}{\rule{0.350pt}{0.400pt}}
\multiput(1193.37,607.95)(-0.685,-0.447){3}{\rule{0.633pt}{0.108pt}}
\multiput(1194.69,608.17)(-2.685,-3.000){2}{\rule{0.317pt}{0.400pt}}
\put(1189,604.17){\rule{0.700pt}{0.400pt}}
\multiput(1190.55,605.17)(-1.547,-2.000){2}{\rule{0.350pt}{0.400pt}}
\multiput(1186.92,602.95)(-0.462,-0.447){3}{\rule{0.500pt}{0.108pt}}
\multiput(1187.96,603.17)(-1.962,-3.000){2}{\rule{0.250pt}{0.400pt}}
\put(1182,599.17){\rule{0.900pt}{0.400pt}}
\multiput(1184.13,600.17)(-2.132,-2.000){2}{\rule{0.450pt}{0.400pt}}
\multiput(1179.92,597.95)(-0.462,-0.447){3}{\rule{0.500pt}{0.108pt}}
\multiput(1180.96,598.17)(-1.962,-3.000){2}{\rule{0.250pt}{0.400pt}}
\put(1176,594.17){\rule{0.700pt}{0.400pt}}
\multiput(1177.55,595.17)(-1.547,-2.000){2}{\rule{0.350pt}{0.400pt}}
\multiput(1173.37,592.95)(-0.685,-0.447){3}{\rule{0.633pt}{0.108pt}}
\multiput(1174.69,593.17)(-2.685,-3.000){2}{\rule{0.317pt}{0.400pt}}
\put(1169,589.17){\rule{0.700pt}{0.400pt}}
\multiput(1170.55,590.17)(-1.547,-2.000){2}{\rule{0.350pt}{0.400pt}}
\multiput(1166.37,587.95)(-0.685,-0.447){3}{\rule{0.633pt}{0.108pt}}
\multiput(1167.69,588.17)(-2.685,-3.000){2}{\rule{0.317pt}{0.400pt}}
\put(1162,584.17){\rule{0.700pt}{0.400pt}}
\multiput(1163.55,585.17)(-1.547,-2.000){2}{\rule{0.350pt}{0.400pt}}
\multiput(1159.92,582.95)(-0.462,-0.447){3}{\rule{0.500pt}{0.108pt}}
\multiput(1160.96,583.17)(-1.962,-3.000){2}{\rule{0.250pt}{0.400pt}}
\put(1155,579.17){\rule{0.900pt}{0.400pt}}
\multiput(1157.13,580.17)(-2.132,-2.000){2}{\rule{0.450pt}{0.400pt}}
\multiput(1152.92,577.95)(-0.462,-0.447){3}{\rule{0.500pt}{0.108pt}}
\multiput(1153.96,578.17)(-1.962,-3.000){2}{\rule{0.250pt}{0.400pt}}
\put(1149,574.17){\rule{0.700pt}{0.400pt}}
\multiput(1150.55,575.17)(-1.547,-2.000){2}{\rule{0.350pt}{0.400pt}}
\put(1145,572.17){\rule{0.900pt}{0.400pt}}
\multiput(1147.13,573.17)(-2.132,-2.000){2}{\rule{0.450pt}{0.400pt}}
\multiput(1142.92,570.95)(-0.462,-0.447){3}{\rule{0.500pt}{0.108pt}}
\multiput(1143.96,571.17)(-1.962,-3.000){2}{\rule{0.250pt}{0.400pt}}
\put(1139,567.17){\rule{0.700pt}{0.400pt}}
\multiput(1140.55,568.17)(-1.547,-2.000){2}{\rule{0.350pt}{0.400pt}}
\multiput(1136.37,565.95)(-0.685,-0.447){3}{\rule{0.633pt}{0.108pt}}
\multiput(1137.69,566.17)(-2.685,-3.000){2}{\rule{0.317pt}{0.400pt}}
\put(1132,562.17){\rule{0.700pt}{0.400pt}}
\multiput(1133.55,563.17)(-1.547,-2.000){2}{\rule{0.350pt}{0.400pt}}
\multiput(1129.37,560.95)(-0.685,-0.447){3}{\rule{0.633pt}{0.108pt}}
\multiput(1130.69,561.17)(-2.685,-3.000){2}{\rule{0.317pt}{0.400pt}}
\put(1125,557.17){\rule{0.700pt}{0.400pt}}
\multiput(1126.55,558.17)(-1.547,-2.000){2}{\rule{0.350pt}{0.400pt}}
\multiput(1122.92,555.95)(-0.462,-0.447){3}{\rule{0.500pt}{0.108pt}}
\multiput(1123.96,556.17)(-1.962,-3.000){2}{\rule{0.250pt}{0.400pt}}
\put(1118,552.17){\rule{0.900pt}{0.400pt}}
\multiput(1120.13,553.17)(-2.132,-2.000){2}{\rule{0.450pt}{0.400pt}}
\multiput(1115.92,550.95)(-0.462,-0.447){3}{\rule{0.500pt}{0.108pt}}
\multiput(1116.96,551.17)(-1.962,-3.000){2}{\rule{0.250pt}{0.400pt}}
\put(1112,547.17){\rule{0.700pt}{0.400pt}}
\multiput(1113.55,548.17)(-1.547,-2.000){2}{\rule{0.350pt}{0.400pt}}
\multiput(1109.37,545.95)(-0.685,-0.447){3}{\rule{0.633pt}{0.108pt}}
\multiput(1110.69,546.17)(-2.685,-3.000){2}{\rule{0.317pt}{0.400pt}}
\put(1105,542.17){\rule{0.700pt}{0.400pt}}
\multiput(1106.55,543.17)(-1.547,-2.000){2}{\rule{0.350pt}{0.400pt}}
\multiput(1102.92,540.95)(-0.462,-0.447){3}{\rule{0.500pt}{0.108pt}}
\multiput(1103.96,541.17)(-1.962,-3.000){2}{\rule{0.250pt}{0.400pt}}
\put(1098,537.17){\rule{0.900pt}{0.400pt}}
\multiput(1100.13,538.17)(-2.132,-2.000){2}{\rule{0.450pt}{0.400pt}}
\put(1095,535.17){\rule{0.700pt}{0.400pt}}
\multiput(1096.55,536.17)(-1.547,-2.000){2}{\rule{0.350pt}{0.400pt}}
\multiput(1092.37,533.95)(-0.685,-0.447){3}{\rule{0.633pt}{0.108pt}}
\multiput(1093.69,534.17)(-2.685,-3.000){2}{\rule{0.317pt}{0.400pt}}
\put(1088,530.17){\rule{0.700pt}{0.400pt}}
\multiput(1089.55,531.17)(-1.547,-2.000){2}{\rule{0.350pt}{0.400pt}}
\multiput(1085.92,528.95)(-0.462,-0.447){3}{\rule{0.500pt}{0.108pt}}
\multiput(1086.96,529.17)(-1.962,-3.000){2}{\rule{0.250pt}{0.400pt}}
\put(1081,525.17){\rule{0.900pt}{0.400pt}}
\multiput(1083.13,526.17)(-2.132,-2.000){2}{\rule{0.450pt}{0.400pt}}
\multiput(1078.92,523.95)(-0.462,-0.447){3}{\rule{0.500pt}{0.108pt}}
\multiput(1079.96,524.17)(-1.962,-3.000){2}{\rule{0.250pt}{0.400pt}}
\put(1075,520.17){\rule{0.700pt}{0.400pt}}
\multiput(1076.55,521.17)(-1.547,-2.000){2}{\rule{0.350pt}{0.400pt}}
\multiput(1072.37,518.95)(-0.685,-0.447){3}{\rule{0.633pt}{0.108pt}}
\multiput(1073.69,519.17)(-2.685,-3.000){2}{\rule{0.317pt}{0.400pt}}
\put(1068,515.17){\rule{0.700pt}{0.400pt}}
\multiput(1069.55,516.17)(-1.547,-2.000){2}{\rule{0.350pt}{0.400pt}}
\multiput(1065.92,513.95)(-0.462,-0.447){3}{\rule{0.500pt}{0.108pt}}
\multiput(1066.96,514.17)(-1.962,-3.000){2}{\rule{0.250pt}{0.400pt}}
\put(1061,510.17){\rule{0.900pt}{0.400pt}}
\multiput(1063.13,511.17)(-2.132,-2.000){2}{\rule{0.450pt}{0.400pt}}
\multiput(1058.92,508.95)(-0.462,-0.447){3}{\rule{0.500pt}{0.108pt}}
\multiput(1059.96,509.17)(-1.962,-3.000){2}{\rule{0.250pt}{0.400pt}}
\put(1054,505.17){\rule{0.900pt}{0.400pt}}
\multiput(1056.13,506.17)(-2.132,-2.000){2}{\rule{0.450pt}{0.400pt}}
\multiput(1051.92,503.95)(-0.462,-0.447){3}{\rule{0.500pt}{0.108pt}}
\multiput(1052.96,504.17)(-1.962,-3.000){2}{\rule{0.250pt}{0.400pt}}
\put(1048,500.17){\rule{0.700pt}{0.400pt}}
\multiput(1049.55,501.17)(-1.547,-2.000){2}{\rule{0.350pt}{0.400pt}}
\put(1044,498.17){\rule{0.900pt}{0.400pt}}
\multiput(1046.13,499.17)(-2.132,-2.000){2}{\rule{0.450pt}{0.400pt}}
\multiput(1041.92,496.95)(-0.462,-0.447){3}{\rule{0.500pt}{0.108pt}}
\multiput(1042.96,497.17)(-1.962,-3.000){2}{\rule{0.250pt}{0.400pt}}
\put(1038,493.17){\rule{0.700pt}{0.400pt}}
\multiput(1039.55,494.17)(-1.547,-2.000){2}{\rule{0.350pt}{0.400pt}}
\multiput(1035.37,491.95)(-0.685,-0.447){3}{\rule{0.633pt}{0.108pt}}
\multiput(1036.69,492.17)(-2.685,-3.000){2}{\rule{0.317pt}{0.400pt}}
\put(1031,488.17){\rule{0.700pt}{0.400pt}}
\multiput(1032.55,489.17)(-1.547,-2.000){2}{\rule{0.350pt}{0.400pt}}
\multiput(1028.92,486.95)(-0.462,-0.447){3}{\rule{0.500pt}{0.108pt}}
\multiput(1029.96,487.17)(-1.962,-3.000){2}{\rule{0.250pt}{0.400pt}}
\put(1024,483.17){\rule{0.900pt}{0.400pt}}
\multiput(1026.13,484.17)(-2.132,-2.000){2}{\rule{0.450pt}{0.400pt}}
\multiput(1021.92,481.95)(-0.462,-0.447){3}{\rule{0.500pt}{0.108pt}}
\multiput(1022.96,482.17)(-1.962,-3.000){2}{\rule{0.250pt}{0.400pt}}
\put(1017,478.17){\rule{0.900pt}{0.400pt}}
\multiput(1019.13,479.17)(-2.132,-2.000){2}{\rule{0.450pt}{0.400pt}}
\multiput(1014.92,476.95)(-0.462,-0.447){3}{\rule{0.500pt}{0.108pt}}
\multiput(1015.96,477.17)(-1.962,-3.000){2}{\rule{0.250pt}{0.400pt}}
\put(1011,473.17){\rule{0.700pt}{0.400pt}}
\multiput(1012.55,474.17)(-1.547,-2.000){2}{\rule{0.350pt}{0.400pt}}
\multiput(1008.37,471.95)(-0.685,-0.447){3}{\rule{0.633pt}{0.108pt}}
\multiput(1009.69,472.17)(-2.685,-3.000){2}{\rule{0.317pt}{0.400pt}}
\put(1004,468.67){\rule{0.723pt}{0.400pt}}
\multiput(1005.50,469.17)(-1.500,-1.000){2}{\rule{0.361pt}{0.400pt}}
\multiput(1001.92,467.95)(-0.462,-0.447){3}{\rule{0.500pt}{0.108pt}}
\multiput(1002.96,468.17)(-1.962,-3.000){2}{\rule{0.250pt}{0.400pt}}
\put(997,464.17){\rule{0.900pt}{0.400pt}}
\multiput(999.13,465.17)(-2.132,-2.000){2}{\rule{0.450pt}{0.400pt}}
\multiput(994.92,462.95)(-0.462,-0.447){3}{\rule{0.500pt}{0.108pt}}
\multiput(995.96,463.17)(-1.962,-3.000){2}{\rule{0.250pt}{0.400pt}}
\put(991,459.17){\rule{0.700pt}{0.400pt}}
\multiput(992.55,460.17)(-1.547,-2.000){2}{\rule{0.350pt}{0.400pt}}
\put(987,457.17){\rule{0.900pt}{0.400pt}}
\multiput(989.13,458.17)(-2.132,-2.000){2}{\rule{0.450pt}{0.400pt}}
\multiput(984.92,455.95)(-0.462,-0.447){3}{\rule{0.500pt}{0.108pt}}
\multiput(985.96,456.17)(-1.962,-3.000){2}{\rule{0.250pt}{0.400pt}}
\put(980,452.17){\rule{0.900pt}{0.400pt}}
\multiput(982.13,453.17)(-2.132,-2.000){2}{\rule{0.450pt}{0.400pt}}
\multiput(977.92,450.95)(-0.462,-0.447){3}{\rule{0.500pt}{0.108pt}}
\multiput(978.96,451.17)(-1.962,-3.000){2}{\rule{0.250pt}{0.400pt}}
\put(974,447.17){\rule{0.700pt}{0.400pt}}
\multiput(975.55,448.17)(-1.547,-2.000){2}{\rule{0.350pt}{0.400pt}}
\multiput(971.37,445.95)(-0.685,-0.447){3}{\rule{0.633pt}{0.108pt}}
\multiput(972.69,446.17)(-2.685,-3.000){2}{\rule{0.317pt}{0.400pt}}
\put(967,442.17){\rule{0.700pt}{0.400pt}}
\multiput(968.55,443.17)(-1.547,-2.000){2}{\rule{0.350pt}{0.400pt}}
\multiput(964.92,440.95)(-0.462,-0.447){3}{\rule{0.500pt}{0.108pt}}
\multiput(965.96,441.17)(-1.962,-3.000){2}{\rule{0.250pt}{0.400pt}}
\put(960,437.17){\rule{0.900pt}{0.400pt}}
\multiput(962.13,438.17)(-2.132,-2.000){2}{\rule{0.450pt}{0.400pt}}
\multiput(957.92,435.95)(-0.462,-0.447){3}{\rule{0.500pt}{0.108pt}}
\multiput(958.96,436.17)(-1.962,-3.000){2}{\rule{0.250pt}{0.400pt}}
\put(954,432.17){\rule{0.700pt}{0.400pt}}
\multiput(955.55,433.17)(-1.547,-2.000){2}{\rule{0.350pt}{0.400pt}}
\multiput(951.37,430.95)(-0.685,-0.447){3}{\rule{0.633pt}{0.108pt}}
\multiput(952.69,431.17)(-2.685,-3.000){2}{\rule{0.317pt}{0.400pt}}
\put(947,427.17){\rule{0.700pt}{0.400pt}}
\multiput(948.55,428.17)(-1.547,-2.000){2}{\rule{0.350pt}{0.400pt}}
\multiput(944.37,425.95)(-0.685,-0.447){3}{\rule{0.633pt}{0.108pt}}
\multiput(945.69,426.17)(-2.685,-3.000){2}{\rule{0.317pt}{0.400pt}}
\put(940,422.17){\rule{0.700pt}{0.400pt}}
\multiput(941.55,423.17)(-1.547,-2.000){2}{\rule{0.350pt}{0.400pt}}
\put(560,535.67){\rule{1.686pt}{0.400pt}}
\multiput(560.00,535.17)(3.500,1.000){2}{\rule{0.843pt}{0.400pt}}
\put(567,537.17){\rule{1.500pt}{0.400pt}}
\multiput(567.00,536.17)(3.887,2.000){2}{\rule{0.750pt}{0.400pt}}
\put(574,538.67){\rule{1.927pt}{0.400pt}}
\multiput(574.00,538.17)(4.000,1.000){2}{\rule{0.964pt}{0.400pt}}
\put(582,539.67){\rule{1.686pt}{0.400pt}}
\multiput(582.00,539.17)(3.500,1.000){2}{\rule{0.843pt}{0.400pt}}
\put(589,541.17){\rule{1.500pt}{0.400pt}}
\multiput(589.00,540.17)(3.887,2.000){2}{\rule{0.750pt}{0.400pt}}
\put(596,542.67){\rule{1.686pt}{0.400pt}}
\multiput(596.00,542.17)(3.500,1.000){2}{\rule{0.843pt}{0.400pt}}
\put(603,543.67){\rule{1.686pt}{0.400pt}}
\multiput(603.00,543.17)(3.500,1.000){2}{\rule{0.843pt}{0.400pt}}
\put(610,545.17){\rule{1.700pt}{0.400pt}}
\multiput(610.00,544.17)(4.472,2.000){2}{\rule{0.850pt}{0.400pt}}
\put(618,546.67){\rule{1.686pt}{0.400pt}}
\multiput(618.00,546.17)(3.500,1.000){2}{\rule{0.843pt}{0.400pt}}
\put(625,547.67){\rule{1.686pt}{0.400pt}}
\multiput(625.00,547.17)(3.500,1.000){2}{\rule{0.843pt}{0.400pt}}
\put(632,548.67){\rule{1.686pt}{0.400pt}}
\multiput(632.00,548.17)(3.500,1.000){2}{\rule{0.843pt}{0.400pt}}
\put(639,550.17){\rule{1.500pt}{0.400pt}}
\multiput(639.00,549.17)(3.887,2.000){2}{\rule{0.750pt}{0.400pt}}
\put(646,551.67){\rule{1.927pt}{0.400pt}}
\multiput(646.00,551.17)(4.000,1.000){2}{\rule{0.964pt}{0.400pt}}
\put(654,552.67){\rule{1.686pt}{0.400pt}}
\multiput(654.00,552.17)(3.500,1.000){2}{\rule{0.843pt}{0.400pt}}
\put(661,554.17){\rule{1.500pt}{0.400pt}}
\multiput(661.00,553.17)(3.887,2.000){2}{\rule{0.750pt}{0.400pt}}
\put(668,555.67){\rule{1.686pt}{0.400pt}}
\multiput(668.00,555.17)(3.500,1.000){2}{\rule{0.843pt}{0.400pt}}
\put(675,556.67){\rule{1.927pt}{0.400pt}}
\multiput(675.00,556.17)(4.000,1.000){2}{\rule{0.964pt}{0.400pt}}
\put(683,558.17){\rule{1.500pt}{0.400pt}}
\multiput(683.00,557.17)(3.887,2.000){2}{\rule{0.750pt}{0.400pt}}
\put(690,559.67){\rule{1.686pt}{0.400pt}}
\multiput(690.00,559.17)(3.500,1.000){2}{\rule{0.843pt}{0.400pt}}
\put(697,560.67){\rule{1.686pt}{0.400pt}}
\multiput(697.00,560.17)(3.500,1.000){2}{\rule{0.843pt}{0.400pt}}
\put(704,562.17){\rule{1.500pt}{0.400pt}}
\multiput(704.00,561.17)(3.887,2.000){2}{\rule{0.750pt}{0.400pt}}
\put(711,563.67){\rule{1.927pt}{0.400pt}}
\multiput(711.00,563.17)(4.000,1.000){2}{\rule{0.964pt}{0.400pt}}
\put(719,564.67){\rule{1.686pt}{0.400pt}}
\multiput(719.00,564.17)(3.500,1.000){2}{\rule{0.843pt}{0.400pt}}
\put(726,565.67){\rule{1.686pt}{0.400pt}}
\multiput(726.00,565.17)(3.500,1.000){2}{\rule{0.843pt}{0.400pt}}
\put(733,567.17){\rule{1.500pt}{0.400pt}}
\multiput(733.00,566.17)(3.887,2.000){2}{\rule{0.750pt}{0.400pt}}
\put(740,568.67){\rule{1.686pt}{0.400pt}}
\multiput(740.00,568.17)(3.500,1.000){2}{\rule{0.843pt}{0.400pt}}
\put(747,569.67){\rule{1.686pt}{0.400pt}}
\multiput(747.00,569.17)(3.500,1.000){2}{\rule{0.843pt}{0.400pt}}
\put(754,571.17){\rule{1.500pt}{0.400pt}}
\multiput(754.00,570.17)(3.887,2.000){2}{\rule{0.750pt}{0.400pt}}
\put(761,572.67){\rule{1.686pt}{0.400pt}}
\multiput(761.00,572.17)(3.500,1.000){2}{\rule{0.843pt}{0.400pt}}
\put(768,573.67){\rule{1.686pt}{0.400pt}}
\multiput(768.00,573.17)(3.500,1.000){2}{\rule{0.843pt}{0.400pt}}
\put(775,575.17){\rule{1.700pt}{0.400pt}}
\multiput(775.00,574.17)(4.472,2.000){2}{\rule{0.850pt}{0.400pt}}
\put(783,576.67){\rule{1.686pt}{0.400pt}}
\multiput(783.00,576.17)(3.500,1.000){2}{\rule{0.843pt}{0.400pt}}
\put(790,577.67){\rule{1.686pt}{0.400pt}}
\multiput(790.00,577.17)(3.500,1.000){2}{\rule{0.843pt}{0.400pt}}
\put(797,578.67){\rule{1.686pt}{0.400pt}}
\multiput(797.00,578.17)(3.500,1.000){2}{\rule{0.843pt}{0.400pt}}
\put(804,580.17){\rule{1.500pt}{0.400pt}}
\multiput(804.00,579.17)(3.887,2.000){2}{\rule{0.750pt}{0.400pt}}
\put(811,581.67){\rule{1.927pt}{0.400pt}}
\multiput(811.00,581.17)(4.000,1.000){2}{\rule{0.964pt}{0.400pt}}
\put(819,582.67){\rule{1.686pt}{0.400pt}}
\multiput(819.00,582.17)(3.500,1.000){2}{\rule{0.843pt}{0.400pt}}
\put(826,584.17){\rule{1.500pt}{0.400pt}}
\multiput(826.00,583.17)(3.887,2.000){2}{\rule{0.750pt}{0.400pt}}
\put(833,585.67){\rule{1.686pt}{0.400pt}}
\multiput(833.00,585.17)(3.500,1.000){2}{\rule{0.843pt}{0.400pt}}
\put(840,586.67){\rule{1.686pt}{0.400pt}}
\multiput(840.00,586.17)(3.500,1.000){2}{\rule{0.843pt}{0.400pt}}
\put(847,588.17){\rule{1.700pt}{0.400pt}}
\multiput(847.00,587.17)(4.472,2.000){2}{\rule{0.850pt}{0.400pt}}
\put(855,589.67){\rule{1.686pt}{0.400pt}}
\multiput(855.00,589.17)(3.500,1.000){2}{\rule{0.843pt}{0.400pt}}
\put(862,590.67){\rule{1.686pt}{0.400pt}}
\multiput(862.00,590.17)(3.500,1.000){2}{\rule{0.843pt}{0.400pt}}
\put(869,592.17){\rule{1.500pt}{0.400pt}}
\multiput(869.00,591.17)(3.887,2.000){2}{\rule{0.750pt}{0.400pt}}
\put(876,593.67){\rule{1.927pt}{0.400pt}}
\multiput(876.00,593.17)(4.000,1.000){2}{\rule{0.964pt}{0.400pt}}
\put(884,594.67){\rule{1.686pt}{0.400pt}}
\multiput(884.00,594.17)(3.500,1.000){2}{\rule{0.843pt}{0.400pt}}
\put(891,595.67){\rule{1.686pt}{0.400pt}}
\multiput(891.00,595.17)(3.500,1.000){2}{\rule{0.843pt}{0.400pt}}
\put(898,597.17){\rule{1.500pt}{0.400pt}}
\multiput(898.00,596.17)(3.887,2.000){2}{\rule{0.750pt}{0.400pt}}
\put(905,598.67){\rule{1.686pt}{0.400pt}}
\multiput(905.00,598.17)(3.500,1.000){2}{\rule{0.843pt}{0.400pt}}
\put(912,599.67){\rule{1.927pt}{0.400pt}}
\multiput(912.00,599.17)(4.000,1.000){2}{\rule{0.964pt}{0.400pt}}
\put(920,601.17){\rule{1.500pt}{0.400pt}}
\multiput(920.00,600.17)(3.887,2.000){2}{\rule{0.750pt}{0.400pt}}
\put(927,602.67){\rule{1.686pt}{0.400pt}}
\multiput(927.00,602.17)(3.500,1.000){2}{\rule{0.843pt}{0.400pt}}
\put(934,603.67){\rule{1.686pt}{0.400pt}}
\multiput(934.00,603.17)(3.500,1.000){2}{\rule{0.843pt}{0.400pt}}
\put(941,605.17){\rule{1.500pt}{0.400pt}}
\multiput(941.00,604.17)(3.887,2.000){2}{\rule{0.750pt}{0.400pt}}
\put(948,606.67){\rule{1.927pt}{0.400pt}}
\multiput(948.00,606.17)(4.000,1.000){2}{\rule{0.964pt}{0.400pt}}
\put(956,607.67){\rule{1.686pt}{0.400pt}}
\multiput(956.00,607.17)(3.500,1.000){2}{\rule{0.843pt}{0.400pt}}
\put(963,609.17){\rule{1.500pt}{0.400pt}}
\multiput(963.00,608.17)(3.887,2.000){2}{\rule{0.750pt}{0.400pt}}
\put(970,610.67){\rule{1.686pt}{0.400pt}}
\multiput(970.00,610.17)(3.500,1.000){2}{\rule{0.843pt}{0.400pt}}
\put(977,611.67){\rule{1.927pt}{0.400pt}}
\multiput(977.00,611.17)(4.000,1.000){2}{\rule{0.964pt}{0.400pt}}
\put(985,612.67){\rule{1.686pt}{0.400pt}}
\multiput(985.00,612.17)(3.500,1.000){2}{\rule{0.843pt}{0.400pt}}
\put(992,614.17){\rule{1.500pt}{0.400pt}}
\multiput(992.00,613.17)(3.887,2.000){2}{\rule{0.750pt}{0.400pt}}
\put(999,615.67){\rule{1.686pt}{0.400pt}}
\multiput(999.00,615.17)(3.500,1.000){2}{\rule{0.843pt}{0.400pt}}
\put(1006,616.67){\rule{1.686pt}{0.400pt}}
\multiput(1006.00,616.17)(3.500,1.000){2}{\rule{0.843pt}{0.400pt}}
\put(1013,618.17){\rule{1.700pt}{0.400pt}}
\multiput(1013.00,617.17)(4.472,2.000){2}{\rule{0.850pt}{0.400pt}}
\put(1021,619.67){\rule{1.686pt}{0.400pt}}
\multiput(1021.00,619.17)(3.500,1.000){2}{\rule{0.843pt}{0.400pt}}
\put(1028,620.67){\rule{1.686pt}{0.400pt}}
\multiput(1028.00,620.17)(3.500,1.000){2}{\rule{0.843pt}{0.400pt}}
\put(1035,622.17){\rule{1.500pt}{0.400pt}}
\multiput(1035.00,621.17)(3.887,2.000){2}{\rule{0.750pt}{0.400pt}}
\put(1042,623.67){\rule{1.686pt}{0.400pt}}
\multiput(1042.00,623.17)(3.500,1.000){2}{\rule{0.843pt}{0.400pt}}
\put(1049,624.67){\rule{1.927pt}{0.400pt}}
\multiput(1049.00,624.17)(4.000,1.000){2}{\rule{0.964pt}{0.400pt}}
\put(1057,625.67){\rule{1.686pt}{0.400pt}}
\multiput(1057.00,625.17)(3.500,1.000){2}{\rule{0.843pt}{0.400pt}}
\put(1064,627.17){\rule{1.500pt}{0.400pt}}
\multiput(1064.00,626.17)(3.887,2.000){2}{\rule{0.750pt}{0.400pt}}
\put(1071,628.67){\rule{1.686pt}{0.400pt}}
\multiput(1071.00,628.17)(3.500,1.000){2}{\rule{0.843pt}{0.400pt}}
\put(1078,629.67){\rule{1.927pt}{0.400pt}}
\multiput(1078.00,629.17)(4.000,1.000){2}{\rule{0.964pt}{0.400pt}}
\put(1086,631.17){\rule{1.500pt}{0.400pt}}
\multiput(1086.00,630.17)(3.887,2.000){2}{\rule{0.750pt}{0.400pt}}
\put(1093,632.67){\rule{1.686pt}{0.400pt}}
\multiput(1093.00,632.17)(3.500,1.000){2}{\rule{0.843pt}{0.400pt}}
\put(1100,633.67){\rule{1.686pt}{0.400pt}}
\multiput(1100.00,633.17)(3.500,1.000){2}{\rule{0.843pt}{0.400pt}}
\put(1107,635.17){\rule{1.500pt}{0.400pt}}
\multiput(1107.00,634.17)(3.887,2.000){2}{\rule{0.750pt}{0.400pt}}
\put(1114,636.67){\rule{1.927pt}{0.400pt}}
\multiput(1114.00,636.17)(4.000,1.000){2}{\rule{0.964pt}{0.400pt}}
\put(1122,637.67){\rule{1.686pt}{0.400pt}}
\multiput(1122.00,637.17)(3.500,1.000){2}{\rule{0.843pt}{0.400pt}}
\put(1129,639.17){\rule{1.500pt}{0.400pt}}
\multiput(1129.00,638.17)(3.887,2.000){2}{\rule{0.750pt}{0.400pt}}
\put(1136,640.67){\rule{1.686pt}{0.400pt}}
\multiput(1136.00,640.17)(3.500,1.000){2}{\rule{0.843pt}{0.400pt}}
\put(1143,641.67){\rule{1.686pt}{0.400pt}}
\multiput(1143.00,641.17)(3.500,1.000){2}{\rule{0.843pt}{0.400pt}}
\put(1150,642.67){\rule{1.927pt}{0.400pt}}
\multiput(1150.00,642.17)(4.000,1.000){2}{\rule{0.964pt}{0.400pt}}
\put(1158,644.17){\rule{1.500pt}{0.400pt}}
\multiput(1158.00,643.17)(3.887,2.000){2}{\rule{0.750pt}{0.400pt}}
\put(1165,645.67){\rule{1.686pt}{0.400pt}}
\multiput(1165.00,645.17)(3.500,1.000){2}{\rule{0.843pt}{0.400pt}}
\put(1172,646.67){\rule{1.686pt}{0.400pt}}
\multiput(1172.00,646.17)(3.500,1.000){2}{\rule{0.843pt}{0.400pt}}
\put(1179,648.17){\rule{1.700pt}{0.400pt}}
\multiput(1179.00,647.17)(4.472,2.000){2}{\rule{0.850pt}{0.400pt}}
\put(1187,649.67){\rule{1.686pt}{0.400pt}}
\multiput(1187.00,649.17)(3.500,1.000){2}{\rule{0.843pt}{0.400pt}}
\put(1194,650.67){\rule{1.686pt}{0.400pt}}
\multiput(1194.00,650.17)(3.500,1.000){2}{\rule{0.843pt}{0.400pt}}
\put(1201,652.17){\rule{1.500pt}{0.400pt}}
\multiput(1201.00,651.17)(3.887,2.000){2}{\rule{0.750pt}{0.400pt}}
\put(1208,653.67){\rule{1.686pt}{0.400pt}}
\multiput(1208.00,653.17)(3.500,1.000){2}{\rule{0.843pt}{0.400pt}}
\put(1215,654.67){\rule{1.927pt}{0.400pt}}
\multiput(1215.00,654.17)(4.000,1.000){2}{\rule{0.964pt}{0.400pt}}
\put(1223,655.67){\rule{1.686pt}{0.400pt}}
\multiput(1223.00,655.17)(3.500,1.000){2}{\rule{0.843pt}{0.400pt}}
\put(1230,657.17){\rule{1.500pt}{0.400pt}}
\multiput(1230.00,656.17)(3.887,2.000){2}{\rule{0.750pt}{0.400pt}}
\put(1237,658.67){\rule{1.686pt}{0.400pt}}
\multiput(1237.00,658.17)(3.500,1.000){2}{\rule{0.843pt}{0.400pt}}
\put(1244,659.67){\rule{1.686pt}{0.400pt}}
\multiput(1244.00,659.17)(3.500,1.000){2}{\rule{0.843pt}{0.400pt}}
\put(1251,661.17){\rule{1.700pt}{0.400pt}}
\multiput(1251.00,660.17)(4.472,2.000){2}{\rule{0.850pt}{0.400pt}}
\put(1259,662.67){\rule{1.686pt}{0.400pt}}
\multiput(1259.00,662.17)(3.500,1.000){2}{\rule{0.843pt}{0.400pt}}
\put(1266,663.67){\rule{1.686pt}{0.400pt}}
\multiput(1266.00,663.17)(3.500,1.000){2}{\rule{0.843pt}{0.400pt}}
\put(504,495.17){\rule{1.700pt}{0.400pt}}
\multiput(504.00,494.17)(4.472,2.000){2}{\rule{0.850pt}{0.400pt}}
\put(512,496.67){\rule{1.686pt}{0.400pt}}
\multiput(512.00,496.17)(3.500,1.000){2}{\rule{0.843pt}{0.400pt}}
\put(519,497.67){\rule{1.686pt}{0.400pt}}
\multiput(519.00,497.17)(3.500,1.000){2}{\rule{0.843pt}{0.400pt}}
\put(526,499.17){\rule{1.500pt}{0.400pt}}
\multiput(526.00,498.17)(3.887,2.000){2}{\rule{0.750pt}{0.400pt}}
\put(533,500.67){\rule{1.686pt}{0.400pt}}
\multiput(533.00,500.17)(3.500,1.000){2}{\rule{0.843pt}{0.400pt}}
\put(540,501.67){\rule{1.927pt}{0.400pt}}
\multiput(540.00,501.17)(4.000,1.000){2}{\rule{0.964pt}{0.400pt}}
\put(548,503.17){\rule{1.500pt}{0.400pt}}
\multiput(548.00,502.17)(3.887,2.000){2}{\rule{0.750pt}{0.400pt}}
\put(555,504.67){\rule{1.686pt}{0.400pt}}
\multiput(555.00,504.17)(3.500,1.000){2}{\rule{0.843pt}{0.400pt}}
\put(562,505.67){\rule{1.686pt}{0.400pt}}
\multiput(562.00,505.17)(3.500,1.000){2}{\rule{0.843pt}{0.400pt}}
\put(569,506.67){\rule{1.927pt}{0.400pt}}
\multiput(569.00,506.17)(4.000,1.000){2}{\rule{0.964pt}{0.400pt}}
\put(577,508.17){\rule{1.500pt}{0.400pt}}
\multiput(577.00,507.17)(3.887,2.000){2}{\rule{0.750pt}{0.400pt}}
\put(584,509.67){\rule{1.686pt}{0.400pt}}
\multiput(584.00,509.17)(3.500,1.000){2}{\rule{0.843pt}{0.400pt}}
\put(591,510.67){\rule{1.686pt}{0.400pt}}
\multiput(591.00,510.17)(3.500,1.000){2}{\rule{0.843pt}{0.400pt}}
\put(598,512.17){\rule{1.500pt}{0.400pt}}
\multiput(598.00,511.17)(3.887,2.000){2}{\rule{0.750pt}{0.400pt}}
\put(605,513.67){\rule{1.927pt}{0.400pt}}
\multiput(605.00,513.17)(4.000,1.000){2}{\rule{0.964pt}{0.400pt}}
\put(613,514.67){\rule{1.686pt}{0.400pt}}
\multiput(613.00,514.17)(3.500,1.000){2}{\rule{0.843pt}{0.400pt}}
\put(620,516.17){\rule{1.500pt}{0.400pt}}
\multiput(620.00,515.17)(3.887,2.000){2}{\rule{0.750pt}{0.400pt}}
\put(627,517.67){\rule{1.686pt}{0.400pt}}
\multiput(627.00,517.17)(3.500,1.000){2}{\rule{0.843pt}{0.400pt}}
\put(634,518.67){\rule{1.686pt}{0.400pt}}
\multiput(634.00,518.17)(3.500,1.000){2}{\rule{0.843pt}{0.400pt}}
\put(641,520.17){\rule{1.700pt}{0.400pt}}
\multiput(641.00,519.17)(4.472,2.000){2}{\rule{0.850pt}{0.400pt}}
\put(649,521.67){\rule{1.686pt}{0.400pt}}
\multiput(649.00,521.17)(3.500,1.000){2}{\rule{0.843pt}{0.400pt}}
\put(656,522.67){\rule{1.686pt}{0.400pt}}
\multiput(656.00,522.17)(3.500,1.000){2}{\rule{0.843pt}{0.400pt}}
\put(663,523.67){\rule{1.686pt}{0.400pt}}
\multiput(663.00,523.17)(3.500,1.000){2}{\rule{0.843pt}{0.400pt}}
\put(670,525.17){\rule{1.700pt}{0.400pt}}
\multiput(670.00,524.17)(4.472,2.000){2}{\rule{0.850pt}{0.400pt}}
\put(678,526.67){\rule{1.686pt}{0.400pt}}
\multiput(678.00,526.17)(3.500,1.000){2}{\rule{0.843pt}{0.400pt}}
\put(685,527.67){\rule{1.686pt}{0.400pt}}
\multiput(685.00,527.17)(3.500,1.000){2}{\rule{0.843pt}{0.400pt}}
\put(692,529.17){\rule{1.500pt}{0.400pt}}
\multiput(692.00,528.17)(3.887,2.000){2}{\rule{0.750pt}{0.400pt}}
\put(699,530.67){\rule{1.686pt}{0.400pt}}
\multiput(699.00,530.17)(3.500,1.000){2}{\rule{0.843pt}{0.400pt}}
\put(706,531.67){\rule{1.927pt}{0.400pt}}
\multiput(706.00,531.17)(4.000,1.000){2}{\rule{0.964pt}{0.400pt}}
\put(714,533.17){\rule{1.500pt}{0.400pt}}
\multiput(714.00,532.17)(3.887,2.000){2}{\rule{0.750pt}{0.400pt}}
\put(721,534.67){\rule{1.686pt}{0.400pt}}
\multiput(721.00,534.17)(3.500,1.000){2}{\rule{0.843pt}{0.400pt}}
\put(728,535.67){\rule{1.686pt}{0.400pt}}
\multiput(728.00,535.17)(3.500,1.000){2}{\rule{0.843pt}{0.400pt}}
\put(735,536.67){\rule{1.686pt}{0.400pt}}
\multiput(735.00,536.17)(3.500,1.000){2}{\rule{0.843pt}{0.400pt}}
\put(742,538.17){\rule{1.700pt}{0.400pt}}
\multiput(742.00,537.17)(4.472,2.000){2}{\rule{0.850pt}{0.400pt}}
\put(750,539.67){\rule{1.445pt}{0.400pt}}
\multiput(750.00,539.17)(3.000,1.000){2}{\rule{0.723pt}{0.400pt}}
\put(756,540.67){\rule{1.686pt}{0.400pt}}
\multiput(756.00,540.17)(3.500,1.000){2}{\rule{0.843pt}{0.400pt}}
\put(763,542.17){\rule{1.500pt}{0.400pt}}
\multiput(763.00,541.17)(3.887,2.000){2}{\rule{0.750pt}{0.400pt}}
\put(770,543.67){\rule{1.927pt}{0.400pt}}
\multiput(770.00,543.17)(4.000,1.000){2}{\rule{0.964pt}{0.400pt}}
\put(778,544.67){\rule{1.686pt}{0.400pt}}
\multiput(778.00,544.17)(3.500,1.000){2}{\rule{0.843pt}{0.400pt}}
\put(785,546.17){\rule{1.500pt}{0.400pt}}
\multiput(785.00,545.17)(3.887,2.000){2}{\rule{0.750pt}{0.400pt}}
\put(792,547.67){\rule{1.686pt}{0.400pt}}
\multiput(792.00,547.17)(3.500,1.000){2}{\rule{0.843pt}{0.400pt}}
\put(799,548.67){\rule{1.686pt}{0.400pt}}
\multiput(799.00,548.17)(3.500,1.000){2}{\rule{0.843pt}{0.400pt}}
\put(806,550.17){\rule{1.700pt}{0.400pt}}
\multiput(806.00,549.17)(4.472,2.000){2}{\rule{0.850pt}{0.400pt}}
\put(814,551.67){\rule{1.686pt}{0.400pt}}
\multiput(814.00,551.17)(3.500,1.000){2}{\rule{0.843pt}{0.400pt}}
\put(821,552.67){\rule{1.686pt}{0.400pt}}
\multiput(821.00,552.17)(3.500,1.000){2}{\rule{0.843pt}{0.400pt}}
\put(828,553.67){\rule{1.686pt}{0.400pt}}
\multiput(828.00,553.17)(3.500,1.000){2}{\rule{0.843pt}{0.400pt}}
\put(835,555.17){\rule{1.500pt}{0.400pt}}
\multiput(835.00,554.17)(3.887,2.000){2}{\rule{0.750pt}{0.400pt}}
\put(842,556.67){\rule{1.927pt}{0.400pt}}
\multiput(842.00,556.17)(4.000,1.000){2}{\rule{0.964pt}{0.400pt}}
\put(850,557.67){\rule{1.686pt}{0.400pt}}
\multiput(850.00,557.17)(3.500,1.000){2}{\rule{0.843pt}{0.400pt}}
\put(857,559.17){\rule{1.500pt}{0.400pt}}
\multiput(857.00,558.17)(3.887,2.000){2}{\rule{0.750pt}{0.400pt}}
\put(864,560.67){\rule{1.686pt}{0.400pt}}
\multiput(864.00,560.17)(3.500,1.000){2}{\rule{0.843pt}{0.400pt}}
\put(871,561.67){\rule{1.927pt}{0.400pt}}
\multiput(871.00,561.17)(4.000,1.000){2}{\rule{0.964pt}{0.400pt}}
\put(879,563.17){\rule{1.500pt}{0.400pt}}
\multiput(879.00,562.17)(3.887,2.000){2}{\rule{0.750pt}{0.400pt}}
\put(886,564.67){\rule{1.686pt}{0.400pt}}
\multiput(886.00,564.17)(3.500,1.000){2}{\rule{0.843pt}{0.400pt}}
\put(893,565.67){\rule{1.686pt}{0.400pt}}
\multiput(893.00,565.17)(3.500,1.000){2}{\rule{0.843pt}{0.400pt}}
\put(900,566.67){\rule{1.686pt}{0.400pt}}
\multiput(900.00,566.17)(3.500,1.000){2}{\rule{0.843pt}{0.400pt}}
\put(907,568.17){\rule{1.700pt}{0.400pt}}
\multiput(907.00,567.17)(4.472,2.000){2}{\rule{0.850pt}{0.400pt}}
\put(915,569.67){\rule{1.686pt}{0.400pt}}
\multiput(915.00,569.17)(3.500,1.000){2}{\rule{0.843pt}{0.400pt}}
\put(922,570.67){\rule{1.686pt}{0.400pt}}
\multiput(922.00,570.17)(3.500,1.000){2}{\rule{0.843pt}{0.400pt}}
\put(929,572.17){\rule{1.500pt}{0.400pt}}
\multiput(929.00,571.17)(3.887,2.000){2}{\rule{0.750pt}{0.400pt}}
\put(936,573.67){\rule{1.686pt}{0.400pt}}
\multiput(936.00,573.17)(3.500,1.000){2}{\rule{0.843pt}{0.400pt}}
\put(943,574.67){\rule{1.927pt}{0.400pt}}
\multiput(943.00,574.17)(4.000,1.000){2}{\rule{0.964pt}{0.400pt}}
\put(951,576.17){\rule{1.500pt}{0.400pt}}
\multiput(951.00,575.17)(3.887,2.000){2}{\rule{0.750pt}{0.400pt}}
\put(958,577.67){\rule{1.686pt}{0.400pt}}
\multiput(958.00,577.17)(3.500,1.000){2}{\rule{0.843pt}{0.400pt}}
\put(965,578.67){\rule{1.686pt}{0.400pt}}
\multiput(965.00,578.17)(3.500,1.000){2}{\rule{0.843pt}{0.400pt}}
\put(972,580.17){\rule{1.700pt}{0.400pt}}
\multiput(972.00,579.17)(4.472,2.000){2}{\rule{0.850pt}{0.400pt}}
\put(980,581.67){\rule{1.686pt}{0.400pt}}
\multiput(980.00,581.17)(3.500,1.000){2}{\rule{0.843pt}{0.400pt}}
\put(987,582.67){\rule{1.686pt}{0.400pt}}
\multiput(987.00,582.17)(3.500,1.000){2}{\rule{0.843pt}{0.400pt}}
\put(994,583.67){\rule{1.686pt}{0.400pt}}
\multiput(994.00,583.17)(3.500,1.000){2}{\rule{0.843pt}{0.400pt}}
\put(1001,585.17){\rule{1.500pt}{0.400pt}}
\multiput(1001.00,584.17)(3.887,2.000){2}{\rule{0.750pt}{0.400pt}}
\put(1008,586.67){\rule{1.927pt}{0.400pt}}
\multiput(1008.00,586.17)(4.000,1.000){2}{\rule{0.964pt}{0.400pt}}
\put(1016,587.67){\rule{1.686pt}{0.400pt}}
\multiput(1016.00,587.17)(3.500,1.000){2}{\rule{0.843pt}{0.400pt}}
\put(1023,589.17){\rule{1.500pt}{0.400pt}}
\multiput(1023.00,588.17)(3.887,2.000){2}{\rule{0.750pt}{0.400pt}}
\put(1030,590.67){\rule{1.686pt}{0.400pt}}
\multiput(1030.00,590.17)(3.500,1.000){2}{\rule{0.843pt}{0.400pt}}
\put(1037,591.67){\rule{1.686pt}{0.400pt}}
\multiput(1037.00,591.17)(3.500,1.000){2}{\rule{0.843pt}{0.400pt}}
\put(1044,593.17){\rule{1.700pt}{0.400pt}}
\multiput(1044.00,592.17)(4.472,2.000){2}{\rule{0.850pt}{0.400pt}}
\put(1052,594.67){\rule{1.686pt}{0.400pt}}
\multiput(1052.00,594.17)(3.500,1.000){2}{\rule{0.843pt}{0.400pt}}
\put(1059,595.67){\rule{1.686pt}{0.400pt}}
\multiput(1059.00,595.17)(3.500,1.000){2}{\rule{0.843pt}{0.400pt}}
\put(1066,596.67){\rule{1.686pt}{0.400pt}}
\multiput(1066.00,596.17)(3.500,1.000){2}{\rule{0.843pt}{0.400pt}}
\put(1073,598.17){\rule{1.700pt}{0.400pt}}
\multiput(1073.00,597.17)(4.472,2.000){2}{\rule{0.850pt}{0.400pt}}
\put(1081,599.67){\rule{1.686pt}{0.400pt}}
\multiput(1081.00,599.17)(3.500,1.000){2}{\rule{0.843pt}{0.400pt}}
\put(1088,600.67){\rule{1.686pt}{0.400pt}}
\multiput(1088.00,600.17)(3.500,1.000){2}{\rule{0.843pt}{0.400pt}}
\put(1095,602.17){\rule{1.500pt}{0.400pt}}
\multiput(1095.00,601.17)(3.887,2.000){2}{\rule{0.750pt}{0.400pt}}
\put(1102,603.67){\rule{1.686pt}{0.400pt}}
\multiput(1102.00,603.17)(3.500,1.000){2}{\rule{0.843pt}{0.400pt}}
\put(1109,604.67){\rule{1.927pt}{0.400pt}}
\multiput(1109.00,604.17)(4.000,1.000){2}{\rule{0.964pt}{0.400pt}}
\put(1117,606.17){\rule{1.500pt}{0.400pt}}
\multiput(1117.00,605.17)(3.887,2.000){2}{\rule{0.750pt}{0.400pt}}
\put(1124,607.67){\rule{1.686pt}{0.400pt}}
\multiput(1124.00,607.17)(3.500,1.000){2}{\rule{0.843pt}{0.400pt}}
\put(1131,608.67){\rule{1.686pt}{0.400pt}}
\multiput(1131.00,608.17)(3.500,1.000){2}{\rule{0.843pt}{0.400pt}}
\put(1138,610.17){\rule{1.500pt}{0.400pt}}
\multiput(1138.00,609.17)(3.887,2.000){2}{\rule{0.750pt}{0.400pt}}
\put(1145,611.67){\rule{1.927pt}{0.400pt}}
\multiput(1145.00,611.17)(4.000,1.000){2}{\rule{0.964pt}{0.400pt}}
\put(1153,612.67){\rule{1.686pt}{0.400pt}}
\multiput(1153.00,612.17)(3.500,1.000){2}{\rule{0.843pt}{0.400pt}}
\put(1160,613.67){\rule{1.686pt}{0.400pt}}
\multiput(1160.00,613.17)(3.500,1.000){2}{\rule{0.843pt}{0.400pt}}
\put(1167,615.17){\rule{1.500pt}{0.400pt}}
\multiput(1167.00,614.17)(3.887,2.000){2}{\rule{0.750pt}{0.400pt}}
\put(1174,616.67){\rule{1.927pt}{0.400pt}}
\multiput(1174.00,616.17)(4.000,1.000){2}{\rule{0.964pt}{0.400pt}}
\put(1182,617.67){\rule{1.686pt}{0.400pt}}
\multiput(1182.00,617.17)(3.500,1.000){2}{\rule{0.843pt}{0.400pt}}
\put(1189,619.17){\rule{1.500pt}{0.400pt}}
\multiput(1189.00,618.17)(3.887,2.000){2}{\rule{0.750pt}{0.400pt}}
\put(1196,620.67){\rule{1.686pt}{0.400pt}}
\multiput(1196.00,620.17)(3.500,1.000){2}{\rule{0.843pt}{0.400pt}}
\put(1203,621.67){\rule{1.686pt}{0.400pt}}
\multiput(1203.00,621.17)(3.500,1.000){2}{\rule{0.843pt}{0.400pt}}
\put(1210,623.17){\rule{1.700pt}{0.400pt}}
\multiput(1210.00,622.17)(4.472,2.000){2}{\rule{0.850pt}{0.400pt}}
\put(449,455.67){\rule{1.686pt}{0.400pt}}
\multiput(449.00,455.17)(3.500,1.000){2}{\rule{0.843pt}{0.400pt}}
\put(456,456.67){\rule{1.686pt}{0.400pt}}
\multiput(456.00,456.17)(3.500,1.000){2}{\rule{0.843pt}{0.400pt}}
\put(463,458.17){\rule{1.700pt}{0.400pt}}
\multiput(463.00,457.17)(4.472,2.000){2}{\rule{0.850pt}{0.400pt}}
\put(471,459.67){\rule{1.686pt}{0.400pt}}
\multiput(471.00,459.17)(3.500,1.000){2}{\rule{0.843pt}{0.400pt}}
\put(478,460.67){\rule{1.686pt}{0.400pt}}
\multiput(478.00,460.17)(3.500,1.000){2}{\rule{0.843pt}{0.400pt}}
\put(485,462.17){\rule{1.500pt}{0.400pt}}
\multiput(485.00,461.17)(3.887,2.000){2}{\rule{0.750pt}{0.400pt}}
\put(492,463.67){\rule{1.686pt}{0.400pt}}
\multiput(492.00,463.17)(3.500,1.000){2}{\rule{0.843pt}{0.400pt}}
\put(499,464.67){\rule{1.927pt}{0.400pt}}
\multiput(499.00,464.17)(4.000,1.000){2}{\rule{0.964pt}{0.400pt}}
\put(507,465.67){\rule{1.686pt}{0.400pt}}
\multiput(507.00,465.17)(3.500,1.000){2}{\rule{0.843pt}{0.400pt}}
\put(514,467.17){\rule{1.500pt}{0.400pt}}
\multiput(514.00,466.17)(3.887,2.000){2}{\rule{0.750pt}{0.400pt}}
\put(521,468.67){\rule{1.686pt}{0.400pt}}
\multiput(521.00,468.17)(3.500,1.000){2}{\rule{0.843pt}{0.400pt}}
\put(535,470.17){\rule{1.700pt}{0.400pt}}
\multiput(535.00,469.17)(4.472,2.000){2}{\rule{0.850pt}{0.400pt}}
\put(543,471.67){\rule{1.686pt}{0.400pt}}
\multiput(543.00,471.17)(3.500,1.000){2}{\rule{0.843pt}{0.400pt}}
\put(550,472.67){\rule{1.686pt}{0.400pt}}
\multiput(550.00,472.17)(3.500,1.000){2}{\rule{0.843pt}{0.400pt}}
\put(557,474.17){\rule{1.500pt}{0.400pt}}
\multiput(557.00,473.17)(3.887,2.000){2}{\rule{0.750pt}{0.400pt}}
\put(564,475.67){\rule{1.927pt}{0.400pt}}
\multiput(564.00,475.17)(4.000,1.000){2}{\rule{0.964pt}{0.400pt}}
\put(572,476.67){\rule{1.686pt}{0.400pt}}
\multiput(572.00,476.17)(3.500,1.000){2}{\rule{0.843pt}{0.400pt}}
\put(579,477.67){\rule{1.686pt}{0.400pt}}
\multiput(579.00,477.17)(3.500,1.000){2}{\rule{0.843pt}{0.400pt}}
\put(586,479.17){\rule{1.500pt}{0.400pt}}
\multiput(586.00,478.17)(3.887,2.000){2}{\rule{0.750pt}{0.400pt}}
\put(593,480.67){\rule{1.686pt}{0.400pt}}
\multiput(593.00,480.17)(3.500,1.000){2}{\rule{0.843pt}{0.400pt}}
\put(600,481.67){\rule{1.927pt}{0.400pt}}
\multiput(600.00,481.17)(4.000,1.000){2}{\rule{0.964pt}{0.400pt}}
\put(608,483.17){\rule{1.500pt}{0.400pt}}
\multiput(608.00,482.17)(3.887,2.000){2}{\rule{0.750pt}{0.400pt}}
\put(615,484.67){\rule{1.686pt}{0.400pt}}
\multiput(615.00,484.17)(3.500,1.000){2}{\rule{0.843pt}{0.400pt}}
\put(622,485.67){\rule{1.686pt}{0.400pt}}
\multiput(622.00,485.17)(3.500,1.000){2}{\rule{0.843pt}{0.400pt}}
\put(629,487.17){\rule{1.500pt}{0.400pt}}
\multiput(629.00,486.17)(3.887,2.000){2}{\rule{0.750pt}{0.400pt}}
\put(636,488.67){\rule{1.927pt}{0.400pt}}
\multiput(636.00,488.17)(4.000,1.000){2}{\rule{0.964pt}{0.400pt}}
\put(644,489.67){\rule{1.686pt}{0.400pt}}
\multiput(644.00,489.17)(3.500,1.000){2}{\rule{0.843pt}{0.400pt}}
\put(651,491.17){\rule{1.500pt}{0.400pt}}
\multiput(651.00,490.17)(3.887,2.000){2}{\rule{0.750pt}{0.400pt}}
\put(658,492.67){\rule{1.686pt}{0.400pt}}
\multiput(658.00,492.17)(3.500,1.000){2}{\rule{0.843pt}{0.400pt}}
\put(665,493.67){\rule{1.927pt}{0.400pt}}
\multiput(665.00,493.17)(4.000,1.000){2}{\rule{0.964pt}{0.400pt}}
\put(673,494.67){\rule{1.686pt}{0.400pt}}
\multiput(673.00,494.17)(3.500,1.000){2}{\rule{0.843pt}{0.400pt}}
\put(680,496.17){\rule{1.500pt}{0.400pt}}
\multiput(680.00,495.17)(3.887,2.000){2}{\rule{0.750pt}{0.400pt}}
\put(687,497.67){\rule{1.686pt}{0.400pt}}
\multiput(687.00,497.17)(3.500,1.000){2}{\rule{0.843pt}{0.400pt}}
\put(694,498.67){\rule{1.686pt}{0.400pt}}
\multiput(694.00,498.17)(3.500,1.000){2}{\rule{0.843pt}{0.400pt}}
\put(701,500.17){\rule{1.700pt}{0.400pt}}
\multiput(701.00,499.17)(4.472,2.000){2}{\rule{0.850pt}{0.400pt}}
\put(709,501.67){\rule{1.686pt}{0.400pt}}
\multiput(709.00,501.17)(3.500,1.000){2}{\rule{0.843pt}{0.400pt}}
\put(716,502.67){\rule{1.686pt}{0.400pt}}
\multiput(716.00,502.17)(3.500,1.000){2}{\rule{0.843pt}{0.400pt}}
\put(723,504.17){\rule{1.500pt}{0.400pt}}
\multiput(723.00,503.17)(3.887,2.000){2}{\rule{0.750pt}{0.400pt}}
\put(730,505.67){\rule{1.686pt}{0.400pt}}
\multiput(730.00,505.17)(3.500,1.000){2}{\rule{0.843pt}{0.400pt}}
\put(737,506.67){\rule{1.927pt}{0.400pt}}
\multiput(737.00,506.17)(4.000,1.000){2}{\rule{0.964pt}{0.400pt}}
\put(745,508.17){\rule{1.300pt}{0.400pt}}
\multiput(745.00,507.17)(3.302,2.000){2}{\rule{0.650pt}{0.400pt}}
\put(751,509.67){\rule{1.686pt}{0.400pt}}
\multiput(751.00,509.17)(3.500,1.000){2}{\rule{0.843pt}{0.400pt}}
\put(758,510.67){\rule{1.686pt}{0.400pt}}
\multiput(758.00,510.17)(3.500,1.000){2}{\rule{0.843pt}{0.400pt}}
\put(765,511.67){\rule{1.927pt}{0.400pt}}
\multiput(765.00,511.17)(4.000,1.000){2}{\rule{0.964pt}{0.400pt}}
\put(773,513.17){\rule{1.500pt}{0.400pt}}
\multiput(773.00,512.17)(3.887,2.000){2}{\rule{0.750pt}{0.400pt}}
\put(780,514.67){\rule{1.686pt}{0.400pt}}
\multiput(780.00,514.17)(3.500,1.000){2}{\rule{0.843pt}{0.400pt}}
\put(787,515.67){\rule{1.686pt}{0.400pt}}
\multiput(787.00,515.17)(3.500,1.000){2}{\rule{0.843pt}{0.400pt}}
\put(794,517.17){\rule{1.500pt}{0.400pt}}
\multiput(794.00,516.17)(3.887,2.000){2}{\rule{0.750pt}{0.400pt}}
\put(801,518.67){\rule{1.927pt}{0.400pt}}
\multiput(801.00,518.17)(4.000,1.000){2}{\rule{0.964pt}{0.400pt}}
\put(809,519.67){\rule{1.686pt}{0.400pt}}
\multiput(809.00,519.17)(3.500,1.000){2}{\rule{0.843pt}{0.400pt}}
\put(816,521.17){\rule{1.500pt}{0.400pt}}
\multiput(816.00,520.17)(3.887,2.000){2}{\rule{0.750pt}{0.400pt}}
\put(823,522.67){\rule{1.686pt}{0.400pt}}
\multiput(823.00,522.17)(3.500,1.000){2}{\rule{0.843pt}{0.400pt}}
\put(830,523.67){\rule{1.686pt}{0.400pt}}
\multiput(830.00,523.17)(3.500,1.000){2}{\rule{0.843pt}{0.400pt}}
\put(837,524.67){\rule{1.927pt}{0.400pt}}
\multiput(837.00,524.17)(4.000,1.000){2}{\rule{0.964pt}{0.400pt}}
\put(845,526.17){\rule{1.500pt}{0.400pt}}
\multiput(845.00,525.17)(3.887,2.000){2}{\rule{0.750pt}{0.400pt}}
\put(852,527.67){\rule{1.686pt}{0.400pt}}
\multiput(852.00,527.17)(3.500,1.000){2}{\rule{0.843pt}{0.400pt}}
\put(859,528.67){\rule{1.686pt}{0.400pt}}
\multiput(859.00,528.17)(3.500,1.000){2}{\rule{0.843pt}{0.400pt}}
\put(866,530.17){\rule{1.700pt}{0.400pt}}
\multiput(866.00,529.17)(4.472,2.000){2}{\rule{0.850pt}{0.400pt}}
\put(874,531.67){\rule{1.686pt}{0.400pt}}
\multiput(874.00,531.17)(3.500,1.000){2}{\rule{0.843pt}{0.400pt}}
\put(881,532.67){\rule{1.686pt}{0.400pt}}
\multiput(881.00,532.17)(3.500,1.000){2}{\rule{0.843pt}{0.400pt}}
\put(888,534.17){\rule{1.500pt}{0.400pt}}
\multiput(888.00,533.17)(3.887,2.000){2}{\rule{0.750pt}{0.400pt}}
\put(895,535.67){\rule{1.686pt}{0.400pt}}
\multiput(895.00,535.17)(3.500,1.000){2}{\rule{0.843pt}{0.400pt}}
\put(902,536.67){\rule{1.927pt}{0.400pt}}
\multiput(902.00,536.17)(4.000,1.000){2}{\rule{0.964pt}{0.400pt}}
\put(910,538.17){\rule{1.500pt}{0.400pt}}
\multiput(910.00,537.17)(3.887,2.000){2}{\rule{0.750pt}{0.400pt}}
\put(917,539.67){\rule{1.686pt}{0.400pt}}
\multiput(917.00,539.17)(3.500,1.000){2}{\rule{0.843pt}{0.400pt}}
\put(924,540.67){\rule{1.686pt}{0.400pt}}
\multiput(924.00,540.17)(3.500,1.000){2}{\rule{0.843pt}{0.400pt}}
\put(931,541.67){\rule{1.686pt}{0.400pt}}
\multiput(931.00,541.17)(3.500,1.000){2}{\rule{0.843pt}{0.400pt}}
\put(938,543.17){\rule{1.700pt}{0.400pt}}
\multiput(938.00,542.17)(4.472,2.000){2}{\rule{0.850pt}{0.400pt}}
\put(946,544.67){\rule{1.686pt}{0.400pt}}
\multiput(946.00,544.17)(3.500,1.000){2}{\rule{0.843pt}{0.400pt}}
\put(953,545.67){\rule{1.686pt}{0.400pt}}
\multiput(953.00,545.17)(3.500,1.000){2}{\rule{0.843pt}{0.400pt}}
\put(960,547.17){\rule{1.500pt}{0.400pt}}
\multiput(960.00,546.17)(3.887,2.000){2}{\rule{0.750pt}{0.400pt}}
\put(967,548.67){\rule{1.927pt}{0.400pt}}
\multiput(967.00,548.17)(4.000,1.000){2}{\rule{0.964pt}{0.400pt}}
\put(975,549.67){\rule{1.686pt}{0.400pt}}
\multiput(975.00,549.17)(3.500,1.000){2}{\rule{0.843pt}{0.400pt}}
\put(982,551.17){\rule{1.500pt}{0.400pt}}
\multiput(982.00,550.17)(3.887,2.000){2}{\rule{0.750pt}{0.400pt}}
\put(989,552.67){\rule{1.686pt}{0.400pt}}
\multiput(989.00,552.17)(3.500,1.000){2}{\rule{0.843pt}{0.400pt}}
\put(996,553.67){\rule{1.686pt}{0.400pt}}
\multiput(996.00,553.17)(3.500,1.000){2}{\rule{0.843pt}{0.400pt}}
\put(1003,554.67){\rule{1.927pt}{0.400pt}}
\multiput(1003.00,554.17)(4.000,1.000){2}{\rule{0.964pt}{0.400pt}}
\put(1011,556.17){\rule{1.500pt}{0.400pt}}
\multiput(1011.00,555.17)(3.887,2.000){2}{\rule{0.750pt}{0.400pt}}
\put(1018,557.67){\rule{1.686pt}{0.400pt}}
\multiput(1018.00,557.17)(3.500,1.000){2}{\rule{0.843pt}{0.400pt}}
\put(1025,558.67){\rule{1.686pt}{0.400pt}}
\multiput(1025.00,558.17)(3.500,1.000){2}{\rule{0.843pt}{0.400pt}}
\put(1032,560.17){\rule{1.500pt}{0.400pt}}
\multiput(1032.00,559.17)(3.887,2.000){2}{\rule{0.750pt}{0.400pt}}
\put(1039,561.67){\rule{1.927pt}{0.400pt}}
\multiput(1039.00,561.17)(4.000,1.000){2}{\rule{0.964pt}{0.400pt}}
\put(1047,562.67){\rule{1.686pt}{0.400pt}}
\multiput(1047.00,562.17)(3.500,1.000){2}{\rule{0.843pt}{0.400pt}}
\put(1054,564.17){\rule{1.500pt}{0.400pt}}
\multiput(1054.00,563.17)(3.887,2.000){2}{\rule{0.750pt}{0.400pt}}
\put(1061,565.67){\rule{1.686pt}{0.400pt}}
\multiput(1061.00,565.17)(3.500,1.000){2}{\rule{0.843pt}{0.400pt}}
\put(1068,566.67){\rule{1.927pt}{0.400pt}}
\multiput(1068.00,566.17)(4.000,1.000){2}{\rule{0.964pt}{0.400pt}}
\put(1076,568.17){\rule{1.500pt}{0.400pt}}
\multiput(1076.00,567.17)(3.887,2.000){2}{\rule{0.750pt}{0.400pt}}
\put(1083,569.67){\rule{1.686pt}{0.400pt}}
\multiput(1083.00,569.17)(3.500,1.000){2}{\rule{0.843pt}{0.400pt}}
\put(1090,570.67){\rule{1.686pt}{0.400pt}}
\multiput(1090.00,570.17)(3.500,1.000){2}{\rule{0.843pt}{0.400pt}}
\put(1097,571.67){\rule{1.686pt}{0.400pt}}
\multiput(1097.00,571.17)(3.500,1.000){2}{\rule{0.843pt}{0.400pt}}
\put(1104,573.17){\rule{1.700pt}{0.400pt}}
\multiput(1104.00,572.17)(4.472,2.000){2}{\rule{0.850pt}{0.400pt}}
\put(1112,574.67){\rule{1.686pt}{0.400pt}}
\multiput(1112.00,574.17)(3.500,1.000){2}{\rule{0.843pt}{0.400pt}}
\put(1119,575.67){\rule{1.686pt}{0.400pt}}
\multiput(1119.00,575.17)(3.500,1.000){2}{\rule{0.843pt}{0.400pt}}
\put(1126,577.17){\rule{1.500pt}{0.400pt}}
\multiput(1126.00,576.17)(3.887,2.000){2}{\rule{0.750pt}{0.400pt}}
\put(1133,578.67){\rule{1.686pt}{0.400pt}}
\multiput(1133.00,578.17)(3.500,1.000){2}{\rule{0.843pt}{0.400pt}}
\put(1140,579.67){\rule{1.927pt}{0.400pt}}
\multiput(1140.00,579.17)(4.000,1.000){2}{\rule{0.964pt}{0.400pt}}
\put(1148,581.17){\rule{1.500pt}{0.400pt}}
\multiput(1148.00,580.17)(3.887,2.000){2}{\rule{0.750pt}{0.400pt}}
\put(1155,582.67){\rule{1.686pt}{0.400pt}}
\multiput(1155.00,582.17)(3.500,1.000){2}{\rule{0.843pt}{0.400pt}}
\put(393,414.67){\rule{1.927pt}{0.400pt}}
\multiput(393.00,414.17)(4.000,1.000){2}{\rule{0.964pt}{0.400pt}}
\put(401,416.17){\rule{1.500pt}{0.400pt}}
\multiput(401.00,415.17)(3.887,2.000){2}{\rule{0.750pt}{0.400pt}}
\put(408,417.67){\rule{1.686pt}{0.400pt}}
\multiput(408.00,417.17)(3.500,1.000){2}{\rule{0.843pt}{0.400pt}}
\put(415,418.67){\rule{1.686pt}{0.400pt}}
\multiput(415.00,418.17)(3.500,1.000){2}{\rule{0.843pt}{0.400pt}}
\put(422,420.17){\rule{1.500pt}{0.400pt}}
\multiput(422.00,419.17)(3.887,2.000){2}{\rule{0.750pt}{0.400pt}}
\put(429,421.67){\rule{1.927pt}{0.400pt}}
\multiput(429.00,421.17)(4.000,1.000){2}{\rule{0.964pt}{0.400pt}}
\put(437,422.67){\rule{1.686pt}{0.400pt}}
\multiput(437.00,422.17)(3.500,1.000){2}{\rule{0.843pt}{0.400pt}}
\put(444,423.67){\rule{1.686pt}{0.400pt}}
\multiput(444.00,423.17)(3.500,1.000){2}{\rule{0.843pt}{0.400pt}}
\put(451,425.17){\rule{1.500pt}{0.400pt}}
\multiput(451.00,424.17)(3.887,2.000){2}{\rule{0.750pt}{0.400pt}}
\put(458,426.67){\rule{1.927pt}{0.400pt}}
\multiput(458.00,426.17)(4.000,1.000){2}{\rule{0.964pt}{0.400pt}}
\put(466,427.67){\rule{1.686pt}{0.400pt}}
\multiput(466.00,427.17)(3.500,1.000){2}{\rule{0.843pt}{0.400pt}}
\put(473,429.17){\rule{1.500pt}{0.400pt}}
\multiput(473.00,428.17)(3.887,2.000){2}{\rule{0.750pt}{0.400pt}}
\put(480,430.67){\rule{1.686pt}{0.400pt}}
\multiput(480.00,430.17)(3.500,1.000){2}{\rule{0.843pt}{0.400pt}}
\put(487,431.67){\rule{1.686pt}{0.400pt}}
\multiput(487.00,431.17)(3.500,1.000){2}{\rule{0.843pt}{0.400pt}}
\put(494,433.17){\rule{1.700pt}{0.400pt}}
\multiput(494.00,432.17)(4.472,2.000){2}{\rule{0.850pt}{0.400pt}}
\put(502,434.67){\rule{1.686pt}{0.400pt}}
\multiput(502.00,434.17)(3.500,1.000){2}{\rule{0.843pt}{0.400pt}}
\put(509,435.67){\rule{1.686pt}{0.400pt}}
\multiput(509.00,435.17)(3.500,1.000){2}{\rule{0.843pt}{0.400pt}}
\put(516,436.67){\rule{1.686pt}{0.400pt}}
\multiput(516.00,436.17)(3.500,1.000){2}{\rule{0.843pt}{0.400pt}}
\put(523,438.17){\rule{1.500pt}{0.400pt}}
\multiput(523.00,437.17)(3.887,2.000){2}{\rule{0.750pt}{0.400pt}}
\put(530,439.67){\rule{1.927pt}{0.400pt}}
\multiput(530.00,439.17)(4.000,1.000){2}{\rule{0.964pt}{0.400pt}}
\put(538,440.67){\rule{1.686pt}{0.400pt}}
\multiput(538.00,440.17)(3.500,1.000){2}{\rule{0.843pt}{0.400pt}}
\put(545,442.17){\rule{1.500pt}{0.400pt}}
\multiput(545.00,441.17)(3.887,2.000){2}{\rule{0.750pt}{0.400pt}}
\put(552,443.67){\rule{1.686pt}{0.400pt}}
\multiput(552.00,443.17)(3.500,1.000){2}{\rule{0.843pt}{0.400pt}}
\put(559,444.67){\rule{1.927pt}{0.400pt}}
\multiput(559.00,444.17)(4.000,1.000){2}{\rule{0.964pt}{0.400pt}}
\put(567,446.17){\rule{1.500pt}{0.400pt}}
\multiput(567.00,445.17)(3.887,2.000){2}{\rule{0.750pt}{0.400pt}}
\put(574,447.67){\rule{1.686pt}{0.400pt}}
\multiput(574.00,447.17)(3.500,1.000){2}{\rule{0.843pt}{0.400pt}}
\put(581,448.67){\rule{1.686pt}{0.400pt}}
\multiput(581.00,448.17)(3.500,1.000){2}{\rule{0.843pt}{0.400pt}}
\put(588,450.17){\rule{1.500pt}{0.400pt}}
\multiput(588.00,449.17)(3.887,2.000){2}{\rule{0.750pt}{0.400pt}}
\put(595,451.67){\rule{1.927pt}{0.400pt}}
\multiput(595.00,451.17)(4.000,1.000){2}{\rule{0.964pt}{0.400pt}}
\put(603,452.67){\rule{1.686pt}{0.400pt}}
\multiput(603.00,452.17)(3.500,1.000){2}{\rule{0.843pt}{0.400pt}}
\put(610,453.67){\rule{1.686pt}{0.400pt}}
\multiput(610.00,453.17)(3.500,1.000){2}{\rule{0.843pt}{0.400pt}}
\put(617,455.17){\rule{1.500pt}{0.400pt}}
\multiput(617.00,454.17)(3.887,2.000){2}{\rule{0.750pt}{0.400pt}}
\put(624,456.67){\rule{1.686pt}{0.400pt}}
\multiput(624.00,456.17)(3.500,1.000){2}{\rule{0.843pt}{0.400pt}}
\put(631,457.67){\rule{1.927pt}{0.400pt}}
\multiput(631.00,457.17)(4.000,1.000){2}{\rule{0.964pt}{0.400pt}}
\put(639,459.17){\rule{1.500pt}{0.400pt}}
\multiput(639.00,458.17)(3.887,2.000){2}{\rule{0.750pt}{0.400pt}}
\put(646,460.67){\rule{1.686pt}{0.400pt}}
\multiput(646.00,460.17)(3.500,1.000){2}{\rule{0.843pt}{0.400pt}}
\put(653,461.67){\rule{1.686pt}{0.400pt}}
\multiput(653.00,461.17)(3.500,1.000){2}{\rule{0.843pt}{0.400pt}}
\put(660,463.17){\rule{1.700pt}{0.400pt}}
\multiput(660.00,462.17)(4.472,2.000){2}{\rule{0.850pt}{0.400pt}}
\put(668,464.67){\rule{1.686pt}{0.400pt}}
\multiput(668.00,464.17)(3.500,1.000){2}{\rule{0.843pt}{0.400pt}}
\put(675,465.67){\rule{1.686pt}{0.400pt}}
\multiput(675.00,465.17)(3.500,1.000){2}{\rule{0.843pt}{0.400pt}}
\put(682,466.67){\rule{1.686pt}{0.400pt}}
\multiput(682.00,466.17)(3.500,1.000){2}{\rule{0.843pt}{0.400pt}}
\put(689,468.17){\rule{1.500pt}{0.400pt}}
\multiput(689.00,467.17)(3.887,2.000){2}{\rule{0.750pt}{0.400pt}}
\put(528.0,470.0){\rule[-0.200pt]{1.686pt}{0.400pt}}
\put(704,469.67){\rule{1.686pt}{0.400pt}}
\multiput(704.00,469.17)(3.500,1.000){2}{\rule{0.843pt}{0.400pt}}
\put(711,471.17){\rule{1.500pt}{0.400pt}}
\multiput(711.00,470.17)(3.887,2.000){2}{\rule{0.750pt}{0.400pt}}
\put(718,472.67){\rule{1.686pt}{0.400pt}}
\multiput(718.00,472.17)(3.500,1.000){2}{\rule{0.843pt}{0.400pt}}
\put(725,473.67){\rule{1.686pt}{0.400pt}}
\multiput(725.00,473.17)(3.500,1.000){2}{\rule{0.843pt}{0.400pt}}
\put(732,475.17){\rule{1.700pt}{0.400pt}}
\multiput(732.00,474.17)(4.472,2.000){2}{\rule{0.850pt}{0.400pt}}
\put(740,476.67){\rule{1.686pt}{0.400pt}}
\multiput(740.00,476.17)(3.500,1.000){2}{\rule{0.843pt}{0.400pt}}
\put(747,477.67){\rule{1.445pt}{0.400pt}}
\multiput(747.00,477.17)(3.000,1.000){2}{\rule{0.723pt}{0.400pt}}
\put(753,479.17){\rule{1.500pt}{0.400pt}}
\multiput(753.00,478.17)(3.887,2.000){2}{\rule{0.750pt}{0.400pt}}
\put(760,480.67){\rule{1.927pt}{0.400pt}}
\multiput(760.00,480.17)(4.000,1.000){2}{\rule{0.964pt}{0.400pt}}
\put(768,481.67){\rule{1.686pt}{0.400pt}}
\multiput(768.00,481.17)(3.500,1.000){2}{\rule{0.843pt}{0.400pt}}
\put(775,482.67){\rule{1.686pt}{0.400pt}}
\multiput(775.00,482.17)(3.500,1.000){2}{\rule{0.843pt}{0.400pt}}
\put(782,484.17){\rule{1.500pt}{0.400pt}}
\multiput(782.00,483.17)(3.887,2.000){2}{\rule{0.750pt}{0.400pt}}
\put(789,485.67){\rule{1.686pt}{0.400pt}}
\multiput(789.00,485.17)(3.500,1.000){2}{\rule{0.843pt}{0.400pt}}
\put(796,486.67){\rule{1.927pt}{0.400pt}}
\multiput(796.00,486.17)(4.000,1.000){2}{\rule{0.964pt}{0.400pt}}
\put(804,488.17){\rule{1.500pt}{0.400pt}}
\multiput(804.00,487.17)(3.887,2.000){2}{\rule{0.750pt}{0.400pt}}
\put(811,489.67){\rule{1.686pt}{0.400pt}}
\multiput(811.00,489.17)(3.500,1.000){2}{\rule{0.843pt}{0.400pt}}
\put(818,490.67){\rule{1.686pt}{0.400pt}}
\multiput(818.00,490.17)(3.500,1.000){2}{\rule{0.843pt}{0.400pt}}
\put(825,492.17){\rule{1.500pt}{0.400pt}}
\multiput(825.00,491.17)(3.887,2.000){2}{\rule{0.750pt}{0.400pt}}
\put(832,493.67){\rule{1.927pt}{0.400pt}}
\multiput(832.00,493.17)(4.000,1.000){2}{\rule{0.964pt}{0.400pt}}
\put(840,494.67){\rule{1.686pt}{0.400pt}}
\multiput(840.00,494.17)(3.500,1.000){2}{\rule{0.843pt}{0.400pt}}
\put(847,495.67){\rule{1.686pt}{0.400pt}}
\multiput(847.00,495.17)(3.500,1.000){2}{\rule{0.843pt}{0.400pt}}
\put(854,497.17){\rule{1.500pt}{0.400pt}}
\multiput(854.00,496.17)(3.887,2.000){2}{\rule{0.750pt}{0.400pt}}
\put(861,498.67){\rule{1.927pt}{0.400pt}}
\multiput(861.00,498.17)(4.000,1.000){2}{\rule{0.964pt}{0.400pt}}
\put(869,499.67){\rule{1.686pt}{0.400pt}}
\multiput(869.00,499.17)(3.500,1.000){2}{\rule{0.843pt}{0.400pt}}
\put(876,501.17){\rule{1.500pt}{0.400pt}}
\multiput(876.00,500.17)(3.887,2.000){2}{\rule{0.750pt}{0.400pt}}
\put(883,502.67){\rule{1.686pt}{0.400pt}}
\multiput(883.00,502.17)(3.500,1.000){2}{\rule{0.843pt}{0.400pt}}
\put(890,503.67){\rule{1.686pt}{0.400pt}}
\multiput(890.00,503.17)(3.500,1.000){2}{\rule{0.843pt}{0.400pt}}
\put(897,505.17){\rule{1.700pt}{0.400pt}}
\multiput(897.00,504.17)(4.472,2.000){2}{\rule{0.850pt}{0.400pt}}
\put(905,506.67){\rule{1.686pt}{0.400pt}}
\multiput(905.00,506.17)(3.500,1.000){2}{\rule{0.843pt}{0.400pt}}
\put(912,507.67){\rule{1.686pt}{0.400pt}}
\multiput(912.00,507.17)(3.500,1.000){2}{\rule{0.843pt}{0.400pt}}
\put(919,509.17){\rule{1.500pt}{0.400pt}}
\multiput(919.00,508.17)(3.887,2.000){2}{\rule{0.750pt}{0.400pt}}
\put(926,510.67){\rule{1.686pt}{0.400pt}}
\multiput(926.00,510.17)(3.500,1.000){2}{\rule{0.843pt}{0.400pt}}
\put(933,511.67){\rule{1.927pt}{0.400pt}}
\multiput(933.00,511.17)(4.000,1.000){2}{\rule{0.964pt}{0.400pt}}
\put(941,512.67){\rule{1.686pt}{0.400pt}}
\multiput(941.00,512.17)(3.500,1.000){2}{\rule{0.843pt}{0.400pt}}
\put(948,514.17){\rule{1.500pt}{0.400pt}}
\multiput(948.00,513.17)(3.887,2.000){2}{\rule{0.750pt}{0.400pt}}
\put(955,515.67){\rule{1.686pt}{0.400pt}}
\multiput(955.00,515.17)(3.500,1.000){2}{\rule{0.843pt}{0.400pt}}
\put(962,516.67){\rule{1.927pt}{0.400pt}}
\multiput(962.00,516.17)(4.000,1.000){2}{\rule{0.964pt}{0.400pt}}
\put(970,518.17){\rule{1.500pt}{0.400pt}}
\multiput(970.00,517.17)(3.887,2.000){2}{\rule{0.750pt}{0.400pt}}
\put(977,519.67){\rule{1.686pt}{0.400pt}}
\multiput(977.00,519.17)(3.500,1.000){2}{\rule{0.843pt}{0.400pt}}
\put(984,520.67){\rule{1.686pt}{0.400pt}}
\multiput(984.00,520.17)(3.500,1.000){2}{\rule{0.843pt}{0.400pt}}
\put(991,522.17){\rule{1.500pt}{0.400pt}}
\multiput(991.00,521.17)(3.887,2.000){2}{\rule{0.750pt}{0.400pt}}
\put(998,523.67){\rule{1.927pt}{0.400pt}}
\multiput(998.00,523.17)(4.000,1.000){2}{\rule{0.964pt}{0.400pt}}
\put(1006,524.67){\rule{1.686pt}{0.400pt}}
\multiput(1006.00,524.17)(3.500,1.000){2}{\rule{0.843pt}{0.400pt}}
\put(1013,526.17){\rule{1.500pt}{0.400pt}}
\multiput(1013.00,525.17)(3.887,2.000){2}{\rule{0.750pt}{0.400pt}}
\put(1020,527.67){\rule{1.686pt}{0.400pt}}
\multiput(1020.00,527.17)(3.500,1.000){2}{\rule{0.843pt}{0.400pt}}
\put(1027,528.67){\rule{1.686pt}{0.400pt}}
\multiput(1027.00,528.17)(3.500,1.000){2}{\rule{0.843pt}{0.400pt}}
\put(1034,529.67){\rule{1.927pt}{0.400pt}}
\multiput(1034.00,529.17)(4.000,1.000){2}{\rule{0.964pt}{0.400pt}}
\put(1042,531.17){\rule{1.500pt}{0.400pt}}
\multiput(1042.00,530.17)(3.887,2.000){2}{\rule{0.750pt}{0.400pt}}
\put(1049,532.67){\rule{1.686pt}{0.400pt}}
\multiput(1049.00,532.17)(3.500,1.000){2}{\rule{0.843pt}{0.400pt}}
\put(1056,533.67){\rule{1.686pt}{0.400pt}}
\multiput(1056.00,533.17)(3.500,1.000){2}{\rule{0.843pt}{0.400pt}}
\put(1063,535.17){\rule{1.700pt}{0.400pt}}
\multiput(1063.00,534.17)(4.472,2.000){2}{\rule{0.850pt}{0.400pt}}
\put(1071,536.67){\rule{1.686pt}{0.400pt}}
\multiput(1071.00,536.17)(3.500,1.000){2}{\rule{0.843pt}{0.400pt}}
\put(1078,537.67){\rule{1.686pt}{0.400pt}}
\multiput(1078.00,537.17)(3.500,1.000){2}{\rule{0.843pt}{0.400pt}}
\put(1085,539.17){\rule{1.500pt}{0.400pt}}
\multiput(1085.00,538.17)(3.887,2.000){2}{\rule{0.750pt}{0.400pt}}
\put(1092,540.67){\rule{1.686pt}{0.400pt}}
\multiput(1092.00,540.17)(3.500,1.000){2}{\rule{0.843pt}{0.400pt}}
\put(1099,541.67){\rule{1.927pt}{0.400pt}}
\multiput(1099.00,541.17)(4.000,1.000){2}{\rule{0.964pt}{0.400pt}}
\put(338,374.17){\rule{1.500pt}{0.400pt}}
\multiput(338.00,373.17)(3.887,2.000){2}{\rule{0.750pt}{0.400pt}}
\put(345,375.67){\rule{1.686pt}{0.400pt}}
\multiput(345.00,375.17)(3.500,1.000){2}{\rule{0.843pt}{0.400pt}}
\put(352,376.67){\rule{1.927pt}{0.400pt}}
\multiput(352.00,376.17)(4.000,1.000){2}{\rule{0.964pt}{0.400pt}}
\put(360,377.67){\rule{1.686pt}{0.400pt}}
\multiput(360.00,377.17)(3.500,1.000){2}{\rule{0.843pt}{0.400pt}}
\put(367,379.17){\rule{1.500pt}{0.400pt}}
\multiput(367.00,378.17)(3.887,2.000){2}{\rule{0.750pt}{0.400pt}}
\put(374,380.67){\rule{1.686pt}{0.400pt}}
\multiput(374.00,380.17)(3.500,1.000){2}{\rule{0.843pt}{0.400pt}}
\put(381,381.67){\rule{1.686pt}{0.400pt}}
\multiput(381.00,381.17)(3.500,1.000){2}{\rule{0.843pt}{0.400pt}}
\put(388,383.17){\rule{1.700pt}{0.400pt}}
\multiput(388.00,382.17)(4.472,2.000){2}{\rule{0.850pt}{0.400pt}}
\put(396,384.67){\rule{1.686pt}{0.400pt}}
\multiput(396.00,384.17)(3.500,1.000){2}{\rule{0.843pt}{0.400pt}}
\put(403,385.67){\rule{1.686pt}{0.400pt}}
\multiput(403.00,385.17)(3.500,1.000){2}{\rule{0.843pt}{0.400pt}}
\put(410,387.17){\rule{1.500pt}{0.400pt}}
\multiput(410.00,386.17)(3.887,2.000){2}{\rule{0.750pt}{0.400pt}}
\put(417,388.67){\rule{1.686pt}{0.400pt}}
\multiput(417.00,388.17)(3.500,1.000){2}{\rule{0.843pt}{0.400pt}}
\put(424,389.67){\rule{1.927pt}{0.400pt}}
\multiput(424.00,389.17)(4.000,1.000){2}{\rule{0.964pt}{0.400pt}}
\put(432,391.17){\rule{1.500pt}{0.400pt}}
\multiput(432.00,390.17)(3.887,2.000){2}{\rule{0.750pt}{0.400pt}}
\put(439,392.67){\rule{1.686pt}{0.400pt}}
\multiput(439.00,392.17)(3.500,1.000){2}{\rule{0.843pt}{0.400pt}}
\put(446,393.67){\rule{1.686pt}{0.400pt}}
\multiput(446.00,393.17)(3.500,1.000){2}{\rule{0.843pt}{0.400pt}}
\put(453,394.67){\rule{1.927pt}{0.400pt}}
\multiput(453.00,394.17)(4.000,1.000){2}{\rule{0.964pt}{0.400pt}}
\put(461,396.17){\rule{1.500pt}{0.400pt}}
\multiput(461.00,395.17)(3.887,2.000){2}{\rule{0.750pt}{0.400pt}}
\put(468,397.67){\rule{1.686pt}{0.400pt}}
\multiput(468.00,397.17)(3.500,1.000){2}{\rule{0.843pt}{0.400pt}}
\put(475,398.67){\rule{1.686pt}{0.400pt}}
\multiput(475.00,398.17)(3.500,1.000){2}{\rule{0.843pt}{0.400pt}}
\put(482,400.17){\rule{1.500pt}{0.400pt}}
\multiput(482.00,399.17)(3.887,2.000){2}{\rule{0.750pt}{0.400pt}}
\put(489,401.67){\rule{1.927pt}{0.400pt}}
\multiput(489.00,401.17)(4.000,1.000){2}{\rule{0.964pt}{0.400pt}}
\put(497,402.67){\rule{1.686pt}{0.400pt}}
\multiput(497.00,402.17)(3.500,1.000){2}{\rule{0.843pt}{0.400pt}}
\put(504,404.17){\rule{1.500pt}{0.400pt}}
\multiput(504.00,403.17)(3.887,2.000){2}{\rule{0.750pt}{0.400pt}}
\put(511,405.67){\rule{1.686pt}{0.400pt}}
\multiput(511.00,405.17)(3.500,1.000){2}{\rule{0.843pt}{0.400pt}}
\put(518,406.67){\rule{1.686pt}{0.400pt}}
\multiput(518.00,406.17)(3.500,1.000){2}{\rule{0.843pt}{0.400pt}}
\put(525,407.67){\rule{1.927pt}{0.400pt}}
\multiput(525.00,407.17)(4.000,1.000){2}{\rule{0.964pt}{0.400pt}}
\put(533,409.17){\rule{1.500pt}{0.400pt}}
\multiput(533.00,408.17)(3.887,2.000){2}{\rule{0.750pt}{0.400pt}}
\put(540,410.67){\rule{1.686pt}{0.400pt}}
\multiput(540.00,410.17)(3.500,1.000){2}{\rule{0.843pt}{0.400pt}}
\put(547,411.67){\rule{1.686pt}{0.400pt}}
\multiput(547.00,411.17)(3.500,1.000){2}{\rule{0.843pt}{0.400pt}}
\put(554,413.17){\rule{1.700pt}{0.400pt}}
\multiput(554.00,412.17)(4.472,2.000){2}{\rule{0.850pt}{0.400pt}}
\put(562,414.67){\rule{1.686pt}{0.400pt}}
\multiput(562.00,414.17)(3.500,1.000){2}{\rule{0.843pt}{0.400pt}}
\put(569,415.67){\rule{1.686pt}{0.400pt}}
\multiput(569.00,415.17)(3.500,1.000){2}{\rule{0.843pt}{0.400pt}}
\put(576,417.17){\rule{1.500pt}{0.400pt}}
\multiput(576.00,416.17)(3.887,2.000){2}{\rule{0.750pt}{0.400pt}}
\put(583,418.67){\rule{1.686pt}{0.400pt}}
\multiput(583.00,418.17)(3.500,1.000){2}{\rule{0.843pt}{0.400pt}}
\put(590,419.67){\rule{1.927pt}{0.400pt}}
\multiput(590.00,419.17)(4.000,1.000){2}{\rule{0.964pt}{0.400pt}}
\put(598,421.17){\rule{1.500pt}{0.400pt}}
\multiput(598.00,420.17)(3.887,2.000){2}{\rule{0.750pt}{0.400pt}}
\put(605,422.67){\rule{1.686pt}{0.400pt}}
\multiput(605.00,422.17)(3.500,1.000){2}{\rule{0.843pt}{0.400pt}}
\put(612,423.67){\rule{1.686pt}{0.400pt}}
\multiput(612.00,423.17)(3.500,1.000){2}{\rule{0.843pt}{0.400pt}}
\put(619,424.67){\rule{1.686pt}{0.400pt}}
\multiput(619.00,424.17)(3.500,1.000){2}{\rule{0.843pt}{0.400pt}}
\put(626,426.17){\rule{1.700pt}{0.400pt}}
\multiput(626.00,425.17)(4.472,2.000){2}{\rule{0.850pt}{0.400pt}}
\put(634,427.67){\rule{1.686pt}{0.400pt}}
\multiput(634.00,427.17)(3.500,1.000){2}{\rule{0.843pt}{0.400pt}}
\put(641,428.67){\rule{1.686pt}{0.400pt}}
\multiput(641.00,428.17)(3.500,1.000){2}{\rule{0.843pt}{0.400pt}}
\put(648,430.17){\rule{1.500pt}{0.400pt}}
\multiput(648.00,429.17)(3.887,2.000){2}{\rule{0.750pt}{0.400pt}}
\put(655,431.67){\rule{1.927pt}{0.400pt}}
\multiput(655.00,431.17)(4.000,1.000){2}{\rule{0.964pt}{0.400pt}}
\put(663,432.67){\rule{1.686pt}{0.400pt}}
\multiput(663.00,432.17)(3.500,1.000){2}{\rule{0.843pt}{0.400pt}}
\put(670,434.17){\rule{1.500pt}{0.400pt}}
\multiput(670.00,433.17)(3.887,2.000){2}{\rule{0.750pt}{0.400pt}}
\put(677,435.67){\rule{1.686pt}{0.400pt}}
\multiput(677.00,435.17)(3.500,1.000){2}{\rule{0.843pt}{0.400pt}}
\put(684,436.67){\rule{1.686pt}{0.400pt}}
\multiput(684.00,436.17)(3.500,1.000){2}{\rule{0.843pt}{0.400pt}}
\put(691,438.17){\rule{1.700pt}{0.400pt}}
\multiput(691.00,437.17)(4.472,2.000){2}{\rule{0.850pt}{0.400pt}}
\put(699,439.67){\rule{1.686pt}{0.400pt}}
\multiput(699.00,439.17)(3.500,1.000){2}{\rule{0.843pt}{0.400pt}}
\put(706,440.67){\rule{1.686pt}{0.400pt}}
\multiput(706.00,440.17)(3.500,1.000){2}{\rule{0.843pt}{0.400pt}}
\put(713,441.67){\rule{1.686pt}{0.400pt}}
\multiput(713.00,441.17)(3.500,1.000){2}{\rule{0.843pt}{0.400pt}}
\put(720,443.17){\rule{1.500pt}{0.400pt}}
\multiput(720.00,442.17)(3.887,2.000){2}{\rule{0.750pt}{0.400pt}}
\put(727,444.67){\rule{1.927pt}{0.400pt}}
\multiput(727.00,444.17)(4.000,1.000){2}{\rule{0.964pt}{0.400pt}}
\put(735,445.67){\rule{1.686pt}{0.400pt}}
\multiput(735.00,445.17)(3.500,1.000){2}{\rule{0.843pt}{0.400pt}}
\put(742,447.17){\rule{1.500pt}{0.400pt}}
\multiput(742.00,446.17)(3.887,2.000){2}{\rule{0.750pt}{0.400pt}}
\put(749,448.67){\rule{1.445pt}{0.400pt}}
\multiput(749.00,448.17)(3.000,1.000){2}{\rule{0.723pt}{0.400pt}}
\put(755,449.67){\rule{1.927pt}{0.400pt}}
\multiput(755.00,449.17)(4.000,1.000){2}{\rule{0.964pt}{0.400pt}}
\put(763,451.17){\rule{1.500pt}{0.400pt}}
\multiput(763.00,450.17)(3.887,2.000){2}{\rule{0.750pt}{0.400pt}}
\put(770,452.67){\rule{1.686pt}{0.400pt}}
\multiput(770.00,452.17)(3.500,1.000){2}{\rule{0.843pt}{0.400pt}}
\put(777,453.67){\rule{1.686pt}{0.400pt}}
\multiput(777.00,453.17)(3.500,1.000){2}{\rule{0.843pt}{0.400pt}}
\put(784,454.67){\rule{1.686pt}{0.400pt}}
\multiput(784.00,454.17)(3.500,1.000){2}{\rule{0.843pt}{0.400pt}}
\put(791,456.17){\rule{1.700pt}{0.400pt}}
\multiput(791.00,455.17)(4.472,2.000){2}{\rule{0.850pt}{0.400pt}}
\put(799,457.67){\rule{1.686pt}{0.400pt}}
\multiput(799.00,457.17)(3.500,1.000){2}{\rule{0.843pt}{0.400pt}}
\put(806,458.67){\rule{1.686pt}{0.400pt}}
\multiput(806.00,458.17)(3.500,1.000){2}{\rule{0.843pt}{0.400pt}}
\put(813,460.17){\rule{1.500pt}{0.400pt}}
\multiput(813.00,459.17)(3.887,2.000){2}{\rule{0.750pt}{0.400pt}}
\put(820,461.67){\rule{1.686pt}{0.400pt}}
\multiput(820.00,461.17)(3.500,1.000){2}{\rule{0.843pt}{0.400pt}}
\put(827,462.67){\rule{1.927pt}{0.400pt}}
\multiput(827.00,462.17)(4.000,1.000){2}{\rule{0.964pt}{0.400pt}}
\put(835,464.17){\rule{1.500pt}{0.400pt}}
\multiput(835.00,463.17)(3.887,2.000){2}{\rule{0.750pt}{0.400pt}}
\put(842,465.67){\rule{1.686pt}{0.400pt}}
\multiput(842.00,465.17)(3.500,1.000){2}{\rule{0.843pt}{0.400pt}}
\put(849,466.67){\rule{1.686pt}{0.400pt}}
\multiput(849.00,466.17)(3.500,1.000){2}{\rule{0.843pt}{0.400pt}}
\put(856,468.17){\rule{1.700pt}{0.400pt}}
\multiput(856.00,467.17)(4.472,2.000){2}{\rule{0.850pt}{0.400pt}}
\put(696.0,470.0){\rule[-0.200pt]{1.927pt}{0.400pt}}
\put(871,469.67){\rule{1.686pt}{0.400pt}}
\multiput(871.00,469.17)(3.500,1.000){2}{\rule{0.843pt}{0.400pt}}
\put(878,470.67){\rule{1.686pt}{0.400pt}}
\multiput(878.00,470.17)(3.500,1.000){2}{\rule{0.843pt}{0.400pt}}
\put(885,472.17){\rule{1.500pt}{0.400pt}}
\multiput(885.00,471.17)(3.887,2.000){2}{\rule{0.750pt}{0.400pt}}
\put(892,473.67){\rule{1.927pt}{0.400pt}}
\multiput(892.00,473.17)(4.000,1.000){2}{\rule{0.964pt}{0.400pt}}
\put(900,474.67){\rule{1.686pt}{0.400pt}}
\multiput(900.00,474.17)(3.500,1.000){2}{\rule{0.843pt}{0.400pt}}
\put(907,476.17){\rule{1.500pt}{0.400pt}}
\multiput(907.00,475.17)(3.887,2.000){2}{\rule{0.750pt}{0.400pt}}
\put(914,477.67){\rule{1.686pt}{0.400pt}}
\multiput(914.00,477.17)(3.500,1.000){2}{\rule{0.843pt}{0.400pt}}
\put(921,478.67){\rule{1.686pt}{0.400pt}}
\multiput(921.00,478.17)(3.500,1.000){2}{\rule{0.843pt}{0.400pt}}
\put(928,480.17){\rule{1.700pt}{0.400pt}}
\multiput(928.00,479.17)(4.472,2.000){2}{\rule{0.850pt}{0.400pt}}
\put(936,481.67){\rule{1.686pt}{0.400pt}}
\multiput(936.00,481.17)(3.500,1.000){2}{\rule{0.843pt}{0.400pt}}
\put(943,482.67){\rule{1.686pt}{0.400pt}}
\multiput(943.00,482.17)(3.500,1.000){2}{\rule{0.843pt}{0.400pt}}
\put(950,483.67){\rule{1.686pt}{0.400pt}}
\multiput(950.00,483.17)(3.500,1.000){2}{\rule{0.843pt}{0.400pt}}
\put(957,485.17){\rule{1.700pt}{0.400pt}}
\multiput(957.00,484.17)(4.472,2.000){2}{\rule{0.850pt}{0.400pt}}
\put(965,486.67){\rule{1.686pt}{0.400pt}}
\multiput(965.00,486.17)(3.500,1.000){2}{\rule{0.843pt}{0.400pt}}
\put(972,487.67){\rule{1.686pt}{0.400pt}}
\multiput(972.00,487.17)(3.500,1.000){2}{\rule{0.843pt}{0.400pt}}
\put(979,489.17){\rule{1.500pt}{0.400pt}}
\multiput(979.00,488.17)(3.887,2.000){2}{\rule{0.750pt}{0.400pt}}
\put(986,490.67){\rule{1.686pt}{0.400pt}}
\multiput(986.00,490.17)(3.500,1.000){2}{\rule{0.843pt}{0.400pt}}
\put(993,491.67){\rule{1.927pt}{0.400pt}}
\multiput(993.00,491.17)(4.000,1.000){2}{\rule{0.964pt}{0.400pt}}
\put(1001,493.17){\rule{1.500pt}{0.400pt}}
\multiput(1001.00,492.17)(3.887,2.000){2}{\rule{0.750pt}{0.400pt}}
\put(1008,494.67){\rule{1.686pt}{0.400pt}}
\multiput(1008.00,494.17)(3.500,1.000){2}{\rule{0.843pt}{0.400pt}}
\put(1015,495.67){\rule{1.686pt}{0.400pt}}
\multiput(1015.00,495.17)(3.500,1.000){2}{\rule{0.843pt}{0.400pt}}
\put(1022,497.17){\rule{1.500pt}{0.400pt}}
\multiput(1022.00,496.17)(3.887,2.000){2}{\rule{0.750pt}{0.400pt}}
\put(1029,498.67){\rule{1.927pt}{0.400pt}}
\multiput(1029.00,498.17)(4.000,1.000){2}{\rule{0.964pt}{0.400pt}}
\put(1037,499.67){\rule{1.686pt}{0.400pt}}
\multiput(1037.00,499.17)(3.500,1.000){2}{\rule{0.843pt}{0.400pt}}
\put(1044,500.67){\rule{1.686pt}{0.400pt}}
\multiput(1044.00,500.17)(3.500,1.000){2}{\rule{0.843pt}{0.400pt}}
\put(282,333.67){\rule{1.927pt}{0.400pt}}
\multiput(282.00,333.17)(4.000,1.000){2}{\rule{0.964pt}{0.400pt}}
\put(290,334.67){\rule{1.686pt}{0.400pt}}
\multiput(290.00,334.17)(3.500,1.000){2}{\rule{0.843pt}{0.400pt}}
\put(297,335.67){\rule{1.686pt}{0.400pt}}
\multiput(297.00,335.17)(3.500,1.000){2}{\rule{0.843pt}{0.400pt}}
\put(304,337.17){\rule{1.500pt}{0.400pt}}
\multiput(304.00,336.17)(3.887,2.000){2}{\rule{0.750pt}{0.400pt}}
\put(311,338.67){\rule{1.686pt}{0.400pt}}
\multiput(311.00,338.17)(3.500,1.000){2}{\rule{0.843pt}{0.400pt}}
\put(318,339.67){\rule{1.927pt}{0.400pt}}
\multiput(318.00,339.17)(4.000,1.000){2}{\rule{0.964pt}{0.400pt}}
\put(326,341.17){\rule{1.500pt}{0.400pt}}
\multiput(326.00,340.17)(3.887,2.000){2}{\rule{0.750pt}{0.400pt}}
\put(333,342.67){\rule{1.686pt}{0.400pt}}
\multiput(333.00,342.17)(3.500,1.000){2}{\rule{0.843pt}{0.400pt}}
\put(340,343.67){\rule{1.686pt}{0.400pt}}
\multiput(340.00,343.17)(3.500,1.000){2}{\rule{0.843pt}{0.400pt}}
\put(347,345.17){\rule{1.700pt}{0.400pt}}
\multiput(347.00,344.17)(4.472,2.000){2}{\rule{0.850pt}{0.400pt}}
\put(355,346.67){\rule{1.686pt}{0.400pt}}
\multiput(355.00,346.17)(3.500,1.000){2}{\rule{0.843pt}{0.400pt}}
\put(362,347.67){\rule{1.686pt}{0.400pt}}
\multiput(362.00,347.17)(3.500,1.000){2}{\rule{0.843pt}{0.400pt}}
\put(369,349.17){\rule{1.500pt}{0.400pt}}
\multiput(369.00,348.17)(3.887,2.000){2}{\rule{0.750pt}{0.400pt}}
\put(376,350.67){\rule{1.686pt}{0.400pt}}
\multiput(376.00,350.17)(3.500,1.000){2}{\rule{0.843pt}{0.400pt}}
\put(383,351.67){\rule{1.927pt}{0.400pt}}
\multiput(383.00,351.17)(4.000,1.000){2}{\rule{0.964pt}{0.400pt}}
\put(391,352.67){\rule{1.686pt}{0.400pt}}
\multiput(391.00,352.17)(3.500,1.000){2}{\rule{0.843pt}{0.400pt}}
\put(398,354.17){\rule{1.500pt}{0.400pt}}
\multiput(398.00,353.17)(3.887,2.000){2}{\rule{0.750pt}{0.400pt}}
\put(405,355.67){\rule{1.686pt}{0.400pt}}
\multiput(405.00,355.17)(3.500,1.000){2}{\rule{0.843pt}{0.400pt}}
\put(412,356.67){\rule{1.686pt}{0.400pt}}
\multiput(412.00,356.17)(3.500,1.000){2}{\rule{0.843pt}{0.400pt}}
\put(419,358.17){\rule{1.700pt}{0.400pt}}
\multiput(419.00,357.17)(4.472,2.000){2}{\rule{0.850pt}{0.400pt}}
\put(427,359.67){\rule{1.686pt}{0.400pt}}
\multiput(427.00,359.17)(3.500,1.000){2}{\rule{0.843pt}{0.400pt}}
\put(434,360.67){\rule{1.686pt}{0.400pt}}
\multiput(434.00,360.17)(3.500,1.000){2}{\rule{0.843pt}{0.400pt}}
\put(441,362.17){\rule{1.500pt}{0.400pt}}
\multiput(441.00,361.17)(3.887,2.000){2}{\rule{0.750pt}{0.400pt}}
\put(448,363.67){\rule{1.927pt}{0.400pt}}
\multiput(448.00,363.17)(4.000,1.000){2}{\rule{0.964pt}{0.400pt}}
\put(456,364.67){\rule{1.686pt}{0.400pt}}
\multiput(456.00,364.17)(3.500,1.000){2}{\rule{0.843pt}{0.400pt}}
\put(463,365.67){\rule{1.686pt}{0.400pt}}
\multiput(463.00,365.17)(3.500,1.000){2}{\rule{0.843pt}{0.400pt}}
\put(470,367.17){\rule{1.500pt}{0.400pt}}
\multiput(470.00,366.17)(3.887,2.000){2}{\rule{0.750pt}{0.400pt}}
\put(477,368.67){\rule{1.686pt}{0.400pt}}
\multiput(477.00,368.17)(3.500,1.000){2}{\rule{0.843pt}{0.400pt}}
\put(484,369.67){\rule{1.927pt}{0.400pt}}
\multiput(484.00,369.17)(4.000,1.000){2}{\rule{0.964pt}{0.400pt}}
\put(492,371.17){\rule{1.500pt}{0.400pt}}
\multiput(492.00,370.17)(3.887,2.000){2}{\rule{0.750pt}{0.400pt}}
\put(499,372.67){\rule{1.686pt}{0.400pt}}
\multiput(499.00,372.17)(3.500,1.000){2}{\rule{0.843pt}{0.400pt}}
\put(506,373.67){\rule{1.686pt}{0.400pt}}
\multiput(506.00,373.17)(3.500,1.000){2}{\rule{0.843pt}{0.400pt}}
\put(513,375.17){\rule{1.500pt}{0.400pt}}
\multiput(513.00,374.17)(3.887,2.000){2}{\rule{0.750pt}{0.400pt}}
\put(520,376.67){\rule{1.927pt}{0.400pt}}
\multiput(520.00,376.17)(4.000,1.000){2}{\rule{0.964pt}{0.400pt}}
\put(528,377.67){\rule{1.686pt}{0.400pt}}
\multiput(528.00,377.17)(3.500,1.000){2}{\rule{0.843pt}{0.400pt}}
\put(535,379.17){\rule{1.500pt}{0.400pt}}
\multiput(535.00,378.17)(3.887,2.000){2}{\rule{0.750pt}{0.400pt}}
\put(542,380.67){\rule{1.686pt}{0.400pt}}
\multiput(542.00,380.17)(3.500,1.000){2}{\rule{0.843pt}{0.400pt}}
\put(549,381.67){\rule{1.927pt}{0.400pt}}
\multiput(549.00,381.17)(4.000,1.000){2}{\rule{0.964pt}{0.400pt}}
\put(557,382.67){\rule{1.686pt}{0.400pt}}
\multiput(557.00,382.17)(3.500,1.000){2}{\rule{0.843pt}{0.400pt}}
\put(564,384.17){\rule{1.500pt}{0.400pt}}
\multiput(564.00,383.17)(3.887,2.000){2}{\rule{0.750pt}{0.400pt}}
\put(571,385.67){\rule{1.686pt}{0.400pt}}
\multiput(571.00,385.17)(3.500,1.000){2}{\rule{0.843pt}{0.400pt}}
\put(578,386.67){\rule{1.686pt}{0.400pt}}
\multiput(578.00,386.17)(3.500,1.000){2}{\rule{0.843pt}{0.400pt}}
\put(585,388.17){\rule{1.700pt}{0.400pt}}
\multiput(585.00,387.17)(4.472,2.000){2}{\rule{0.850pt}{0.400pt}}
\put(593,389.67){\rule{1.686pt}{0.400pt}}
\multiput(593.00,389.17)(3.500,1.000){2}{\rule{0.843pt}{0.400pt}}
\put(600,390.67){\rule{1.686pt}{0.400pt}}
\multiput(600.00,390.17)(3.500,1.000){2}{\rule{0.843pt}{0.400pt}}
\put(607,392.17){\rule{1.500pt}{0.400pt}}
\multiput(607.00,391.17)(3.887,2.000){2}{\rule{0.750pt}{0.400pt}}
\put(614,393.67){\rule{1.686pt}{0.400pt}}
\multiput(614.00,393.17)(3.500,1.000){2}{\rule{0.843pt}{0.400pt}}
\put(621,394.67){\rule{1.927pt}{0.400pt}}
\multiput(621.00,394.17)(4.000,1.000){2}{\rule{0.964pt}{0.400pt}}
\put(629,395.67){\rule{1.686pt}{0.400pt}}
\multiput(629.00,395.17)(3.500,1.000){2}{\rule{0.843pt}{0.400pt}}
\put(636,397.17){\rule{1.500pt}{0.400pt}}
\multiput(636.00,396.17)(3.887,2.000){2}{\rule{0.750pt}{0.400pt}}
\put(643,398.67){\rule{1.686pt}{0.400pt}}
\multiput(643.00,398.17)(3.500,1.000){2}{\rule{0.843pt}{0.400pt}}
\put(650,399.67){\rule{1.927pt}{0.400pt}}
\multiput(650.00,399.17)(4.000,1.000){2}{\rule{0.964pt}{0.400pt}}
\put(658,401.17){\rule{1.500pt}{0.400pt}}
\multiput(658.00,400.17)(3.887,2.000){2}{\rule{0.750pt}{0.400pt}}
\put(665,402.67){\rule{1.686pt}{0.400pt}}
\multiput(665.00,402.17)(3.500,1.000){2}{\rule{0.843pt}{0.400pt}}
\put(672,403.67){\rule{1.686pt}{0.400pt}}
\multiput(672.00,403.17)(3.500,1.000){2}{\rule{0.843pt}{0.400pt}}
\put(679,405.17){\rule{1.500pt}{0.400pt}}
\multiput(679.00,404.17)(3.887,2.000){2}{\rule{0.750pt}{0.400pt}}
\put(686,406.67){\rule{1.927pt}{0.400pt}}
\multiput(686.00,406.17)(4.000,1.000){2}{\rule{0.964pt}{0.400pt}}
\put(694,407.67){\rule{1.686pt}{0.400pt}}
\multiput(694.00,407.17)(3.500,1.000){2}{\rule{0.843pt}{0.400pt}}
\put(701,409.17){\rule{1.500pt}{0.400pt}}
\multiput(701.00,408.17)(3.887,2.000){2}{\rule{0.750pt}{0.400pt}}
\put(708,410.67){\rule{1.686pt}{0.400pt}}
\multiput(708.00,410.17)(3.500,1.000){2}{\rule{0.843pt}{0.400pt}}
\put(715,411.67){\rule{1.686pt}{0.400pt}}
\multiput(715.00,411.17)(3.500,1.000){2}{\rule{0.843pt}{0.400pt}}
\put(722,412.67){\rule{1.927pt}{0.400pt}}
\multiput(722.00,412.17)(4.000,1.000){2}{\rule{0.964pt}{0.400pt}}
\put(730,414.17){\rule{1.500pt}{0.400pt}}
\multiput(730.00,413.17)(3.887,2.000){2}{\rule{0.750pt}{0.400pt}}
\put(737,415.67){\rule{1.686pt}{0.400pt}}
\multiput(737.00,415.17)(3.500,1.000){2}{\rule{0.843pt}{0.400pt}}
\put(744,416.67){\rule{1.445pt}{0.400pt}}
\multiput(744.00,416.17)(3.000,1.000){2}{\rule{0.723pt}{0.400pt}}
\put(750,418.17){\rule{1.700pt}{0.400pt}}
\multiput(750.00,417.17)(4.472,2.000){2}{\rule{0.850pt}{0.400pt}}
\put(758,419.67){\rule{1.686pt}{0.400pt}}
\multiput(758.00,419.17)(3.500,1.000){2}{\rule{0.843pt}{0.400pt}}
\put(765,420.67){\rule{1.686pt}{0.400pt}}
\multiput(765.00,420.17)(3.500,1.000){2}{\rule{0.843pt}{0.400pt}}
\put(772,422.17){\rule{1.500pt}{0.400pt}}
\multiput(772.00,421.17)(3.887,2.000){2}{\rule{0.750pt}{0.400pt}}
\put(779,423.67){\rule{1.686pt}{0.400pt}}
\multiput(779.00,423.17)(3.500,1.000){2}{\rule{0.843pt}{0.400pt}}
\put(786,424.67){\rule{1.927pt}{0.400pt}}
\multiput(786.00,424.17)(4.000,1.000){2}{\rule{0.964pt}{0.400pt}}
\put(794,425.67){\rule{1.686pt}{0.400pt}}
\multiput(794.00,425.17)(3.500,1.000){2}{\rule{0.843pt}{0.400pt}}
\put(801,427.17){\rule{1.500pt}{0.400pt}}
\multiput(801.00,426.17)(3.887,2.000){2}{\rule{0.750pt}{0.400pt}}
\put(808,428.67){\rule{1.686pt}{0.400pt}}
\multiput(808.00,428.17)(3.500,1.000){2}{\rule{0.843pt}{0.400pt}}
\put(815,429.67){\rule{1.686pt}{0.400pt}}
\multiput(815.00,429.17)(3.500,1.000){2}{\rule{0.843pt}{0.400pt}}
\put(822,431.17){\rule{1.700pt}{0.400pt}}
\multiput(822.00,430.17)(4.472,2.000){2}{\rule{0.850pt}{0.400pt}}
\put(830,432.67){\rule{1.686pt}{0.400pt}}
\multiput(830.00,432.17)(3.500,1.000){2}{\rule{0.843pt}{0.400pt}}
\put(837,433.67){\rule{1.686pt}{0.400pt}}
\multiput(837.00,433.17)(3.500,1.000){2}{\rule{0.843pt}{0.400pt}}
\put(844,435.17){\rule{1.500pt}{0.400pt}}
\multiput(844.00,434.17)(3.887,2.000){2}{\rule{0.750pt}{0.400pt}}
\put(851,436.67){\rule{1.927pt}{0.400pt}}
\multiput(851.00,436.17)(4.000,1.000){2}{\rule{0.964pt}{0.400pt}}
\put(859,437.67){\rule{1.686pt}{0.400pt}}
\multiput(859.00,437.17)(3.500,1.000){2}{\rule{0.843pt}{0.400pt}}
\put(866,439.17){\rule{1.500pt}{0.400pt}}
\multiput(866.00,438.17)(3.887,2.000){2}{\rule{0.750pt}{0.400pt}}
\put(873,440.67){\rule{1.686pt}{0.400pt}}
\multiput(873.00,440.17)(3.500,1.000){2}{\rule{0.843pt}{0.400pt}}
\put(880,441.67){\rule{1.686pt}{0.400pt}}
\multiput(880.00,441.17)(3.500,1.000){2}{\rule{0.843pt}{0.400pt}}
\put(887,442.67){\rule{1.927pt}{0.400pt}}
\multiput(887.00,442.17)(4.000,1.000){2}{\rule{0.964pt}{0.400pt}}
\put(895,444.17){\rule{1.500pt}{0.400pt}}
\multiput(895.00,443.17)(3.887,2.000){2}{\rule{0.750pt}{0.400pt}}
\put(902,445.67){\rule{1.686pt}{0.400pt}}
\multiput(902.00,445.17)(3.500,1.000){2}{\rule{0.843pt}{0.400pt}}
\put(909,446.67){\rule{1.686pt}{0.400pt}}
\multiput(909.00,446.17)(3.500,1.000){2}{\rule{0.843pt}{0.400pt}}
\put(916,448.17){\rule{1.500pt}{0.400pt}}
\multiput(916.00,447.17)(3.887,2.000){2}{\rule{0.750pt}{0.400pt}}
\put(923,449.67){\rule{1.927pt}{0.400pt}}
\multiput(923.00,449.17)(4.000,1.000){2}{\rule{0.964pt}{0.400pt}}
\put(931,450.67){\rule{1.686pt}{0.400pt}}
\multiput(931.00,450.17)(3.500,1.000){2}{\rule{0.843pt}{0.400pt}}
\put(938,452.17){\rule{1.500pt}{0.400pt}}
\multiput(938.00,451.17)(3.887,2.000){2}{\rule{0.750pt}{0.400pt}}
\put(945,453.67){\rule{1.686pt}{0.400pt}}
\multiput(945.00,453.17)(3.500,1.000){2}{\rule{0.843pt}{0.400pt}}
\put(952,454.67){\rule{1.927pt}{0.400pt}}
\multiput(952.00,454.17)(4.000,1.000){2}{\rule{0.964pt}{0.400pt}}
\put(960,456.17){\rule{1.500pt}{0.400pt}}
\multiput(960.00,455.17)(3.887,2.000){2}{\rule{0.750pt}{0.400pt}}
\put(967,457.67){\rule{1.686pt}{0.400pt}}
\multiput(967.00,457.17)(3.500,1.000){2}{\rule{0.843pt}{0.400pt}}
\put(974,458.67){\rule{1.686pt}{0.400pt}}
\multiput(974.00,458.17)(3.500,1.000){2}{\rule{0.843pt}{0.400pt}}
\put(981,459.67){\rule{1.686pt}{0.400pt}}
\multiput(981.00,459.17)(3.500,1.000){2}{\rule{0.843pt}{0.400pt}}
\put(988,461.17){\rule{1.700pt}{0.400pt}}
\multiput(988.00,460.17)(4.472,2.000){2}{\rule{0.850pt}{0.400pt}}
\put(227,292.67){\rule{1.686pt}{0.400pt}}
\multiput(227.00,292.17)(3.500,1.000){2}{\rule{0.843pt}{0.400pt}}
\put(234,293.67){\rule{1.686pt}{0.400pt}}
\multiput(234.00,293.17)(3.500,1.000){2}{\rule{0.843pt}{0.400pt}}
\put(241,295.17){\rule{1.700pt}{0.400pt}}
\multiput(241.00,294.17)(4.472,2.000){2}{\rule{0.850pt}{0.400pt}}
\put(249,296.67){\rule{1.686pt}{0.400pt}}
\multiput(249.00,296.17)(3.500,1.000){2}{\rule{0.843pt}{0.400pt}}
\put(256,297.67){\rule{1.686pt}{0.400pt}}
\multiput(256.00,297.17)(3.500,1.000){2}{\rule{0.843pt}{0.400pt}}
\put(263,299.17){\rule{1.500pt}{0.400pt}}
\multiput(263.00,298.17)(3.887,2.000){2}{\rule{0.750pt}{0.400pt}}
\put(270,300.67){\rule{1.686pt}{0.400pt}}
\multiput(270.00,300.17)(3.500,1.000){2}{\rule{0.843pt}{0.400pt}}
\put(277,301.67){\rule{1.927pt}{0.400pt}}
\multiput(277.00,301.17)(4.000,1.000){2}{\rule{0.964pt}{0.400pt}}
\put(285,303.17){\rule{1.500pt}{0.400pt}}
\multiput(285.00,302.17)(3.887,2.000){2}{\rule{0.750pt}{0.400pt}}
\put(292,304.67){\rule{1.686pt}{0.400pt}}
\multiput(292.00,304.17)(3.500,1.000){2}{\rule{0.843pt}{0.400pt}}
\put(299,305.67){\rule{1.686pt}{0.400pt}}
\multiput(299.00,305.17)(3.500,1.000){2}{\rule{0.843pt}{0.400pt}}
\put(306,306.67){\rule{1.686pt}{0.400pt}}
\multiput(306.00,306.17)(3.500,1.000){2}{\rule{0.843pt}{0.400pt}}
\put(313,308.17){\rule{1.700pt}{0.400pt}}
\multiput(313.00,307.17)(4.472,2.000){2}{\rule{0.850pt}{0.400pt}}
\put(321,309.67){\rule{1.686pt}{0.400pt}}
\multiput(321.00,309.17)(3.500,1.000){2}{\rule{0.843pt}{0.400pt}}
\put(328,310.67){\rule{1.686pt}{0.400pt}}
\multiput(328.00,310.17)(3.500,1.000){2}{\rule{0.843pt}{0.400pt}}
\put(335,312.17){\rule{1.500pt}{0.400pt}}
\multiput(335.00,311.17)(3.887,2.000){2}{\rule{0.750pt}{0.400pt}}
\put(342,313.67){\rule{1.927pt}{0.400pt}}
\multiput(342.00,313.17)(4.000,1.000){2}{\rule{0.964pt}{0.400pt}}
\put(350,314.67){\rule{1.686pt}{0.400pt}}
\multiput(350.00,314.17)(3.500,1.000){2}{\rule{0.843pt}{0.400pt}}
\put(357,316.17){\rule{1.500pt}{0.400pt}}
\multiput(357.00,315.17)(3.887,2.000){2}{\rule{0.750pt}{0.400pt}}
\put(364,317.67){\rule{1.686pt}{0.400pt}}
\multiput(364.00,317.17)(3.500,1.000){2}{\rule{0.843pt}{0.400pt}}
\put(371,318.67){\rule{1.686pt}{0.400pt}}
\multiput(371.00,318.17)(3.500,1.000){2}{\rule{0.843pt}{0.400pt}}
\put(378,320.17){\rule{1.700pt}{0.400pt}}
\multiput(378.00,319.17)(4.472,2.000){2}{\rule{0.850pt}{0.400pt}}
\put(386,321.67){\rule{1.686pt}{0.400pt}}
\multiput(386.00,321.17)(3.500,1.000){2}{\rule{0.843pt}{0.400pt}}
\put(393,322.67){\rule{1.686pt}{0.400pt}}
\multiput(393.00,322.17)(3.500,1.000){2}{\rule{0.843pt}{0.400pt}}
\put(400,323.67){\rule{1.686pt}{0.400pt}}
\multiput(400.00,323.17)(3.500,1.000){2}{\rule{0.843pt}{0.400pt}}
\put(407,325.17){\rule{1.500pt}{0.400pt}}
\multiput(407.00,324.17)(3.887,2.000){2}{\rule{0.750pt}{0.400pt}}
\put(414,326.67){\rule{1.927pt}{0.400pt}}
\multiput(414.00,326.17)(4.000,1.000){2}{\rule{0.964pt}{0.400pt}}
\put(422,327.67){\rule{1.686pt}{0.400pt}}
\multiput(422.00,327.17)(3.500,1.000){2}{\rule{0.843pt}{0.400pt}}
\put(429,329.17){\rule{1.500pt}{0.400pt}}
\multiput(429.00,328.17)(3.887,2.000){2}{\rule{0.750pt}{0.400pt}}
\put(436,330.67){\rule{1.686pt}{0.400pt}}
\multiput(436.00,330.17)(3.500,1.000){2}{\rule{0.843pt}{0.400pt}}
\put(443,331.67){\rule{1.927pt}{0.400pt}}
\multiput(443.00,331.17)(4.000,1.000){2}{\rule{0.964pt}{0.400pt}}
\put(451,333.17){\rule{1.500pt}{0.400pt}}
\multiput(451.00,332.17)(3.887,2.000){2}{\rule{0.750pt}{0.400pt}}
\put(458,334.67){\rule{1.686pt}{0.400pt}}
\multiput(458.00,334.17)(3.500,1.000){2}{\rule{0.843pt}{0.400pt}}
\put(465,335.67){\rule{1.686pt}{0.400pt}}
\multiput(465.00,335.17)(3.500,1.000){2}{\rule{0.843pt}{0.400pt}}
\put(472,337.17){\rule{1.500pt}{0.400pt}}
\multiput(472.00,336.17)(3.887,2.000){2}{\rule{0.750pt}{0.400pt}}
\put(479,338.67){\rule{1.927pt}{0.400pt}}
\multiput(479.00,338.17)(4.000,1.000){2}{\rule{0.964pt}{0.400pt}}
\put(487,339.67){\rule{1.686pt}{0.400pt}}
\multiput(487.00,339.17)(3.500,1.000){2}{\rule{0.843pt}{0.400pt}}
\put(494,340.67){\rule{1.686pt}{0.400pt}}
\multiput(494.00,340.17)(3.500,1.000){2}{\rule{0.843pt}{0.400pt}}
\put(501,342.17){\rule{1.500pt}{0.400pt}}
\multiput(501.00,341.17)(3.887,2.000){2}{\rule{0.750pt}{0.400pt}}
\put(508,343.67){\rule{1.686pt}{0.400pt}}
\multiput(508.00,343.17)(3.500,1.000){2}{\rule{0.843pt}{0.400pt}}
\put(515,344.67){\rule{1.927pt}{0.400pt}}
\multiput(515.00,344.17)(4.000,1.000){2}{\rule{0.964pt}{0.400pt}}
\put(523,346.17){\rule{1.500pt}{0.400pt}}
\multiput(523.00,345.17)(3.887,2.000){2}{\rule{0.750pt}{0.400pt}}
\put(530,347.67){\rule{1.686pt}{0.400pt}}
\multiput(530.00,347.17)(3.500,1.000){2}{\rule{0.843pt}{0.400pt}}
\put(537,348.67){\rule{1.686pt}{0.400pt}}
\multiput(537.00,348.17)(3.500,1.000){2}{\rule{0.843pt}{0.400pt}}
\put(544,350.17){\rule{1.700pt}{0.400pt}}
\multiput(544.00,349.17)(4.472,2.000){2}{\rule{0.850pt}{0.400pt}}
\put(552,351.67){\rule{1.686pt}{0.400pt}}
\multiput(552.00,351.17)(3.500,1.000){2}{\rule{0.843pt}{0.400pt}}
\put(559,352.67){\rule{1.686pt}{0.400pt}}
\multiput(559.00,352.17)(3.500,1.000){2}{\rule{0.843pt}{0.400pt}}
\put(566,353.67){\rule{1.686pt}{0.400pt}}
\multiput(566.00,353.17)(3.500,1.000){2}{\rule{0.843pt}{0.400pt}}
\put(573,355.17){\rule{1.500pt}{0.400pt}}
\multiput(573.00,354.17)(3.887,2.000){2}{\rule{0.750pt}{0.400pt}}
\put(580,356.67){\rule{1.927pt}{0.400pt}}
\multiput(580.00,356.17)(4.000,1.000){2}{\rule{0.964pt}{0.400pt}}
\put(588,357.67){\rule{1.686pt}{0.400pt}}
\multiput(588.00,357.17)(3.500,1.000){2}{\rule{0.843pt}{0.400pt}}
\put(595,359.17){\rule{1.500pt}{0.400pt}}
\multiput(595.00,358.17)(3.887,2.000){2}{\rule{0.750pt}{0.400pt}}
\put(602,360.67){\rule{1.686pt}{0.400pt}}
\multiput(602.00,360.17)(3.500,1.000){2}{\rule{0.843pt}{0.400pt}}
\put(609,361.67){\rule{1.686pt}{0.400pt}}
\multiput(609.00,361.17)(3.500,1.000){2}{\rule{0.843pt}{0.400pt}}
\put(616,363.17){\rule{1.700pt}{0.400pt}}
\multiput(616.00,362.17)(4.472,2.000){2}{\rule{0.850pt}{0.400pt}}
\put(624,364.67){\rule{1.686pt}{0.400pt}}
\multiput(624.00,364.17)(3.500,1.000){2}{\rule{0.843pt}{0.400pt}}
\put(631,365.67){\rule{1.686pt}{0.400pt}}
\multiput(631.00,365.17)(3.500,1.000){2}{\rule{0.843pt}{0.400pt}}
\put(638,367.17){\rule{1.500pt}{0.400pt}}
\multiput(638.00,366.17)(3.887,2.000){2}{\rule{0.750pt}{0.400pt}}
\put(645,368.67){\rule{1.927pt}{0.400pt}}
\multiput(645.00,368.17)(4.000,1.000){2}{\rule{0.964pt}{0.400pt}}
\put(653,369.67){\rule{1.686pt}{0.400pt}}
\multiput(653.00,369.17)(3.500,1.000){2}{\rule{0.843pt}{0.400pt}}
\put(660,370.67){\rule{1.686pt}{0.400pt}}
\multiput(660.00,370.17)(3.500,1.000){2}{\rule{0.843pt}{0.400pt}}
\put(667,372.17){\rule{1.500pt}{0.400pt}}
\multiput(667.00,371.17)(3.887,2.000){2}{\rule{0.750pt}{0.400pt}}
\put(674,373.67){\rule{1.686pt}{0.400pt}}
\multiput(674.00,373.17)(3.500,1.000){2}{\rule{0.843pt}{0.400pt}}
\put(681,374.67){\rule{1.927pt}{0.400pt}}
\multiput(681.00,374.17)(4.000,1.000){2}{\rule{0.964pt}{0.400pt}}
\put(689,376.17){\rule{1.500pt}{0.400pt}}
\multiput(689.00,375.17)(3.887,2.000){2}{\rule{0.750pt}{0.400pt}}
\put(696,377.67){\rule{1.686pt}{0.400pt}}
\multiput(696.00,377.17)(3.500,1.000){2}{\rule{0.843pt}{0.400pt}}
\put(703,378.67){\rule{1.686pt}{0.400pt}}
\multiput(703.00,378.17)(3.500,1.000){2}{\rule{0.843pt}{0.400pt}}
\put(710,380.17){\rule{1.500pt}{0.400pt}}
\multiput(710.00,379.17)(3.887,2.000){2}{\rule{0.750pt}{0.400pt}}
\put(717,381.67){\rule{1.927pt}{0.400pt}}
\multiput(717.00,381.17)(4.000,1.000){2}{\rule{0.964pt}{0.400pt}}
\put(725,382.67){\rule{1.686pt}{0.400pt}}
\multiput(725.00,382.17)(3.500,1.000){2}{\rule{0.843pt}{0.400pt}}
\put(732,383.67){\rule{1.686pt}{0.400pt}}
\multiput(732.00,383.17)(3.500,1.000){2}{\rule{0.843pt}{0.400pt}}
\put(739,385.17){\rule{1.500pt}{0.400pt}}
\multiput(739.00,384.17)(3.887,2.000){2}{\rule{0.750pt}{0.400pt}}
\put(746,386.67){\rule{1.686pt}{0.400pt}}
\multiput(746.00,386.17)(3.500,1.000){2}{\rule{0.843pt}{0.400pt}}
\put(753,387.67){\rule{1.686pt}{0.400pt}}
\multiput(753.00,387.17)(3.500,1.000){2}{\rule{0.843pt}{0.400pt}}
\put(760,389.17){\rule{1.500pt}{0.400pt}}
\multiput(760.00,388.17)(3.887,2.000){2}{\rule{0.750pt}{0.400pt}}
\put(767,390.67){\rule{1.686pt}{0.400pt}}
\multiput(767.00,390.17)(3.500,1.000){2}{\rule{0.843pt}{0.400pt}}
\put(774,391.67){\rule{1.686pt}{0.400pt}}
\multiput(774.00,391.17)(3.500,1.000){2}{\rule{0.843pt}{0.400pt}}
\put(781,393.17){\rule{1.700pt}{0.400pt}}
\multiput(781.00,392.17)(4.472,2.000){2}{\rule{0.850pt}{0.400pt}}
\put(789,394.67){\rule{1.686pt}{0.400pt}}
\multiput(789.00,394.17)(3.500,1.000){2}{\rule{0.843pt}{0.400pt}}
\put(796,395.67){\rule{1.686pt}{0.400pt}}
\multiput(796.00,395.17)(3.500,1.000){2}{\rule{0.843pt}{0.400pt}}
\put(803,397.17){\rule{1.500pt}{0.400pt}}
\multiput(803.00,396.17)(3.887,2.000){2}{\rule{0.750pt}{0.400pt}}
\put(810,398.67){\rule{1.686pt}{0.400pt}}
\multiput(810.00,398.17)(3.500,1.000){2}{\rule{0.843pt}{0.400pt}}
\put(817,399.67){\rule{1.927pt}{0.400pt}}
\multiput(817.00,399.17)(4.000,1.000){2}{\rule{0.964pt}{0.400pt}}
\put(825,400.67){\rule{1.686pt}{0.400pt}}
\multiput(825.00,400.17)(3.500,1.000){2}{\rule{0.843pt}{0.400pt}}
\put(832,402.17){\rule{1.500pt}{0.400pt}}
\multiput(832.00,401.17)(3.887,2.000){2}{\rule{0.750pt}{0.400pt}}
\put(839,403.67){\rule{1.686pt}{0.400pt}}
\multiput(839.00,403.17)(3.500,1.000){2}{\rule{0.843pt}{0.400pt}}
\put(846,404.67){\rule{1.927pt}{0.400pt}}
\multiput(846.00,404.17)(4.000,1.000){2}{\rule{0.964pt}{0.400pt}}
\put(854,406.17){\rule{1.500pt}{0.400pt}}
\multiput(854.00,405.17)(3.887,2.000){2}{\rule{0.750pt}{0.400pt}}
\put(861,407.67){\rule{1.686pt}{0.400pt}}
\multiput(861.00,407.17)(3.500,1.000){2}{\rule{0.843pt}{0.400pt}}
\put(868,408.67){\rule{1.686pt}{0.400pt}}
\multiput(868.00,408.17)(3.500,1.000){2}{\rule{0.843pt}{0.400pt}}
\put(875,410.17){\rule{1.500pt}{0.400pt}}
\multiput(875.00,409.17)(3.887,2.000){2}{\rule{0.750pt}{0.400pt}}
\put(882,411.67){\rule{1.927pt}{0.400pt}}
\multiput(882.00,411.17)(4.000,1.000){2}{\rule{0.964pt}{0.400pt}}
\put(890,412.67){\rule{1.686pt}{0.400pt}}
\multiput(890.00,412.17)(3.500,1.000){2}{\rule{0.843pt}{0.400pt}}
\put(897,413.67){\rule{1.686pt}{0.400pt}}
\multiput(897.00,413.17)(3.500,1.000){2}{\rule{0.843pt}{0.400pt}}
\put(904,415.17){\rule{1.500pt}{0.400pt}}
\multiput(904.00,414.17)(3.887,2.000){2}{\rule{0.750pt}{0.400pt}}
\put(911,416.67){\rule{1.686pt}{0.400pt}}
\multiput(911.00,416.17)(3.500,1.000){2}{\rule{0.843pt}{0.400pt}}
\put(918,417.67){\rule{1.927pt}{0.400pt}}
\multiput(918.00,417.17)(4.000,1.000){2}{\rule{0.964pt}{0.400pt}}
\put(926,419.17){\rule{1.500pt}{0.400pt}}
\multiput(926.00,418.17)(3.887,2.000){2}{\rule{0.750pt}{0.400pt}}
\put(933,420.67){\rule{1.686pt}{0.400pt}}
\multiput(933.00,420.17)(3.500,1.000){2}{\rule{0.843pt}{0.400pt}}
\put(864.0,470.0){\rule[-0.200pt]{1.686pt}{0.400pt}}
\put(449.0,365.0){\rule[-0.200pt]{0.400pt}{87.206pt}}
\end{picture}

\end{center}
\caption{Plotting the points from Table \ref{table_2d_numbers}}
\label{fig.simple_plane}
\end {figure}

Let us then try to predict the value of $y_n$ given the pair of attributes ($x_{u1}$, $x_{u2}$). We will follow a similar procedure to the one we followed for the 1-dimension case, but now we have two attributes to consider. From Figure \ref{fig.simple_plane} it is easy to see that you can obtain the value of $y_u$ by walking first in the $x_{*1}$  direction and then on the $x_{*2}$:

\begin{equation}
y_u= y_{00} +  x_{u1} \tan\alpha_1 + x_{u2} \tan\alpha_2
\label{eq.linear_combinations_of_inputs_vectors_2}
\end{equation}

Each of the dimension in which $\mathcal{X}$ spans is being treated independently: for computing the value of $y_u$ we simply ``add" the effect that each feature over value intersect value $y_{00}$. 

However, there is one little detail that we still have not addressed: how do we compute the tangents $\tan\alpha_1$ and $\tan\alpha_2$? As is is possible to see, in general the X points that we have access will have coordinate differences in all axis, so it is not straighforward to compute the tangents based only on values of the corresponding axis. In other words, $\tan\alpha_1$ and $\tan\alpha_2$ can \emph{not} be computed just by doing:
\begin{equation}
\tan\alpha_1 = \frac{y_2 - y_1}{x_{21} - x_{11}}
\label{eq.simple_tangent_2d_1}
\end{equation}
\begin{equation}
\tan\alpha_2 = \frac{y_2 - y_1}{x_{22} - x_{12}}
\label{eq.simple_tangent_2d_2t}
\end{equation}
since the change in the Y value (the numerator of both fraction) depends simultaneously on the difference over all dimensions of the X axis (in this case just two).

This is were the matrix notation is really poweful. Similarly to the 1-D case, we can easily transform these equation to the vector / matrix notation by making:
\begin{equation}
y_u =  
\begin{bmatrix}
1 &
x_{u1} &
x_{u2} &
\end{bmatrix}
\cdot
\begin{bmatrix}
y_{00} \\ 
\tan\alpha_1 \\
\tan\alpha_2
\end{bmatrix} 
=
1 \cdot y_{00} + x_{u1} \cdot \tan\alpha_1 + x_{u2} \cdot \tan\alpha_2
\end{equation}

If we then represent our dataset seen in Table \ref{table_2d_abstract} in such matrix notation, we obtain:
\begin{equation}
y_1 = 1 \cdot y_{00} + x_{11} \cdot \tan\alpha_1 + x_{12} \cdot \tan\alpha_2
\end{equation}
\begin{equation}
y_2 = 1 \cdot y_{00} + x_{21} \cdot \tan\alpha_1 + x_{22} \cdot \tan\alpha_2
\end{equation}
\begin{equation}
y_3 = 1 \cdot y_{00} + x_{31} \cdot \tan\alpha_1 + x_{32} \cdot \tan\alpha_2
\end{equation}

Packing $y_1$, $y_2$ and $y_3$ in a vertical Y vector, we can write:

\begin{equation}
Y = \begin{bmatrix}
y_1 \\
y_2
y_3
\end{bmatrix}
=
\begin{bmatrix}
1 \\
1 \\
1
\end{bmatrix}
\cdot y_{00}
+
\begin{bmatrix}
x_{11} \\
x_{21} \\
x_{31}
\end{bmatrix}
\cdot 
\tan\alpha_1
+
\begin{bmatrix}
x_{12} \\
x_{22} \\
x_{32}
\end{bmatrix}
\cdot 
\tan\alpha_2
\label{equation_comb_of_dimension_vectors_2_dim}
\end{equation}
which in a more compact way can be represented as:
\begin{equation}
\begin{bmatrix}
y_1 \\
y_2
\end{bmatrix}
=
\begin{bmatrix}
1 & x_{11} & x_{12}\\
1 & x_{21} & x_{22}\\
1 & x_{31} & x_{32}
\end{bmatrix}
\cdot
\begin{bmatrix}
y_{00} \\ 
\tan\alpha_1 \\
\tan\alpha_2 \\
\end{bmatrix} = X \cdot w
\end{equation}
Which can be solved by making:
\begin{equation}
w = X^{-1} Y
\end{equation}
or, more generically, by making:
\begin{equation}
w = (X^T X)^{-1} X^T Y 
\end{equation}
One interesting point of the matrix formulation is that it makes is very clear that, besides storing the intercept, the w vector is basically a vector containing the ''slopes" in each of the dimensions. Very informally, the w vector that contains information about how much Y changes per change in each of the dimensions of X and that becomes very obvious by looking at Equation \ref{slope_equation}: Y is being "divided" by X, i.e we have a Y per X ratio. This observation allows us to work n-dimensions while keeping some of the intuitions of the simple 1-d case.

\subsubsection{Show me the numbers}
\label{section_example_plane}
Let's go back to Figure \ref{fig.simple_plane} and to Table \ref{table_2d_numbers}. Using matrix notation, our X and Y are given by:

\begin{equation}
X =
\begin{bmatrix}
1 & 1 & 4\\
1 & 2 &-2\\
1 &6 & 2\\
\end{bmatrix}
\label{x_matrix_2d}
\end{equation}
and
\begin{equation}
Y = \begin{bmatrix}
1\\
6\\
12
\end{bmatrix}
\label{y_matrix_2d}
\end{equation}
To compute w we will use Equation XX:
\begin{equation}
w = X^{-1} Y =
\begin{bmatrix}
4/7 & 11/14 & -5/14\\
-1/7 & -1/14 &  3/14\\
1/7 & -5/28 &  1/28
\end{bmatrix}
\begin{bmatrix}
1\\
6\\
12
\end{bmatrix}
=
\begin{bmatrix}
1\\
2\\
-1/2
\end{bmatrix}
\label{equation_for_w_2d_example}
\end{equation}
These leads to the values of the tangents being $\tan\alpha_1 = 2$ and $\tan\alpha_2 = -1/2$, with the intersect being given by $y_{00}=1$.

\subsection{Adding one extra observation pair - $(y_3, x_{31})$}
As we have seen in the previous sections, we can define a 1-d plane using two data points, and in the 2-d case we can define the plane using just three points. In general, we need d+1 points to be able to find the d-dimension plane. So, what happens now if we add ``redundant" points in our dataset, that is, points that also lie in the plane defined by the d+1 previously given points  

Let us focus on the 1-d case, since we already saw that by using matrix notation we can think in 1-d and then easily generalize to more dimension. For example, lets assume that we add the point $(y_3, x_3)=(7,3)$ to table \ref{table_1d_example}. By looking at Figure \ref{fig.simple_line_with_values} it is possible to confirm that this point does in fact lie on the line already defined by the previous points. If we add this extra point, the X and Y matrixes become: 

\begin{equation}
Y =
\begin{bmatrix}
5\\
7\\
13
\end{bmatrix}
\end{equation}

\begin{equation}
X = 
\begin{bmatrix}
1&2\\
1&3\\
1&6\\
\end{bmatrix}
\end{equation}
The problem here is that is we want to compute w by doing
\begin{equation}
w = X^{-1} Y = 
\end{equation}
we will not be able to invert X since it X is not a square matrix. Clearly, X has enough information to allow us to compute the intercept and the tangent values that constitute the w vector since it stores 3 colinear points and we only need 2 points. It seems that we got stuck.

The way forward comes from a very simple observation. Only need to ''invert" X for the sake of computing w. We don't actually need to know $X^{-1}$, which can be seen as an intermediate result to get to our end (computing w!). So, let's get back to the Equation \ref{slope_equation}:
\begin{equation}
w = X^{-1} Y
\end{equation}
and see what we can do. So, let us assume that there is a matrix G which is non-singular (i.e. can be inverted). So, without changing this equility we can add the factors $G^{-1} G$ such that:
\begin{equation}
w = X^{-1} G^{-1} G Y
\end{equation}
By the rules of matrix inversion (see Section \ref{appendix_matrix_inversion}) we can transform this equation in:
\begin{equation}
w = (G X)^{-1} G Y 
\end{equation}
This suggest that if we find a matrix G that multiplied by X generates an invertible matrix, then we can compute we are back on track to computing w by using matrix calculus. So, what could that matrix G be?

It turns out that there is one very simple solution: $G = X^T$. If we make $G = X^T$, then we have:
\begin{equation}
w = (X^T X)^{-1} X^T Y 
\end{equation}
so we need to invert $X^T X$ which is now a (d+1) x (d+1) square matrix,(i.e. the number of dimensions d plus one extra ''dimension" for the intercept). This $X^T X$ matrix as a number of important properties (e.g. symmetric, positive-semidefinite) which we will detail in the next chapter but for now it suffices to say that  this matrix is non-singular, i.e. we can invert it. Thus, we elegantly solved our algebraic problem, and we can now compute w even if X has extra redundant points.

\subsection{Let's see the numbers!}
So our X matrix is now:
\begin{equation}
X = 
\begin{bmatrix}
1&2\\
1&3\\
1&6\\
\end{bmatrix}
\end{equation}
We can easily obtain $X^T$:
\begin{equation}
X^T=
\begin{bmatrix}
1& 1 &1\\
2& 3 &6\\
\end{bmatrix}
\end{equation}
so that the product $X^T X$ is:
\begin{equation}
X^T X=
\begin{bmatrix}
1& 1 &1\\
2& 3 &6\\
\end{bmatrix}
\begin{bmatrix}
1&2\\
1&3\\
1&6\\
\end{bmatrix}
= 
\begin{bmatrix}
3&11\\
11&49\\
\end{bmatrix}
\end{equation}
which is a 2 x 2 symmetric matrix (for dimension 1 and for the intercept). Now $X^T X$ is clearly invertible:
\begin{equation}
(X^T X)^{-1}=
\begin{bmatrix}
1.88461538 & -0.42307692\\
-0.42307692 &  0.11538462
\end{bmatrix}
\end{equation}
and is obviously a symmetric matrix too. The next step is to compute $(X^T X)^{-1} X^T$:
\begin{equation}
(X^T X)^{-1} X^T=
\begin{bmatrix}
1.88461538 & -0.42307692\\
-0.42307692 &  0.11538462
\end{bmatrix}
\begin{bmatrix}
1& 1 &1\\
2& 3 &6\\
\end{bmatrix}=
\begin{bmatrix}
1.03846154 &  0.61538462 & -0.65384615\\
-0.19230769 & -0.07692308 &  0.26923077
\end{bmatrix}
\end{equation}
At this point, we solved the core of our problem which was to compute the ''inverse" of X. We now have a 2 x 3 matrix which will allow us to compute the values of intercept and tangent by multiplying it with a 3 x 1 Y vector:
\begin{equation}
w = (X^T X)^{-1} X^T Y=
\begin{bmatrix}
1.03846154 &  0.61538462 & -0.65384615\\
-0.19230769 & -0.07692308 &  0.26923077
\end{bmatrix}
\begin{bmatrix}
5\\
7\\
13
\end{bmatrix}
= 
\begin{bmatrix}
1.\\ 
2.
\end{bmatrix}
\end{equation}
These values match exactly what we had obtained before. This makes sense: adding redundant information should not change the solution. But the key lesson is that matrix algebra provides us with a principled solution to processing ``redundant" datapoints by allowing us to invert a certain type of rectangular matrixes (more on this later).

\subsection{Conclusion}
In this chapter we introduced the matrix notation, and demonstrated the usefulness of using such notation in expressing relations that involve multi-dimensional data. The matrix notation allows us to express relations independently of the number of dimensions involved. In that way, it helps us apply the intuitions that we can more easily develop for the low dimension cases, which are usually easier to visualize or imagine, to cases where we are dealing with a very high number of dimension and where visualization (and intuition) becomes much harder.

The main matrix operation involved in our formulation has been inversion of matrix X where we store our observable features. Matrix inversion usually requires the matrix to invert to be square (and non-singular). We showed that adding ``redundant" points to our dataset and thus making the X matrix a tall rectangular matrix does not represent a problem in terms of matrix inversion since we can obtain the same effect of inverting, i.e. $X^{-1}$ by using $(X^T X)^{-1} X^T$, which only involves inverting a square matrix (we will have more on this later). 

As we have seen, the X matrix, which holds the information about the features of the objects we are observing, plays a center role in all the modeling we have been doing. Therefore, in the next chapter we will focus on deepening our intuition about the X matrix, and about several other related matrixes that are function of the X matrix.

\subsection{Exercises}

\textbf{Exercise 1.1:} Show that Equation \ref{intercept_equation_symbolic_2d_derivation} presented in Section does in fact correspond to intercept of the plane in the 2D case.
\\
\\
\textbf{Exercise 1.2:} Implement the example given in Section \ref{section_example_plane} using Python and the sympy package.




\pagebreak

